\documentclass{llncs}


\title{Wyner's Wire-Tap Channel, Forty Years Later}
\author{Leonid Reyzin}
\titlerunning{Forty Years of the Wire-Tap Channel}
\authorrunning{Leonid Reyzin}
\institute{Boston University\\Department of Computer Science\\Boston MA 02215 USA\\\url{http://www.cs.bu.edu/fac/reyzin}} 
\tocauthor{Leonid Reyzin (Boston University)}


\begin{document}
\maketitle

\begin{abstract} 
Wyner's information theory paper ``The Wire-Tap Channel'' turns forty this year. Its importance is underappreciated in cryptography, where its intellectual progeny includes pseudorandom generators, privacy amplification, information reconciliation, and many flavors of randomness extractors (including plain, strong, fuzzy, robust, and nonmalleable). Focusing mostly on privacy amplification and fuzzy extractors, this talk demonstrates the connection from Wyner's paper to today's research, including work on program obfuscation.
\end{abstract}
\end{document}
