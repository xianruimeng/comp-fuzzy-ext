\documentclass[11pt]{article}
%\documentclass{llncs}
\def\shownotes{1}

\usepackage[top=3cm, bottom=3cm, left=2cm, right=2cm]{geometry}      % [top=2cm, bottom=2cm, left=2cm, right=2cm]
\geometry{letterpaper}                   % ... or a4paper or a5paper or ... 
%\geometry{landscape}                % Activate for for rotated page geometry
%\usepackage[parfill]{parskip}    
\usepackage{graphicx}
\usepackage{amssymb, amsmath, amsfonts}
\usepackage{amsthm}
\usepackage{enumerate}
\usepackage{hyperref}
\usepackage{xspace}
\usepackage{graphicx}
\usepackage{latexsym}
\usepackage{color}
\usepackage{framed}
\usepackage{algpseudocode} 

\mathchardef\mhyphen="2D

\newcommand{\secref}[1]{\mbox{Section~\ref{#1}}}
\newcommand{\subsecref}[1]{\mbox{Subsection~\ref{#1}}}
\newcommand{\apref}[1]{\mbox{Appendix~\ref{#1}}}
\newcommand{\thref}[1]{\mbox{Theorem~\ref{#1}}}
\newcommand{\exref}[1]{\mbox{Example~\ref{#1}}}
\newcommand{\defref}[1]{\mbox{Definition~\ref{#1}}}
\newcommand{\corref}[1]{\mbox{Corollary~\ref{#1}}}
\newcommand{\lemref}[1]{\mbox{Lemma~\ref{#1}}}
\newcommand{\assref}[1]{\mbox{Assumption~\ref{#1}}}
\newcommand{\probref}[1]{\mbox{Problem~\ref{#1}}}
\newcommand{\clref}[1]{\mbox{Claim~\ref{#1}}}
\newcommand{\propref}[1]{\mbox{Proposition~\ref{#1}}}
\newcommand{\remref}[1]{\mbox{Remark~\ref{#1}}}
\newcommand{\consref}[1]{\mbox{Construction~\ref{#1}}}
\newcommand{\figref}[1]{\mbox{Figure~\ref{#1}}}
\DeclareMathOperator*{\expe}{\mathbb{E}}


\newcommand{\class}[1]{{\ensuremath{\mathsf{#1}}}}
\newcommand{\gen}{\ensuremath{\class{Gen}}\xspace}
\newcommand{\rep}{\ensuremath{\class{Rep}}\xspace}
\newcommand{\sketch}{\ensuremath{\class{SS}}\xspace}
\newcommand{\rec}{\ensuremath{\class{Rec}}\xspace}
\newcommand{\enc}{\ensuremath{\class{Enc}}\xspace}
\newcommand{\dec}{\ensuremath{\class{Dec}}\xspace}
\newcommand{\prg}{\ensuremath{\class{prg}}\xspace}
\newcommand{\zo}{\ensuremath{\{0, 1\}}}
\newcommand{\vect}[1]{\ensuremath{\mathbf{#1}}}
\newcommand{\zq}{\ensuremath{\mathbb{Z}_q}}
\newcommand{\Fq}{\ensuremath{\mathbb{F}_q}}
\newcommand{\sample}{\ensuremath{\class{Sample}}\xspace}
\newcommand{\neigh}{\ensuremath{\class{Neigh}}\xspace}
\newcommand{\dis}{\ensuremath{\mathsf{dis}}}
\newcommand{\decode}{\ensuremath{\mathsf{Decode}}}

\newcommand{\A}{\mathcal{A}}


\newcommand{\metric}{\ensuremath{\mathtt{Metric}}\xspace}
\newcommand{\hill}{\ensuremath{\mathtt{HILL}}\xspace}
\newcommand{\hillrlx}{\ensuremath{\mathtt{HILL\mhyphen rlx}}\xspace}
\newcommand{\yao}{\ensuremath{\mathtt{Yao}}\xspace}
\newcommand{\unp}{\ensuremath{\mathtt{unp}}\xspace}
\newcommand{\unprlx}{\ensuremath{\mathtt{unp\mhyphen rlx}}\xspace}
\newcommand{\metricstar}{\ensuremath{\mathtt{Metric}^*}\xspace}
\newcommand{\metricd}{\ensuremath{\mathtt{Metric}^*\mathtt{-d}}\xspace}
\newcommand{\hillstar}{\ensuremath{\mathtt{HILL}^*}\xspace}
\newcommand{\hillprime}{\ensuremath{\mathtt{HILL'}}\xspace}
\newcommand{\metricprime}{\ensuremath{\mathtt{Metric'}}\xspace}
\newcommand{\metricprimestar}{\ensuremath{\mathtt{Metric'}^*}\xspace}
\newcommand{\hillprimestar}{\ensuremath{\mathtt{HILL'}^*}\xspace}
\newcommand{\poly}{\ensuremath{\mathtt{poly}}\xspace}
\newcommand{\rank}{\ensuremath{\mathtt{rank}}\xspace}
\newcommand{\ngl}{\ensuremath{\mathtt{ngl}}\xspace}
\newcommand{\Hoo}{\mathrm{H}_\infty}
\newcommand{\Hav}{\tilde{\mathrm{H}}_\infty}
\newcommand{\Dom}{\mathsl{Dom}}
\newcommand{\Range}{\mathsl{Rng}}
\newcommand{\Keys}{\mathsl{Keys}}
\def\col{\mathrm{Col}}

\newcommand{\ddetbin}{\ensuremath{\mathcal{D}^{det,\{0,1\}}}}
\newcommand{\drandbin}{\ensuremath{\mathcal{D}^{rand,\{0,1\}}}}
\newcommand{\ddetrange}{\ensuremath{\mathcal{D}^{det,[0,1]}}}
\newcommand{\drandrange}{\ensuremath{\mathcal{D}^{rand,[0,1]}}}

\newcommand{\expinfo}{\ensuremath{\mathcal{E}}}
\newcommand{\ext}{\ensuremath{\mathtt{ext}}}
\newcommand{\rext}{\ensuremath{\mathtt{rext}}}
\newcommand{\cons}{\ensuremath{\mathtt{cons}}}
\newcommand{\decons}{\ensuremath{\mathtt{decons}}}


\newcommand{\lwe}{\class{LWE}}
\newcommand{\LWE}{\class{LWE}}
\newcommand{\distLWE}{\ensuremath{\class{dist\mbox{-}LWE}}}

\newtheorem{theorem}{Theorem}[section]
\newtheorem{lemma}[theorem]{Lemma}
\newtheorem{proposition}[theorem]{Proposition}
\newtheorem{corollary}[theorem]{Corollary}
\newtheorem{definition}[theorem]{Definition}
\newtheorem{assumption}[theorem]{Assumption}
\newtheorem{claim}[theorem]{Claim}
\newtheorem{problem}[theorem]{Problem}
\newtheorem{construction}[theorem]{Construction}

\newcounter{ctr}
\newcounter{savectr}
\newcounter{ectr}

\newenvironment{newitemize}{%
\begin{list}{\mbox{}\hspace{5pt}$\bullet$\hfill}{\labelwidth=15pt%
\labelsep=5pt \leftmargin=20pt \topsep=3pt%
\setlength{\listparindent}{\saveparindent}%
\setlength{\parsep}{\saveparskip}%
\setlength{\itemsep}{3pt} }}{\end{list}}


\newenvironment{newenum}{%
\begin{list}{{\rm (\arabic{ctr})}\hfill}{\usecounter{ctr} \labelwidth=17pt%
\labelsep=5pt \leftmargin=22pt \topsep=3pt%
\setlength{\listparindent}{\saveparindent}%
\setlength{\parsep}{\saveparskip}%
\setlength{\itemsep}{2pt} }}{\end{list}}

\newenvironment{tiret}{%
\begin{list}{\hspace{2pt}\rule[0.5ex]{6pt}{1pt}\hfill}{\labelwidth=15pt%
\labelsep=3pt \leftmargin=22pt \topsep=3pt%
\setlength{\listparindent}{\saveparindent}%
\setlength{\parsep}{\saveparskip}%
\setlength{\itemsep}{2pt}}}{\end{list}}


\newenvironment{blocklist}{\begin{list}{}{\labelwidth=0pt%
\labelsep=0pt \leftmargin=0pt \topsep=10pt%
\setlength{\listparindent}{\saveparindent}%
\setlength{\parsep}{\saveparskip}%
\setlength{\itemsep}{20pt}}}{\end{list}}

\newenvironment{blocklistindented}{\begin{list}{}{\labelwidth=0pt%
\labelsep=30pt \leftmargin=30pt\topsep=5pt%
\setlength{\listparindent}{\saveparindent}%
\setlength{\parsep}{\saveparskip}%
\setlength{\itemsep}{10pt}}}{\end{list}}

\newenvironment{onelist}{%
\begin{list}{{\rm (\arabic{ctr})}\hfill}{\usecounter{ctr} \labelwidth=18pt%
\labelsep=7pt \leftmargin=25pt \topsep=2pt%
\setlength{\listparindent}{\saveparindent}%
\setlength{\parsep}{\saveparskip}%
\setlength{\itemsep}{2pt} }}{\end{list}}

\newenvironment{twolist}{%
\begin{list}{{\rm (\arabic{ctr}.\arabic{ectr})}%
\hfill}{\usecounter{ectr} \labelwidth=26pt%
\labelsep=7pt \leftmargin=33pt \topsep=2pt%
\setlength{\listparindent}{\saveparindent}%
\setlength{\parsep}{\saveparskip}%
\setlength{\itemsep}{2pt} }}{\end{list}}

\newenvironment{centerlist}{%
\begin{list}{\mbox{}}{\labelwidth=0pt%
\labelsep=0pt \leftmargin=0pt \topsep=10pt%
\setlength{\listparindent}{\saveparindent}%
\setlength{\parsep}{\saveparskip}%
\setlength{\itemsep}{10pt} }}{\end{list}}

\newenvironment{newcenter}[1]{\begin{centerlist}\centering%
\item #1}{\end{centerlist}}

\newenvironment{codecenter}[1]{\begin{small}\begin{centerlist}\centering%
\item #1}{\end{centerlist}\end{small}}

\ifnum\shownotes=1
\newcommand{\authnote}[2]{{\textcolor{red}{\textsf{#1 notes: }\textcolor{blue}{ #2}}\marginpar{\textcolor{red}{\textbf{!!!!!}}}}}
\else
\newcommand{\authnote}[2]{}
\fi
\newcommand{\bnote}[1]{{\authnote{Ben}{#1}}}
\newcommand{\lnote}[1]{{\authnote{Leo}{#1}}}
\newcommand{\xnote}[1]{{\authnote{Xianrui}{#1}}}

\newcommand{\ve}{\vect{e}}
\newcommand{\vm}{\vect{m}}
\newcommand{\vy}{\vect{y}}
\newcommand{\vE}{\vect{E}}
\newcommand{\vS}{\vect{S}}
\newcommand{\vA}{\vect{A}}
\newcommand{\vc}{\vect{c}}
\newcommand{\vW}{\vect{W}}
\newcommand{\vQ}{\vect{Q}}
\newcommand{\vR}{\vect{R}}
\newcommand{\vU}{\vect{U}}
\newcommand{\vT}{\vect{T}}
\newcommand{\vX}{\vect{X}}
\newcommand{\vB}{\vect{B}}
\newcommand{\vz}{\vect{z}}
\newcommand{\vd}{\vect{d}}
\newcommand{\vs}{\vect{s}}
\newcommand{\vx}{\vect{x}}
\newcommand{\va}{\vect{a}}
\newcommand{\vb}{\vect{b}}
\newcommand{\vgamma}{\mathbf{\Gamma}}
\newcommand{\vt}{\vect{t}}
\newcommand{\vu}{\vect{u}}
\newcommand{\vF}{\vect{F}}
\newcommand{\recout}{x}
\newcommand{\ignore}[1]{}
\newcommand{\M}{\mathcal{M}}

\title{\textbf{Computational Fuzzy Extractors}}
\author{Benjamin Fuller 
\footnote{Email: {\tt bfuller@cs.bu.edu}.  Work done in part while the author was at MIT Lincoln Laboratory.}
~~~~~~~~Xianrui Meng\footnote{Email: {\tt xmeng@cs.bu.edu}.}~~~~~~~~~Leonid Reyzin\footnote{Email: {\tt reyzin@cs.bu.edu}.} \\ Boston University}
%\author{}
\begin{document}
\maketitle


\begin{abstract} 
Fuzzy extractors derive strong keys from noisy sources.  Their
security is defined information-theoretically, which limits the length
of the derived key, sometimes making it too short to be useful. We ask
whether it is possible to obtain longer keys by considering
computational security, and show the following.

\begin{itemize}
\item\textbf{Negative Result:} Noise tolerance in fuzzy extractors is usually
achieved using an information reconciliation component called a ``secure
sketch.'' The security of this component, which directly affects the
length of the resulting key, is subject to lower bounds from coding
theory.  We show that, even when defined computationally, secure
sketches are still subject to lower bounds from coding theory. Specifically, we consider
two computational relaxations of the information-theoretic security requirement of secure sketches, using conditional HILL entropy and unpredictability entropy. For both cases we show  that computational secure sketches cannot outperform the best information-theoretic secure sketches in the case of high-entropy Hamming metric sources.

\item\textbf{Positive Result:} We show that the negative result can be overcome by
analyzing computational fuzzy extractors directly.  Namely, we show
how to build a computational fuzzy extractor whose output key length
equals the entropy of the source (this is impossible in the
information-theoretic setting). Our construction is based on the
hardness of the Learning with Errors (LWE) problem, and is secure when
the noisy source is uniform or symbol-fixing (that is, each dimension
is either uniform or fixed). As part of the security proof, we show a result of independent interest, namely
that the decision version of LWE is secure even when a small number of
dimensions has no error.
\end{itemize}
\textbf{Keywords} Fuzzy extractors, secure sketches, key derivation, Learning with Errors, error-correcting codes, computational entropy, randomness extractors.
\end{abstract}


\section{Introduction}\label{sec:introduction}

Authentication generally requires a secret drawn from some high-entropy source.  One of the primary building blocks for authentication is reliable key derivation.  Unfortunately, many sources that contain sufficient entropy to derive a key are  noisy, and provide similar, but not identical secret values at each reading (examples of such sources include biometrics~\cite{daugman2004}, human memory~\cite{zviran1993comparison}, pictorial passwords~\cite{brostoff2000passfaces}, measurements of capacitance~\cite{tuyls2006puf}, timing~\cite{suh2007physical}, motion~\cite{castelluccia2005shake},  quantum information~\cite{bennett1988privacy} etc.).  %Information reconciliation protocols~\cite{bennett1988privacy} remove the noise without revealing the secrets.

Fuzzy extractors~\cite{DBLP:journals/siamcomp/DodisORS08} achieve reliable key derivation from noisy sources~(see \cite{Boyen05secureremote,dodisWichs2009,chandran2010privacy} for applications of fuzzy extractors).  The setting 
%We will specifically focus on a one-round information reconciliation  mechanism called ``secure sketch,''  which, aside from its immediate application to authentication, is also
%a crucial ingredient in the construction of fuzzy extractors~\cite{DBLP:journals/siamcomp/DodisORS08} (as well as in other contexts, e.g.,~\cite{Boyen05secureremote,dodisWichs2009,chandran2010privacy}).
consists of  two algorithms: Generate (used once) and Reproduce (used subsequently).  The Generate ($\gen$) algorithm takes an input $w$ and produces a key $r$ and a public value $p$.  This information allows
the Reproduce ($\rep$) algorithm to reproduce $r$ given $p$ and some value $w'$ that is close to $w$ (according to some predefined metric, such as Hamming distance). 
Crucially for security,  knowledge of $p$ should not reveal $r$; that is, $r$ should be uniformly distributed conditioned on $p$.  This feature is needed because $p$ is not secret: for example, in a single-user setting (where the user wants to reproduce the key $r$ from a subsequent reading $w'$), it would be stored in the clear; and in a key agreement application~\cite{Boyen05secureremote} (where two parties have $w$ and $w'$, respectively), it would be transmitted between the parties.

%As defined in \cite{DBLP:journals/siamcomp/DodisORS08}, 
Fuzzy extractors use ideas from information-reconciliation~\cite{bennett1988privacy} and are defined (traditionally) as information-theoretic objects.  The entropy loss of a fuzzy extractor is the difference between the entropy of $w$ and the length of the derived key $r$.  In the information-theoretic setting, some entropy loss is necessary as the value $p$ contains enough information to reproduce $r$ from any close value $w'$. 
A goal of fuzzy extractor constructions is to minimize the entropy loss, increasing the security of the resulting application.  Indeed, if the entropy loss is too high, the resulting secret key may be too short to be useful. 

We ask whether it is possible to obtain longer keys by considering
computational, rather than information theoretic, security.

\paragraph {Our Negative Results}
We first study (in  \secref{sec:impossCompSecSketch}) whether it could be fruitful to relax the definition of the main building block of a fuzzy extractor, called a \emph{secure sketch}.  A secure sketch is a one-round information reconciliation protocol: it  produces a public value $s$ that allows recovery of $w$ from any close value $w'$.  The traditional secrecy requirement of a secure sketch is that $w$ has high min-entropy conditioned on $s$.  This allows the fuzzy extractor of~\cite{DBLP:journals/siamcomp/DodisORS08}  to form the key $r$ by applying a randomness extractor~\cite{nisan1993randomness} to $w$, because randomness extractors produce random strings from strings with conditional min-entropy. We call this the \emph{sketch-and-extract} construction.

The most natural relaxation of the min-entropy requirement of the secure sketch is to require HILL entropy~\cite{DBLP:journals/siamcomp/HastadILL99}~(namely, that the distribution of $w$ conditioned on $s$ be \emph{indistinguishable} from a high-min-entropy distribution).  Under this definition, we could still use a randomness extractor to obtain $r$ from $w$, because it would yield a pseudorandom key.  Unfortunately, it is unlikely that such a relaxation will yield fruitful results: we prove in Theorem~\ref{thm:impSketchArbitraryW} that the entropy loss of such secure sketches is subject to the same coding bounds as the ones that constrain information-theoretic secure sketches.  
%Furthermore, we have (asymptotic)~constructions that meet these bounds.

Another possible relaxation is to require that the value $w$ is unpredictable conditioned on $s$. This definition would also allow the use of a randomness extractor to get a pseudorandom key, although it would have to be a special extractor---one that has  a reconstruction procedure (see \cite[Lemma 6]{DBLP:conf/eurocrypt/HsiaoLR07}).  Unfortunately, this relaxation is also unlikely to be fruitful:  we prove in \thref{thm:imp of unp entropy} that the unpredictability is at most $\log$ the size of the metric space minus $\log$ the volume of the ball of radius $t$.  For high-entropy sources of $w$ over the Hamming metric, this bound matches the best information-theoretic security sketches.




%%Secret entropy of convenient sources is often at a premium; a goal of secure sketch constructions is to minimize the entropy loss, increasing the security of the resulting application. When used in a fuzzy extractor, the resulting secret key may be too short to be useful if the entropy loss is too high. 
%The minimum entropy loss for secure sketches has been quantified in past work.  Even if $\rep$ is allowed to produce an incorrect value with small probability~\cite[Section 8]{DBLP:journals/siamcomp/DodisORS08}, 
%%Even if we don't require perfect correctness (i.e., allow $\rep$ to err with small probability~\cite[Section 8]{DBLP:journals/siamcomp/DodisORS08}), 
%the entropy loss of a secure sketch that can reconcile $t$ errors
% %with high probability 
% is bounded by the redundancy of
%the best code that can correct $t$ random errors with high probability~\cite[Proposition 8.2]{DBLP:journals/siamcomp/DodisORS08}.  Constructions of fuzzy extractors that use a secure sketch inherit this bound.
%%(the argument is the same as in \cite[Lemma C.1]{DBLP:journals/siamcomp/DodisORS08}\lnote{probably we should double-check that this is correct}).



%to focus on producing a consistent string $\recout$  each time $\rec(w', s)$ is called.  This string $\recout$ is  required to have high HILL entropy even conditioned on $s$. Because this object \emph{conducts} the entropy from $w$ to $\recout$ and tolerates noise in its input, following \cite{CRVW02,KanukurthiR09}, we call it a \emph{computational fuzzy conductor}.


\paragraph {Our Positive Results}

Both of the above negative results arise because a secure sketch functions like a decoder of an error-correcting code.  To avoid them, we give up on building computational secure sketches and focus directly on the entropy loss in fuzzy extractors.  Our goal is to decrease the entropy loss in a fuzzy extractor by allowing the key $r$ to be pseudorandom conditioned on $p$.  

By considering this computational secrecy requirement, we construct the first \emph{lossless} computational fuzzy extractors (\consref{cons:informal construction}), where the derived key $r$ is as long as the entropy of the source $w$. Our construction is for the Hamming metric and uses the code-offset construction~\cite{JW99},\cite[Section 5]{DBLP:journals/siamcomp/DodisORS08} used in prior work, but with two crucial differences.  First, the key $r$ is not extracted from $w$ like in the sketch-and-extract approach; rather $w$ ``encrypts'' $r$ in a way that is decryptable with the knowledge of some close $w'$ (this idea is similar to the way the code-offset construction is presented in 
\cite{JW99} as  a ``fuzzy commitment''). Our construction uses private randomness, which is allowed in the fuzzy extractor setting but not in noiseless randomness extraction.  Second, the code used is a random linear code, which allows us to use the Learning with Errors~(LWE) assumption due to Regev~\cite{regev2005LWE, regevLWEsurvey} and derive a longer key $r$.


Specifically,  we use the  recent result of D\"{o}ttling and M\"{u}ller-Quade~\cite{dottling2012}, which shows the hardness of decoding random linear codes when the error vector comes from the uniform distribution, with each coordinate ranging over a small interval. This allows us to use $w$ as the error vector, assuming it is uniform.  We also use a result of Akavia, Goldwasser, and Vaikuntanathan~\cite{akavia2009},  which says that LWE has many hardcore bits, to hide $r$.

Because we use a random linear code,  our decoding is limited to reconciling a logarithmic number of differences.  Unfortunately, we  cannot utilize the results that improve the decoding radius through the use of trapdoors (such as \cite{regev2005LWE}), because in a fuzzy extractor, there is no secret storage place for the trapdoor. If improved decoding algorithms are obtained for random linear codes, they will improve error-tolerance of our construction.  Given the hardness of decoding random linear codes~\cite{berlekamp1978}, we do not expect significant improvement in error-tolerance of our construction.

%We show that this relaxation of the security requirement is  useful.  Specifically, we also define the notion of computational fuzzy extractors, in which the extracted key is required to be pseudorandom---i.e., to look random to any computationally bounded observer.
%We show that our relaxed notion of secure sketches is sufficient to construct computational  fuzzy extractors via a simple and efficient construction, utilizing reconstructive extractors~\cite{barak-computational}.  The resulting fuzzy extractor produces a key that is as long as the entropy of $w$, minus only the loss necessary in any extractor construction.
%\lnote{it would be good to find another application} 

In \secref{sec:LWE block fixing sources}, we are able to relax the assumption that $w$ comes from the uniform distribution, and instead allow $w$ to come from a symbol-fixing source~\cite{KZ07} (each dimension is either uniform or fixed). This relaxation follows from our results about the hardness of LWE when samples have a fixed~(and adversarially known) error vector, which may be of independent interest~(\thref{thm:blockLWE}). 

\paragraph{An Alternative Approach}  Computational extractors~\cite{krawczyk2010cryptographic, barak2011leftover, dachman2012computational} have the same goal of obtaining a pseudorandom key $r$ from a source $w$ in the setting without errors.  They can be constructed, for example, by applying a pseudorandom generator to the output of an information-theoretic extractor.  One way to build a computational \emph{fuzzy} extractor is by using a computational extractor instead  of the information-theoretic extractor in the sketch-and-extract construction of  \cite{DBLP:journals/siamcomp/DodisORS08}.   However, this approach is possible only if conditional min-entropy of $w$ conditioned on the sketch $s$ is high enough.  Furthermore, this approach does not allow the use of private randomness; private randomness is a crucial ingredient in our construction.
%We ask whether it is possible to avoid this loss by making whole process ``computational.''
We compare the two approaches in \secref{sec:prg based comparison}.


%Authentication generally requires secrets.  Unfortunately, many useful  sources that contain sufficient entropy for a secret  are also noisy, and provide similar, but not identical secret values at each invocation (examples of such sources include biometrics~\cite{daugman2004}, human memory~\cite{zviran1993comparison}, pictorial passwords~\cite{brostoff2000passfaces}, measurements of capacitance~\cite{tuyls2006puf}, timing~\cite{suh2007physical}, motion~\cite{castelluccia2005shake},  quantum information~\cite{bennett1988privacy} etc.).  Information reconciliation protocols~\cite{bennett1988privacy} remove the noise without revealing the secrets.
%
%We will specifically focus on a one-round information reconciliation  mechanism called ``secure sketch,''  which, aside from its immediate application to authentication, is also
%a crucial ingredient in the construction of fuzzy extractors~\cite{DBLP:journals/siamcomp/DodisORS08} (as well as in other contexts, e.g.,~\cite{Boyen05secureremote,dodisWichs2009,chandran2010privacy}).
%It consists of  two algorithms: Sketch (used once) and Recover (used subsequently).  The Sketch ($\sketch$) algorithm takes an input $w$ and produces a sketch $s$.  This information allows
%the Recover ($\rec$) algorithm to recover $w$ given $s$ and some value $w'$ that is close (according to some predefined metric, such as Hamming distance) to $w$. 
%Crucially for security,  knowledge of $s$ should not reveal $w$; that is, $w$ should still have entropy  conditioned on $s$.  This feature is needed because $s$ is not secret: for example, in a single-user setting (where the user wants to recover the original reading $w$ from a subsequent reading $w'$), it would be stored in the clear; and in a key agreement application~\cite{Boyen05secureremote} (where two parties have $w$ and $w'$, respectively), it would be transmitted between the parties.
%As defined in \cite{DBLP:journals/siamcomp/DodisORS08}, the entropy loss of a secure sketch is the difference in the entropy of $w$ and the entropy of $w$ conditioned on $s$.  
%
%Secret entropy of convenient sources is often at a premium; a goal of secure sketch constructions is to minimize the entropy loss, increasing the security of the resulting application. For example, in the application of secure sketches to fuzzy extractors (whose goal is to reliably produce a uniformly random secret key from a noisy input), the resulting secret key may be too short to be useful if the entropy loss is too high. However, because secure sketches are defined as information-theoretic objects, some entropy loss is inherent in any secure sketch construction, because $s$ contains enough information to recover $w$ from any value $w'$ that is close to it.  
%
%The minimum necessary  entropy loss has been quantified in past work.  Even if we don't require perfect correctness (i.e., allow $\rec$ to err with small probability~\cite[Section 8]{DBLP:journals/siamcomp/DodisORS08}), the entropy loss of a secure sketch that can reconcile $t$ errors with high probability is bounded by the redundancy of
%the best code that can correct $t$ random errors with high probability~\cite[Proposition 8.2]{DBLP:journals/siamcomp/DodisORS08}
%%(the argument is the same as in \cite[Lemma C.1]{DBLP:journals/siamcomp/DodisORS08}\lnote{probably we should double-check that this is correct}).
%
%\paragraph {Our Contributions}
%To see if entropy loss can be improved, we explore secure sketches with the relaxed secrecy requirement that is computational rather than information-theoretic.  
%The natural relaxation of the secrecy requirement is to require HILL entropy~\cite{DBLP:journals/siamcomp/HastadILL99}~(namely, that the distribution of $w$ conditioned on $s$ be \emph{indistinguishable} from a high-min-entropy distribution).  Under this definition, using the paradigm for constructing fuzzy extractors from~\cite{DBLP:journals/siamcomp/DodisORS08} would yield a pseudorandom key.  
%
%On the negative side, we show defining sketches using HILL entropy is unlikely to be fruitful. We prove in Theorem~\ref{thm:impSketchArbitraryW} that the entropy loss of such secure sketches is subject to the same coding bounds as the ones that constrain information-theoretic secure sketches.
%%Furthermore, we have (asymptotic)~constructions that meet these bounds.
%
%On the positive side, we show that a different relaxation of the definition can in fact lead to a \emph{lossless} construction. The idea is to give up on reconstructing $w$ and, instead, to focus on producing a consistent string $\recout$  each time $\rec(w', s)$ is called.  This string $\recout$ is  required to have high HILL entropy even conditioned on $s$. Because this object \emph{conducts} the entropy from $w$ to $\recout$ and tolerates noise in its input, following \cite{CRVW02,KanukurthiR09}, we call it a \emph{computational fuzzy conductor}.
%
%Given this relaxation, we are able to build \emph{lossless} computational fuzzy conductors, in which the HILL entropy of $\recout$ conditioned on $s$ is the same as the entropy of $w$ without conditioning on anything---i.e., before $\sketch$ is run (see Theorem~%s~%\ref{thm:lossless secure sketch} and
%~\ref{thm:lossless secure extractor log})).  We demonstrate how to build such lossless conductors based on the Learning with Errors~(LWE) assumption due to Regev~\cite{regev2005LWE, regevLWEsurvey},  which says that decoding a random linear code is computationally difficult. 
%
%We show that this relaxation of the security requirement is  useful.  Specifically, we also define the notion of computational fuzzy extractors, in which the extracted key is required to be pseudorandom---i.e., to look random to any computationally bounded observer.
%We show that our relaxed notion of secure sketches is sufficient to construct computational  fuzzy extractors via a simple and efficient construction, utilizing reconstructive extractors~\cite{barak-computational}.  The resulting fuzzy extractor produces a key that is as long as the entropy of $w$, minus only the loss necessary in any extractor construction.
%\lnote{it would be good to find another application} 
%
%Both our lossless fuzzy conductor and the resulting fuzzy extractor are built via the code-offset construction~\cite{JW99},\cite[Section 5]{DBLP:journals/siamcomp/DodisORS08} already used in prior work.  To be able to apply the LWE assumption, we use a random linear code (unfortunately, our decoding algorithm can only reconcile a logarithmic number of differences).  We utilize the recent result of D\"{o}ttling and M\"{u}ller-Quade~\cite{dottling2012} that shows security of LWE when the error vector comes from the uniform distribution, with each coordinate ranging over a small interval.  The LWE error vector, in our case, is the input $w$ itself.  To utilize the result of~\cite{dottling2012}, we need to assume that the input comes from the uniform distribution. We are able to relax this limitation somewhat and allow $w$ to come from a symbol-fixing source~\cite{KZ07} (where each dimension is either uniform or fixed). This relaxation requires new results about the hardness of LWE when samples have a fixed error vector, which may be of independent interest~(\thref{thm:blockLWE}).
\section{Preliminaries}
\label{sec:preliminaries}
For a random variable $X = X_1||...|| X_n$ where each $X_i$ is over some alphabet $\mathcal{Z}$, we denote by $X_{1,..., k} = X_1||...|| X_k$.  The {\em min-entropy} of $X$ is $\Hoo(X) = -\log(\max_x \Pr[X=x])$, 
%the {\em (worst-case) conditional min-entropy} of $X$ given $Y$ is  $\Hoo(X|Y) = -\log(\max_{x,y} \Pr[X=x|Y=y])$, and
and the {\em average (conditional)} min-entropy of $X$ given $Y$ is  $\Hav(X|Y) = -\log(\expe_{y\in Y} \max_{x} \Pr[X=x|Y=y])$~\cite[Section 2.4]{DBLP:journals/siamcomp/DodisORS08}.  
%The {\em collision probability} of  $X$ is 
%$\col(X) = \sum_{x} \Pr[X=x]^2$.
The {\em statistical distance} between random variables $X$ and $Y$ with the same domain is $\Delta(X,Y) = \frac12 \sum_x |\Pr[X=x] - \Pr[Y=x]|$. %We write $X \approx_{\epsilon} Y$ if $\Delta(X,Y) \leq \epsilon$, and when $\epsilon$ is negligible (in the appropriate parameter, as clear from the context) then we say $X$ and $Y$ are \emph{statistically close}.  
For a distinguisher $D$~(or a class of distinguishers $\mathcal{D}$) we write the \emph{computational distance} between $X$ and $Y$ as $\delta^D(X,Y) = \left| \expe[D(X)]-\expe[D(Y)]\right |$.  We denote by $\mathcal{D}_{s_{sec}}$ the class of randomized circuits which output a single bit and have size at most $s_{sec}$.
For a metric space $(\mathcal{M}, \dis)$, the \emph{(closed) ball of radius $t$ around $x$} is the set of all points within radius $t$, that is, $B_t(x) = \{y| \dis(x, y)\leq t\}$.  If the size of a ball in a metric space does not depend on $x$, we denote by $|B_t(\cdot)|$ the size of a ball of radius $t$.  For the Hamming metric over $\mathcal{Z}^n$, $|B_t(\cdot)| = \sum_{i=0}^t {n \choose t} (|\mathcal{Z}|-1)^i $.  $U_n$ denotes the uniformly  distributed random variable on $\{0,1\}^n$.
Usually, we use bold letters for vectors or matrices, capitalized letters for random variables, and lowercase letters for elements in a vector or samples from a random variable. 

\subsection{Fuzzy Extractors and Secure Sketches}
\label{sec:fuzzy extractors}

We now recall definitions and lemmas from the work of Dodis et. al.~\cite[Sections 2.5--4.1]{DBLP:journals/siamcomp/DodisORS08}, adapted to allow for a small probability of error, as discussed in \cite[Sections 8]{DBLP:journals/siamcomp/DodisORS08}.  Let $\mathcal{M}$ be a metric space with distance function $\dis$.% \lnote{$d$ is overloaded -- it's also a seed length for an extractor and also a constant in the parameter setting.  can we change this $d$ to $\mathsf{dis}$?}	


\begin{definition}%\protect{\cite[Definition 5]{DBLP:journals/siamcomp/DodisORS08}}
\label{def:fuzzy extractor}
An $(\mathcal{M}, m, \ell, t, \epsilon)$-\emph{fuzzy extractor} with error $\delta$ is a pair of randomized procedures, ``generate'' $(\gen)$ and ``reproduce'' $(\rep)$, with the following properties: 
\begin{enumerate}
\item The generate procedure \gen on input $w\in \mathcal{M}$ outputs an extracted string $r\in\{0,1\}^\ell$ and a helper string $p\in\{0,1\}^*$.
\item The reproduction procedure \rep takes an element $w'\in \mathcal{M}$ and a bit string $p\in\{0,1\}^*$ as inputs.  The \emph{correctness} property of fuzzy extractors guarantees that for $w$ and $w'$ such that $\dis(w,w')\leq t$, if $R,P$ were generated by $(R,P)\leftarrow\gen(w)$, then $\rep(w',P)=R$ with probability~(over the coins of $\gen, \rep$) at least $1-\delta$.  If $\dis(w,w')>t$, then no guarantee is provided about the output of \rep.
\item The \emph{security} property guarantees that for any distribution $W$ on $\mathcal{M}$ of min-entropy $m$, the string $R$ is nearly uniform even for those who observe $P$:  if $(R,P)\leftarrow\gen (W)$, then $\mathbf{SD}((R,P),(U_\ell,P))\leq \epsilon$.
\end{enumerate}
A fuzzy extractor is efficient if $\gen$ and $\rep$ run in expected polynomial time.
\end{definition}

Secure sketches are the main technical tool in the construction of fuzzy extractors.  Secure sketches produce a string $s$ that does not decrease the entropy of $w$ too much, while allowing recovery of $w$ from a  close $w'$:
\begin{definition}%\protect{\cite[Definition 3]{DBLP:journals/siamcomp/DodisORS08}}
\label{def:secure sketch}
An $(\mathcal{M},m, \tilde{m}, t)$-\emph{secure sketch} with error $\delta$ is a pair of randomized procedures, ``sketch'' $(\sketch)$ and ``recover'' $(\rec)$, with the following properties:
\begin{enumerate}
\item The sketching procedure \sketch on input $w\in\mathcal{M}$ returns a bit string $s\in\{0,1\}^*$.
\item The recovery procedure \rec takes an element $w'\in\mathcal{M}$ and a bit string $s\in\{0,1\}^*$.  The \emph{correctness} property of secure sketches guarantees that if $\dis(w,w')\leq t$, then $\Pr[\rec(w',\sketch(w))=w]\geq 1-\delta$ where the probability is taken over the coins of $\sketch$ and $\rec$.  If $\dis(w,w')>t$, then no guarantee is provided about the output of \rec.
\item The \emph{security} property guarantees that for any distribution $W$ over $\mathcal{M}$ with min-entropy $m$, the value of $W$ can be recovered by the adversary who observes $w$ with probability no greater than $2^{-\tilde{m}}$.  That is, $\Hav(W|\sketch(W))\geq \tilde{m}$.
\end{enumerate}
A secure sketch is \emph{efficient} if \sketch and \rec run in expected polynomial time. 
\end{definition}

Note that in the above definition of secure sketches (resp., fuzzy extractors), the errors are chosen before $s$ (resp., $P$) is known: if the error pattern between $w$ and $w'$ depends on the output of $\sketch$ (resp., $\gen$), then there is no guarantee about the probability of correctness.


A fuzzy extractor can be produced from a \emph{secure sketch} and an \emph{average-case randomness extractor}. An average-case extractor is a generalization of a strong randomness extractor \cite[Definition 2]{nisan1993randomness}) (in particular, Vadhan~\cite[Problem 6.8]{Vad12} showed that all strong extractors are average-case extractors with a slight loss of parameters):

\begin{definition}
Let $\chi_1$, $\chi_2$ be finite sets.
A function $\ext: \chi_1\times \{0,1\}^d \rightarrow \{0,1\}^\ell$ a \emph{$(m, \epsilon)$-average-case extractor} if for all pairs
of random variables $X, Y$ over $\chi_1, \chi_2$ such that
$\tilde{H}_\infty(X|Y) \ge m$, we have $\Delta((\ext(X, U_d), U_d, Y), U_\ell\times
U_d \times Y) \le \epsilon$.
\end{definition}



%In \secref{sec:impossCompSecSketch} we will use a slightly weaker notion, in which \rec is allowed to fail with small probability.\lnote{Can we add a comment that this corresponds to errors that can depend on the input but not the sketch? It's not obvious from the way this is written.} \lnote{can we just have a single definition --- the one below --- instead of two?}
%\begin{definition}
%We say that pair of procedures $(\sketch, \rec)$ is $(\mathcal{M}, m, \tilde{m}, t, \epsilon)$ is an \emph{approximately correct secure sketch} if \defref{def:secure sketch} holds with correctness replaced with the following property, $\forall w, w'$ such that $\dis(w, w')\leq t$:
%\[
%\Pr[\rec(w', \sketch(w)) = w]\geq 1-\epsilon.
%\] where the probability is taken over the coins of \sketch and \rec.
%\end{definition}
\begin{lemma}%\protect{\cite[Lemma 4.1]{DBLP:journals/siamcomp/DodisORS08}}
\label{lem:fuzzy ext construction}
Assume $(\sketch, \rec)$ is an $(\mathcal{M}, m, \tilde{m}, t)$-secure sketch with error $\delta$, and let $\ext:\mathcal{M}\times \zo^d \rightarrow \zo^\ell$ be a $(\tilde{m}, \epsilon)$-average-case extractor.  Then the following $(\gen, \rep)$ is an $(\mathcal{M}, m, \ell, t, \epsilon)$-fuzzy extractor with error $\delta$:
\begin{itemize}
\item $\gen(w):$ generate $x\leftarrow \zo^d$, set $p=(\sketch(w), x), r=\ext(w;x)$, and output $(r,p)$.
\item $\rep(w', (s, x)):$ recover $w=\rec(w',s)$ and output $r=\ext(w;x)$.
\end{itemize}
\end{lemma}
The main parameter we will be concerned with is the entropy loss of the construction.  In this paper, we ask whether a smaller entropy loss can be achieved by considering a fuzzy extractor with a computational security requirement.  We therefore relax the security requirement of \defref{def:fuzzy extractor} to require a pseudorandom output instead of a truly random output.  Also, for notational convenience, we modify the definition so that we can specify a general class of sources for which the fuzzy extractor is designed to work, rather than limiting ourselves to the class of sources that consists of all sources of a given min-entropy $m$, as in definitions above (of course, this modification can also be applied to prior definitions of information-theoretic secure sketches and fuzzy extractors).

\begin{definition}[Computational Fuzzy Extractor]\label{def:comp fuzzy extractor}
Let $\mathcal{W}$ be a family of probability distributions over $\mathcal{M}$. A pair of randomized procedures ``generate'' $(\gen)$ and ``reproduce'' $(\rep)$ is a $(\mathcal{M}, \mathcal{W}, \ell, t)$-\emph{computational fuzzy extractor} that is $(\epsilon, s_{sec})$-hard with error $\delta$ if \gen and \rep satisfy the following properties:
\begin{itemize}
\item The generate procedure \gen on input $w\in \mathcal{M}$ outputs an extracted string $R\in\{0,1\}^\ell$ and a helper string $P\in\{0,1\}^*$.
\item The reproduction procedure \rep takes an element $w'\in\mathcal{M}$ and a bit string $P\in\{0,1\}^*$ as inputs.  The \emph{correctness} property guarantees that if $\dis(w, w')\leq t$ and $(R, P)\leftarrow \gen(w)$, then $\Pr[\rep( w', P) = R] \geq 1-\delta$ where the probability is over the randomness of $(\gen, \rep)$.  %%Furthermore \rep is computable by a circuit of size at most $s_{rec}$.  
If $\dis(w, w') > t$, then no guarantee is provided about the output of \rep.
\item The \emph{security} property guarantees that for any distribution $W\in \mathcal{W}$, the string $R$ is pseudorandom conditioned on $P$, that is $\delta^{\mathcal{D}_{s_{sec}}}((R, P), (U_\ell, P))\leq \epsilon$.
\end{itemize}
\end{definition}
Any efficient fuzzy extractor is also a computational fuzzy extractor with the same parameters.

\paragraph{Remark}  Fuzzy extractor definitions make no guarantee about \rep behavior when the distance between $w$ and $w'$ is larger than $t$.  In the information-theoretic setting this seemed inherent as the ``correct'' $R$ should be information-theoretically unknown conditioned on $P$.  However, in the computationally setting this is not true.  Looking ahead, in our construction $R$ is information-theoretically determined conditioned on $P$~(with high probability over the coins of \gen).  Our $\rep$ algorithm will never output an incorrect key but may run for an unbounded time~(with high probability over the coins of \gen).  However, it is not clear this is the desired behavior.  For this reason, we leave the behavior of \rep ambiguous when $\dis(w, w')>t$.
% ($s_{rec}$ is the size of the circuit that computes \rec and $s_{sec}$ is unbounded).  

%\bnote{There needs to be more linking language here.}
%It seems natural to construct a computational fuzzy extractor using a sketch that retains computational entropy and then apply a randomness extractor~(extractors convert distributions with high computational entropy to pseudorandom).  In the next section, we show this paradigm is difficult to realize. 

%\textbf{Note: }A computational fuzzy extractor could also be constructed by making the extractor computational~(using a standard extractor and then running a pseudorandom generator on its output).  We compare the two approaches in \secref{sec:prg based comparison} after presenting our construction.
%, specifically, the entropy loss in the definition of the secure sketch~($m-\tilde{m}$).  %The cause for the entropy loss in the secure sketch is the additional bits added for error correction.  In the definition of a fuzzy extractor this is the input min-entropy less the size of the output: $m-\ell$.   
%The entropy loss in the fuzzy extractor is due to the use of an extractor (where the loss is inversely proportional to the statistical distance of the output distribution from uniform) and the use of the secure sketch.


%=============================================================
\section{Impossibility of Computational Secure Sketches}
\label{sec:impossCompSecSketch}
In this section, we consider whether it is possible in build a secure sketch that retains significantly more computational than information-theoretic entropy.  We consider two different notions for computational entropy, and for both of them show that corresponding secure sketches are subject to the same upper bounds as those for information-theoretic secure sketches. Thus, it seems that relaxing security of sketches from information-theoretic to computational does not help.

%We will not provide formal definitions for these sketches but use the notation HILL/unpredictability~$(\mathcal{M}, m, \tilde{m}, t)$-secure sketch to denote a sketch that retains $\tilde{m}$ bits of HILL~(resp. unpredictability) entropy when given a source with $m$ bits of entropy.  
In particular, for the case of the Hamming metric and inputs that have full entropy, our results are as follows.  In  \secref{sec:imp HILL sketch} we show that a sketch that retains HILL entropy implies a sketch that retains nearly the same amount of min-entropy.  In \secref{sec:imp unp sketch}, we show that the computational unpredictability of a sketch is at most $\log |\mathcal{M}| - \log |B_t(\cdot)|$. Dodis et al. \cite[Section 8.2]{DBLP:journals/siamcomp/DodisORS08}  construct sketches with essentially the same information-theoretic security\footnote{The security in  \cite[Section 8.2]{DBLP:journals/siamcomp/DodisORS08}  is expressed in terms of entropy of the error rate; recall that $\log B_t(\cdot)\approx H_q(t/n)$, where $n$ is the number of symbols, $q$ is the alphabet size, and $H_q$ is the $q$-ary entropy function.} . 


In \secref{ssec:avoiding bounds}, we discuss mechanisms for avoiding these bounds.

%These results are strongest for high entropy sources.  The results get weaker as we consider sources with lower starting entropy.  If a source contains points from a good error code, then the sketch procedure does not need to output any public value so no drop occurs in entropy.  

\subsection{Bounds on Secure Sketches using HILL entropy}
\label{sec:imp HILL sketch}
HILL entropy is a commonly used computational notion of entropy \cite{DBLP:journals/siamcomp/HastadILL99}.  It was extended to the conditional case by Hsiao, Lu, Reyzin~\cite{DBLP:conf/eurocrypt/HsiaoLR07}. Here we recall a weaker definition due to Gentry and Wichs~\cite{gentry2011separating}~(the term relaxed HILL entropy was introduced in~\cite{reyzin2011some}); since we show impossibility even for this weaker definition, impossibility for the stronger definition follows immediately.

\begin{definition}
\label{def:relaxed hill}
Let $(W, S)$ be a pair of random variables.  $W$ has 
\emph{relaxed HILL entropy} at least $k$ conditioned on $S$,
denoted $H^{\hillrlx}_{\epsilon, s_{sec}}(W|S)\geq k$ if there exists a joint distribution $(X, Y)$, such that $\tilde{H}_\infty(X|Y)\geq k$ and $\delta^{\mathcal{D}_{s_{sec}}} ((W, S),(X,Y))\leq \epsilon$.
\end{definition}

%\lnote{is there any reason we can't work for the more relaxed notion of HILL, where the condition can be changed, too?}

Intuitively, HILL entropy is as good as average min-entropy for all computationally-bounded observers.  Thus, redefining secure sketches using HILL entropy is a  natural relaxation of the original information-theoretic definition; in particular, the sketch-and-extract construction in \lemref{lem:fuzzy ext construction} would yield pseudorandom outputs if the secure sketch ensured high HILL entropy.  
%Instead, in \secref{sec:unp secure sketch}, we relax the definition by allowing by outputting a new random variable that can be recovered when $\dis (w, w')\leq t$.  This object is the computational analogue of a fuzzy conductor~\cite{KanukurthiR09}.%This is a noisy computational conductor define computational secure sketches using unpredictability, and work with that definition for the rest of the paper.  
%  In \secref{sec:comp fuzzy extractors}, we show that computational fuzzy extractors can be constructed from  computational fuzzy conductors.
%The size of the circuits becomes important, we would like the \rec procedure to be computable by a circuit significantly smaller than the \hill entropy circuit size.  We simply include both of these parameters in the definition.  We review other definitions of computational entropy in Appendix~\ref{sec:notionsEntropy}.
We will consider secure sketches that retain relaxed HILL entropy: that is, we say that $(\sketch, \rec)$ is a  \emph{HILL-entropy~$(\mathcal{M}, m, \tilde{m}, t)$ secure sketch} that is $(\epsilon,s_{sec})$-hard with error $\delta$ if it satisfies \defref{def:secure sketch}, with the security requirement replaced by $H^{\hillrlx}_{\epsilon, s_{sec}}(W|\sketch(W))\geq \tilde{m}$. 

Unfortunately, we will show below that such a secure sketch implies an error correcting code with approximately $2^{\tilde{m}}$ points that can correct $t$ random errors (see  \cite[Lemma C.1]{DBLP:journals/siamcomp/DodisORS08} for a similar bound on information-theoretic secure sketches). For the Hamming metric, our result essentially matches the bound on information-theoretic secure sketches of \cite[Proposition 8.2]{DBLP:journals/siamcomp/DodisORS08}.  In fact, we show that, for the Hamming metric, HILL-entropy secure sketches imply information-theoretic ones with similar parameters, and, therefore, the HILL relaxation gives no advantage. 


The intuition for building error-correcting codes from HILL-entropy secure sketches is as follows.  In order to have  $H^{\hillrlx}_{\epsilon, s_{sec}}(W|\sketch(W))\ge \tilde{m}$, there must be a distribution $X, Y$ such that $\Hav(X | Y)\geq \tilde{m}$ and $(X, Y)$ is computationally indistinguishable from $(W, \sketch(W))$.  Sample a sketch $s\leftarrow \sketch(W)$. We know that $\sketch$ followed by $\rec$ likely succeeds on $W|s$  (i.e., $\rec (w', s) = w$ with high probability for $w\leftarrow W|s$ and $w'\leftarrow B_t(w)$).  %So, by indistinguishability, it must also succeed on $Y$. 
 Consider the following experiment: 1) sample $y\leftarrow Y$, 2) draw $x\leftarrow X|y$ and 3) $x'\leftarrow B_t(x)$. By indistinguishability, $\rec (x',y) = x$ with high probability.
 This means we can construct a large set $\mathcal{C}$ from the support of $X|y$.  $\mathcal{C}$ will be an error correcting code and $\rec$ an efficient decoder.  We can then use standard arguments to turn this code into an information theoretic sketch.  


To make this intuition precise, we need an additional technical condition:  sampling a random neighbor of a point is efficient.
\begin{definition}
\label{def:neighborhood samplable}
We say a metric space $(\mathcal{M}, \dis)$ is $(s_{neigh}, t)$-\emph{neighborhood samplable} if there exists a randomized circuit $\neigh$ of size $s_{neigh}$ that for all $t'\leq t$, $\neigh (w, t')$ outputs a random point at distance $t'$ of $w$.  
%That is $\exists \sample$ of size at most $s$ such that $\forall x\in \mathcal{M},$
%\[
%\Delta(\sample(x, t') , \{x'| \dis(x,x') = t'\})=0
%\]
\end{definition}
%We require that \sample returns a point different from $x$ to ensure that each $B_t(x)$ contains at least one other point.  While we could consider metric spaces where balls only contain their center, secure sketches are not interesting in this context. \lnote{ do we need this? if distance is exactly $t'$, then the point is different as long as $t'>0$.  So why not just say "exactly $t'$ in the definition?}

%We will concentrate on metrics where the size of each ball is fixed.  We say a metric space that obeys this property is regular~(using terminology from graph theory).  This will allow us to show that the neighbor of a uniformly chosen point is also uniformly random.  
%\begin{definition}
%A metric space $(\mathcal{M},d)$ is \emph{$(t, c)-$distance regular} if for all $x\in \mathcal{M}, |B_t(x)| = c$. 
%\end{definition}
%For a distribution $X$ we define the distribution $N_t(X)$ as the process of sampling a random neighbor of a random point in $X$.  That is $\Pr[N_t(X) = z] = \sum_{x\in X}\Pr[X=x \wedge y \leftarrow B_t(x) \wedge z=y]$.
%\begin{lemma}\label{lem:uniformNeighbor}
%Let $(\mathcal{M}, d)$ be a $(t, c)-$distance regular and let $U$ be uniform over $\mathcal{M}$.  Then $U \overset{d}= N_t(U)$.
%
%\end{lemma}
%\begin{proof}
%Note that $\forall z, \Pr[U = z] =1/|\mathcal{M}|$.  Thus, it suffices to show that $\forall z , \Pr[N_t(U) =z] =1/|\mathcal{M}|$.
%
%Fix an arbitrary $z\in \mathcal{M}$.  Then, 
%\begin{align*}
%\sum_{x\in U} \Pr[U=x \wedge y\leftarrow B_t(x) \wedge y=z] &=
%\sum_{x\in U} \Pr[U = x \wedge x\in B_t(z)] \Pr[y\leftarrow B_t(x) \wedge y=z | x\in B_t(z)]\\
%&=\frac{c}{|\mathcal{M}|} \frac{1}{c} = \frac{1}{\mathcal{M}}.
%\end{align*}
%%The uniform distribution over $\mathcal{M}$ is a random bit string of length $n$.  Let $x\leftarrow \mathcal{M}$, then $\sample (x) = z = x\oplus y$ where $y$ has at weight at least $1$ and at most $t$.  Since $x$ was uniformly random $x\oplus y$ is uniformly random.
%\end{proof}

\ignore{
Finally, we will 
dealing with an arbitrary secure sketch that only needs to satisfy the correctness property of a secure sketch~(see \defref{def:secure sketch}).  \lnote{why do we need this definition?}

\begin{definition}
A pair of functions $(\sketch, \rec) $ is a $(s_{rec}, t)$-\emph{recover functionality for a metric space $\mathcal{M}, d$} if $\sketch : \mathcal{M}\rightarrow \{0,1 \}^*$ is a randomized function and $\rec: \mathcal{M}\times \{0, 1\}^*\rightarrow \mathcal{M}$ is a function computable by a circuit of size at most $s_{rec}$, such that $\rec (w', \sketch (w)) = w$ if $\dis(w, w')\leq t$ for all $w, w' \in \mathcal{M}$.
\end{definition}
}

\ignore{For any recover functionality, there is an inherent tradeoff between the ability to keep points stationary and to perform error correction.

\begin{theorem}\label{thm:compSketchImpUniform}
Let $\mathcal{M}, d$ be a metric space that is $(s_{neigh}, t)$-neighborhood samplable and $(t, c)$-distance regular.
Furthermore, let $(\sketch, \rec)$ be any $(s_{rec}, t)$-recover functionality for $\mathcal{M}, d$.  Then $H^\hill_{\epsilon, s}( U |\sketch(U))\not \geq \Hoo(U)$ for $s \approx (s_{neigh} + 2s_{rec})$ and $\epsilon=1/2$~(if $s_{neigh}, s_{rec}$ are polynomial so is $s$).
\end{theorem}
%\emph{Intuition:  }The goal that a recover functionality cannot be both stable~(maps points to themselves) and correcting~(maps nearby points back to themselves).  If a recovery functionality is correcting then we can check if the input point maps to itself.  If a recovery functionality is stable then we can see if recovery succeeds with nearby points.  

\begin{proof}
Let $U_\mathcal{M}$ be the uniform distribution over $\mathcal{M}$ and let $d$ a  metric.  Furthermore assume $\mathcal{M}$ is $(s_{neigh}, t)$-neighborhood samplable.  Let $(\sketch, \rec)$ be a $(s_{rec}, t)$-recover functionality for $\mathcal{M}, d$.  Let $X\overset{d}= Y\overset{d}= U_\mathcal{M}$ be i.i.d..  It suffices to construct a distinguisher $D$~(whose size is $s\approx s_{neigh}+2s_{rec}$) such that $\expe[D(X, SS(X))]- \expe[D(Y, SS(X))]\geq \epsilon = 1/2$.  
%Let $q$ be a polynomial.  
$D$ is defined as follows:
\begin{enumerate}
\item Input $z\in\mathcal{M}, s \in\{0, 1\}^*$. 
\item Sample $b\leftarrow \{0,1\}, z'\leftarrow \sample(z)$.
\item If $b=1$:
\subitem If $\rec(z', s) = z\wedge \rec(z, s) = z$ output $1$.
\item If $b=0$:
\subitem If $\rec(z, s) = z \wedge \rec(z', s)=z$ output $1$.
\item Output $0$.
\end{enumerate}
First note that $D$ is of size $O(s_{neigh}+2s_{rec})$ and thus in polynomial in $s_{sam}, s_{neigh}$.  It remains to show that $D$ separates $(X,\sketch(X))$ and $(Y,\sketch(X))$.  We being by noting that on input $(X, \sketch(X)), D$ outputs $1$.
By \lemref{lem:uniformNeighbor} we have:
\begin{align*}
0&=\delta([Y, \sample(Y), \sketch(X)],& &[\sample(Y), Y, \sketch(X)])\\
&=\delta([Y, \sketch(X), \sample(Y), \sketch(X)],& &[\sample(Y), \sketch(X), Y, \sketch(X)])\\
&=\delta([\rec(Y,\sketch(X)), \rec(\sample(Y),\sketch(X))],& &[\rec(\sample(Y), \sketch(X)), \rec(Y, \sketch(X))])
\end{align*}
Thus, the output of $\rec$ is the same when given $(\sample(Y), Y)$ or $(Y, \sample(Y))$\footnote{This is also true for a stronger class of recover functionalities that is able to keep state between queries.}.  This means it can properly recover $Y$ with probability at most $1/2$ over the ordering of inputs.  Thus we have, 
\begin{align*}
\Pr[D'(X, \sketch(X))=1] - \Pr[D'(Y, \sketch(X))=1] &=\\
1 - \frac{1}{2}(\Pr[\rec(Y, \sketch(X)) = Y \wedge \rec(\sample(Y),\sketch(X))=Y]&+\\\Pr[\rec(\sample(Y), \sketch(X)) = Y \wedge \rec(Y, \sketch(X)) = Y])&\geq 
1-\frac{1}{2}\left(1\right) =\frac{1}{2}
\end{align*}

%\begin{enumerate}
%\item Input $z\in\mathcal{M}, s \in\{0, 1\}^*$.
%\item Initialize $correct, stable = 0$.
%\item For $i=1...q$:
%\subitem $x_i \leftarrow U$, if $\rec(x_i, s) = x_i, stable +=1/q$.
%\subitem $x_i' \leftarrow \sample (x_i)$, if $\rec(x_i', s) = x_i, correct+=1/q$.
%\item If $\rec(z, s)\neq z$ output $0$.
%\item If $stable<1/2$ output $1$
%\item Else $z'\leftarrow \sample(z)$.
%\subitem If $\rec(z', s) = z$ output $1$.
%\subitem Else output $0$.
%\end{enumerate}
%
%We begin by noting that $D$ is of size $O(q(s_{sam}+s_{neigh}+2s_{rec})+2s_{rec}+s_{neigh})$ and thus is polynomial in $s_{sam}, s_{neigh}, s_{rec}$.  It remains to show that $D$ separates $(X, SS(X))$ and $(Y, SS(X))$ by a noticeable $\epsilon$.  Let $STABLE = \Pr[\rec(U, \sketch(X)) = U]$.  We begin by noting that $\Pr[|stable-STABLE|>\ngl]<\ngl(n)$ assuming enough queries~(\bnote{fill in this information}).  That is, after step (3) $D$ has a good approximation
%of $STABLE$.  Similarly, let $CORRECT = \Pr[\rec(\sample(U), \sketch(X)) = U]$.  As before $\Pr[|stable-STABLE|>\ngl]<\ngl(n)$ assuming enough queries~(\bnote{fill in this information}).
%Thus, after step (3), $\Pr[|CORRECT-correct|>\ngl \wedge |STABLE-stable|>\ngl|]<\ngl$.  We proceed assuming, that $CORRECT$ and $STABLE$ are known exactly at step (4), and note the output of $D$ is negligibly close to the output of a distinguisher that knows $CORRECT$ and $STABLE$ exactly~(we label this distinguisher $D'$).
%\begin{claim}
%$CORRECT+STABLE\leq 1$.
%\end{claim}
%\begin{proof}
%\begin{align*}
%\Pr[\rec(U, \sketch(X) = U] + \Pr[\rec(\sample(U), \sketch(X)) = U]&\leq\\
%\Pr[\rec(U, \sketch(X) = U]  + \Pr[\rec(Y), \sketch(X)\neq Y] &=1
%\end{align*}
%Where the first inequality proceeds because of \lemref{lem:uniformNeighbor} and the fact that \sample never returns itself.  The equality proceeds as these events as these events are complementary events after renaming. 
%\end{proof}
%We now show that when $(\sketch, \rec)$ is stable or correcting the distinguisher has nonnegligible advantage:
%\begin{itemize}
%\item Case 1: $STABLE<1/2$.
%\begin{align*}
%\Pr[D'(X, \sketch(X) )= 1] - \Pr[D'(Y, \sketch(X) )= 1] &=\\
%\Pr[\rec(X, \sketch(X) = X] - \Pr[\rec(Y, \sketch(X) = Y] &\geq
% 1- 1/2  = 1/2
%\end{align*} 
%\item Case 2: $STABLE\geq 1/2, CORRECT \leq 1/2$.
%\begin{align*}
%\Pr[D'(X, \sketch(X) )= 1] - \Pr[D'(Y, \sketch(X) )= 1]&=\\
%\Pr[\rec(X, \sketch(X)) = X \wedge \rec(\sample(X), \sketch(X)) = X] &-\\\Pr[\rec(Y, \sketch(X) = Y) \wedge \rec(\sample(Y), \sketch(X)) = Y)]&\geq\\
%1 - \Pr[\rec(\sample(Y), \sketch(X)) = Y)]= 1-1/2 = 1/2
%\end{align*}
%\end{itemize}
%Thus, in both cases we have $\Pr[D'(X, \sketch(X) )= 1] - \Pr[D'(Y, \sketch(X) )= 1]\geq 1/2$.  As noted above, $CORRECT, STABLE$ are only known approximately and thus:
%\[
%\Pr[D(X, \sketch(X) )= 1] - \Pr[D(Y, \sketch(X) )= 1]\geq 
%\Pr[D'(X, \sketch(X) )= 1] - \Pr[D'(Y, \sketch(X) )= 1] - \ngl \geq 1/2 -\ngl
%\]
This completes the proof.
\end{proof}

This proof relied on two crucial facts: 1) that the neighbor of a uniform point was uniformly distributed 2) for an point uncorrelated with the value of the secure sketch it was impossible to distinguish between a point and its neighbor.  We can extend this result to distributions other than the uniform distribution but by drawing on more general results from coding theory.  

\subsection{Indistinguishable sketches imply information theoretic sketches}
In this section we show that the results of \thref{thm:compSketchImpUniform} can be extended to whenever $X$ is indistinguishable from a high min-entropy distribution.  }

We review definitions of Shannon codes~\cite{shannon1949mathematical}:
\begin{definition}
\label{def:shannon-code}
Let $\mathcal{C}$ be a set over space $\mathcal{M}$.  We say that $\mathcal{C}$ is an $(t,\epsilon)$-\emph{Shannon code} if there exists an efficient procedure $\rec$ such that for all $t'\le t$ and for all $c\in \mathcal{C}$, $\Pr[\rec(\neigh(c, t')) \neq c]\le \epsilon$. To distinguish it from the average-error Shannon code defined below, we will sometimes call it \emph{maximal-error} Shannon code.
\end{definition}
This is a slightly stronger formulation than usual, in that for every size  $t'<t$ we require the code to correct $t'$ random errors\footnote{In the standard formulation, the code must correct a random error of size up to $t$, which may not imply that it can correct a random error of a much smaller size $t'$, because the volume of the ball of size $t'$ may be negligible compared to the volume of the ball of size $t$.  For codes that are monotone~(if decoding succeeds on a set of errors, it succeeds on all subsets), these formulations are equivalent.  However, we work with an arbitrary recover functionality that is not necessarily monotone.}.  

Shannon codes work for all codewords. We can also consider a formulation that works for an ``average'' codeword. 

 \begin{definition}
Let $C$ be a distribution over space $\mathcal{M}$.  We say that $C$ is an $(t,\epsilon)$-\emph{average error Shannon code} if there exists an efficient procedure $\rec$ such that for all $t'\le t$
$\Pr_{c\leftarrow C}[\rec(\neigh(c, t')) \neq c]\le \epsilon$.
\end{definition}
An average error Shannon code is one whose average probability of error is bounded by $\epsilon$.  See~\cite[Pages 192-194]{cover2006elements} for definitions of average and maximal error probability.  An average-error Shannon code is convertible to a maximal-error Shannon code with a small loss.  We use the following pruning argument from~\cite[Pages 202-204]{cover2006elements} (we provide a proof in \secref{sec:proof of average to maximal error}):
\begin{lemma}
\label{lem:averageToMaximalError}
Let $C$ be a $(t, \epsilon)$-average error Shannon code with recovery procedure $\rec$ such that $\Hoo(C)\geq k$.  There is a set $\mathcal{C}'$ with $|\mathcal{C}'|\ge2^{k-1}$ that  is a $(t, 2\epsilon)$-(maximal error) Shannon code with recovery procedure $\rec$.
\end{lemma}

We can now formalize the intuition above and show that a sketch that retains $\tilde{m}$-bits of relaxed HILL entropy implies a good error correcting code with nearly $2^{\tilde{m}}$ points~(proof in \secref{sec:proof of thm sketch implies code}).
\begin{theorem}\label{thm:impSketchArbitraryW}
Let $(\mathcal{M}, \dis)$ be a metric space that is $(s_{neigh}, t)$-neighborhood samplable.  Let $(\sketch, \rec)$ be an HILL-entropy $(\mathcal{M}, m, \tilde{m}, t)$-secure sketch that is $(\epsilon, s_{sec})$-secure with error $\delta$.  Let $s_{rec}$ denote the size of the circuit that computes $\rec$.  If $s_{sec}\geq (t(s_{neigh}+s_{rec}))$,  then there exists a value $s$ and a set $\mathcal{C}$ with $|\mathcal{C}|\geq 2^{\tilde{m}-2}$  that is a $(t, 4(\epsilon+t\delta))$-Shannon code with recovery procedure $\rec(\cdot, s)$.
%Let $W$ be a flat distribution~(all points in the support of $W$ have the same probability) over $\mathcal{M}$ where $\Hoo(W)\geq k$ and   Let $X$ be arbitrarily but independently distributed over $\mathcal{M}$.  If $\delta^D((X, \sketch(X)), (W, \sketch(X)))<\epsilon$ for negligible $\epsilon$ and $D$ of size at least $O(s_{neigh}+s_{rec})$, then $\exists x, W'$ where $|W'|\geq |W|/2 \geq 2^{k-1}$ such that $W'$ is a $(t, 2\epsilon)$ Shannon code with recovery procedure $\rec(\cdot, \sketch(x))$.
\end{theorem}
%The code provided in \thref{thm:impSketchArbitraryW} is weaker than a Shannon error correcting code~\cite{shannon1948}.  In a Shannon code, for all codewords $c, \Pr[c'\leftarrow \sample(c) \wedge \rec(c') \neq c]<\epsilon$.  Bounds for this type of code are well known~\cite{shannon1948} and efficient constructions exist~\cite{forney1966}.  Requiring proper decoding for only a random codeword is a significant restriction particularly in the case where the probability of outcomes in $W$ differs significantly~(e.g. where $\exists w_1,w_2\in W$ such that $\Pr[W=w_1]/\Pr[W=w_2] = \ngl$).  However, this is still a meaningful set of error correcting codes.  A similar question is considered in Appendix C of \cite{DBLP:journals/siamcomp/DodisORS08}.  They are able to achieve results for Hamming codes~(as they are not restricted to sampling a polynomial portion of the space)\bnote{expand on this explanation}.


%It is known (see \cite[Section 8.2]{DBLP:journals/siamcomp/DodisORS08} and references therein) that, 
For the Hamming metric, any Shannon code (as defined in Definition~\ref{def:shannon-code}) can be converted into an information-theoretic secure sketch~(as described in \cite[Section 8.2]{DBLP:journals/siamcomp/DodisORS08} and references therein).  The idea is to use the code offset construction, and convert worst-case errors to random errors by randomizing the order of the bits of $w$ first, via a randomly chosen  permutation $\pi$  (which  becomes part of the sketch and is applied to $w'$ during $\rec$). The formal statement of this result  can be expressed in the following Lemma (which is implicit in \cite[Section 8.2]{DBLP:journals/siamcomp/DodisORS08}).
\begin{lemma}
\label{lem:shannon to sketch}
For an alphabet $\mathcal{Z}$, let $\mathcal{C}$ over $\mathcal{Z}^n$ be a $(t, \delta)$ Shannon code.  % with recovery procedure $\rec'$.
%  Let $W$ be a distribution with $\Hoo(W)\geq k$. 
%
% Define the following procedures
%\lnote{I think this description is buggy}
%\begin{center}
%\begin{tabular}{c|c}
%\begin{minipage}{3in}
%\textbf{\sketch}
%\begin{enumerate}
%\item Input $w\leftarrow W$.
%\item Sample $\pi$ from the space of all permutations from $\pi: [n]\rightarrow [n]$.
%\item Sample $c\leftarrow \mathcal{C}$.
%\item Apply $\pi$ bitwise to get $x_i = w_{\pi(i)}$.
%\item Output $p = x \oplus c, \pi$.
%\end{enumerate}
% \end{minipage} &
%\begin{minipage}{3in}
%\textbf{\rec}
%\begin{enumerate}
%\item Input $(w', p, \pi)$
%\item Apply $\pi^{-1}$ bitwise to get $x_i' = w_{\pi(i)}$.
%\item Compute $c' = p \oplus x'$.
%\item Compute $c = \rec'(c)$.
%\item Output $p\oplus c$.
%\\
%\end{enumerate}
%\end{minipage} 
%\end{tabular}
%\end{center}
Then there exists a $(\mathcal{Z}^n, m, m-(n\log|\mathcal{Z}|-\log |\mathcal{C}|), t)$ secure sketch with error $\delta$ for the Hamming metric over $\mathcal{Z}^n$. 
%satisfies the following properties~(probability over the choice of $\pi$ in $\sketch$):
%\begin{itemize}
%\item Correctness: $\forall w, w'$ where $dis(w, w')\leq t$, $\Pr[\rec(w', \sketch(w))\neq w]<\epsilon$.
%\item Security: $\Hav(W |\sketch(W)) \geq \Hoo(W) - (n-\log |C|)$.
%\item Efficiency: $\sketch$ is efficient if sampling from $C$ is efficient and $\rec$ is efficient if $\rec'$ is efficient.
%\end{itemize}
\end{lemma}
%\begin{construction}
%Let $\delta >0$ be a parameter.
%Let $C$ be a $(t, \epsilon)$ Shannon code over $\{0,1\}^n$.  Choose a permutation $\pi:\mathcal{M}rightarrow \mathcal{M}$ that is $\beta$-almost independent $\ell = \delta n/\log(n)$-wise permutation.  Thefn for $\beta = 2^{-\ell}
%\end{construction}
Putting together \thref{thm:impSketchArbitraryW} and \lemref{lem:shannon to sketch} gives us the negative result for the Hamming metric: a HILL-entropy secure sketch (for the uniform distribution) implies an information-theoretic one with similar parameters:
\begin{corollary}
\label{cor:rec yields sketch}
Let $\mathcal{Z}$ be an alphabet. Let $(\sketch', \rec')$ be an $(\epsilon,s_{sec})$-HILL-entropy $(\mathcal{Z}^n, n\log |\mathcal{Z}|, \tilde{m}, t)$-secure sketch with error $\delta$ for the Hamming metric over $\mathcal{Z}^n$, with $\rec'$ of circuit size $s_{rec}$.
If $s_{sec}\geq t(s_{rec} + n\log |\mathcal{Z}|)$, then there exists a   $(\mathcal{Z}^n, n\log |\mathcal{Z}|, \tilde{m}-2,t)$ (information-theoretic) secure sketch with error
$4(\epsilon+t\delta)$. 
%Let $(\mathcal{M}, d)$ be a metric space that is $(s_{neigh}, t)$-neighborhood samplable.  If no $(t, \epsilon)$-average error Shannon code over $\mathcal{M}$ exists of size $2^k$ then, $\forall (s_{rec}, t)$ recovery procedures $(\sketch, \rec)$ $H^{\hill}_{\epsilon, s}(X|\sketch(X))\not\geq k$ for $s=O(s_{rec}+s_{neigh})$.
\end{corollary}
\textbf{Notes:} In \corref{cor:rec yields sketch} we make no claim about the efficiency of the resulting  $(\sketch, \rec)$, because the proof of \thref{thm:impSketchArbitraryW} is not constructive.  

\corref{cor:rec yields sketch} extends to non-uniform distributions: if there exists a distribution whose HILL sketch retains $\tilde{m}$ bits of entropy, then for all distributions $W$, there is an information theoretic sketch that retains $\Hoo(W) - (n\log |\mathcal{Z}|-\tilde{m})-2$ bits of entropy.

%\corref{cor:rec yields sketch} shows that sketches that retain HILL entropy can be converted into an information-theoretic sketch.  Unfortunately, we have lower bounds for the entropy drop for information-theoretic sketches:

%\begin{proposition}\protect{\cite[Proposition 8.2]{DBLP:journals/siamcomp/DodisORS08}}
%\label{prop:lower bound entropy drop}
%For any error rate $1\leq t\leq n$, any secure sketch which corrects $t$ random errors with probability at least $2/3$ has entropy loss at least $n(h(t/n) -o(1))$; that is, $\Hav(W|\sketch(W))\leq n (1-h(t/n) - o(1))$ when $W$ is drawn uniformly from $\{0,1\}^n$.  Here $h(\cdot)$ is the binary entropy function.
%\end{proposition}

%Together \corref{cor:rec yields sketch} and \propref{prop:lower bound entropy drop} say the entropy drop for a ``HILL sketch'' must be the same as the entropy drop of a sketch that corrects random errors.  This entropy drop is tied to the Shannon capacity.  Furthermore, we have information theoretic constructions that meet these bounds.  Thus, defining a computational secure sketch using HILL entropy is unlikely to be useful.


%\begin{theorem}
%Let $\mathcal{M}, d$ be a finite metric space that is $s_{sam}$ efficiently samplable and $s_{ball}, t$ neighborhood efficiently samplable.  Furthermore, let $(\sketch, \rec)$ be any $(s_{rec}, t)$-recover functionality for $\mathcal{M}, d$.  Let $U$ be the uniform distribution over $\mathcal{M}$.  If $s_{ball}, s_{rec}, s_{sam}$ are of polynomial size, then $H^\hill_{\epsilon, s}( U |\sketch(U))\not \geq \Hoo(U)$ for polynomial $s$ and noticeable $\epsilon$.
%\end{theorem}
%\emph{Intuition:  }We will show that a recover functionality cannot be both stable~(maps points to themselves) and correcting~(maps nearby points back to themselves).  If a recovery functionality is correcting then we can check if the input point maps to itself.  If a recovery functionality is stable then we can see if recovery succeeds with nearby points.  
%
%\begin{proof}
%Let $\mathcal{M}, d$ be a finite metric space that is $s_{sam}$ efficiently samplable and $s_{ball}, t$ neighborhood efficiently samplable~(by $\sample$).  Let $(\sketch, \rec)$ be a $s_{rec}, t$-recover functionality for $\mathcal{M}, d$.  Let $X, Y, U$ be i.i.d. and be the uniform distribution over $\mathcal{M}$.  By assumption $s_{ball}, s_{sam}, s_{rec}$ be of polynomial size.  It suffices to construct a polynomial size distinguisher $D$ such that $\expe[D(X, SS(X))]- \expe[D(Y, SS(X)]> \epsilon$ for noticeable $\epsilon$.  Let $q$ be a polynomial.  $D$ is defined as follows:
%
%\begin{enumerate}
%\item Input $z\in\mathcal{M}, s \in\{0, 1\}^*$.
%\item Initialize $correct, stable = 0$.
%\item For $i=1...q$:
%\subitem $x_i \leftarrow U$, if $\rec(x_i, s) = x_i, stable +=1/q$.
%\subitem $x_i' \leftarrow \sample (x_i)$, if $\rec(x_i', s) = x_i, correct+=1/q$.
%\item If $\rec(z, s)\neq z$ output $0$.
%\item If $stable<2/3$ output $1$
%\item Else $z'\leftarrow \sample(z)$.
%\subitem If $\rec(z', s) = z$ output $1$.
%\subitem Else output $0$.
%\end{enumerate}
%
%We begin by noting that $D$ is of size at most $q(s_{sam}+s_{ball}+2s_{rec})+2s_{rec}+s_{ball}$ and thus is polynomial assuming that $q, s_{sam}, s_{ball}, s_{rec}$ are of polynomial size.  It remains to show that $D$ separates $X, SS(X)$ and $Y, SS(X)$ by a noticeable $\epsilon$.  Let $STABLE = \Pr[\rec(U, \sketch(X)) = U]$.  We begin by noting that $\Pr[|stable-STABLE|>\ngl]<\ngl(n)$ assuming enough queries~(\bnote{fill in this information}).  That is, after step (3) $D$ has a good approximation
%of $STABLE$.  Similarly, let $CORRECT = \Pr[\rec(\sample(U), \sketch(X)) = U]$.  As before $\Pr[|correct-STABLE|>\ngl]<\ngl(n)$ assuming enough queries~(\bnote{fill in this information}).
%
%\begin{claim}
%$CORRECT+STABLE\leq 1$.
%\end{claim}
%\bnote{finish proof}
%
%\end{proof}

\subsection{Bounds on Secure Sketches using Unpredictability Entropy}
\label{sec:imp unp sketch}
In the previous section, we showed that any sketch that retained HILL entropy could be transformed into an information theoretic sketch.  However, HILL entropy is a strong notion.  In this section, we therefore ask whether it is useful to consider a sketch that satisfies a minimal requirement: the value of the input is computationally hard to guess given the sketch.  We begin by recalling the definition of conditional unpredictability entropy, which captures the notion of ``hard to guess'' (we relax the definition slightly, similarly to the relaxation of HILL entropy described in the previous section).

\begin{definition}~\cite[Definition 7]{DBLP:conf/eurocrypt/HsiaoLR07}
\label{def:unp entropy}
Let  $(W, S)$ be a pair of random variables. We say that $W$ has \emph{relaxed unpredictability entropy} at least $k$ conditioned on $S,$ denoted by $H^{\unprlx}_{\epsilon, s_{sec}} (W|S) \geq k$, if there exists a pair of distributions $(X, Y)$ such that $\delta^{\mathcal{D}_{s_{sec}}}((W, S),(X, Y))\leq \epsilon$, and for all circuits $\mathcal{I}$ of size $s_{sec}$,
\[
\Pr[\mathcal{I}(Y) = X ] \leq 2^{-k}
.\]
\end{definition}

We will say that a pair of procedures $(\sketch, \rec)$ is a \emph{unpredictability-entropy $(\mathcal{M}, m, \tilde{m}, t)$ secure sketch} that is $(\epsilon, s_{sec})$-hard with error $\delta$ if it satisfies \defref{def:secure sketch}, with the security requirement replaced by $H^{\unprlx}_{\epsilon, s_{sec}}(W| \sketch(W))\geq \tilde{m}$.  
Note this notion is quite natural: combining such a secure sketch in a sketch-and-extract construction of  \lemref{lem:fuzzy ext construction} with a particular type of extractor (called a extractor 
\emph{reconstructive}~\cite{barak-computational}), would yield a computational fuzzy extractor (per \cite[Lemma 6]{DBLP:conf/eurocrypt/HsiaoLR07}).  

Unfortunately, the conditional unpredictability entropy $\tilde{m}$ must decrease as $t$ increases, as the following theorem states.  (The proof of the theorem, generalized to more metric spaces, is in \secref{sec:proof of imp unp entropy}.)

\begin{theorem}
\label{thm:imp of unp entropy}
Let $\mathcal{Z}$ be an alphabet $(\sketch, \rec)$ be an unpredictability-entropy $(\mathcal{Z}^n, m, \tilde{m}, t)$-secure sketch that is $(\epsilon, s_{sec})$-secure with error $\delta$.  If $s_{sec} \geq t(|\rec|+n\log |\mathcal{Z}|)$, then $\tilde{m}\leq n\log |\mathcal{Z}| - \log |B_t(\cdot)| + \log(1-\epsilon -t\delta)$.
\end{theorem}
In particular, if the input is uniform, the entropy loss is about $\log |B_t(\cdot)|$.  As mentioned at the beginning of~\secref{sec:impossCompSecSketch}, essentially the same entropy loss can be achieved with information-theoretic secure sketches, by using the randomized code-offset construction. However, it is conceivable that unpredictability entropy secure sketches could achieve lower entropy loss with greater efficiency for some parameter settings.


%This result has a weaker condition than \corref{cor:rec yields sketch} and yields a weaker conclusion.  The main difference with \thref{thm:impSketchArbitraryW} is that it is unknown if the indistinguishable distribution contains many points.  This means we cannot construct a good code out of the points of $Y$.  \lnote{I find the previous three sentences more confusing than helpful; do we really need them?  should we reword or remove?} %Our bound is a result of the Hamming bound for the space.  

\subsection{Avoiding sketch entropy upper bounds}
\label{ssec:avoiding bounds}

The lower bounds of \corref{cor:rec yields sketch} and \thref{thm:imp of unp entropy} are strongest for high entropy sources.  
%The results get weaker as we consider sources with lower starter entropy.  
This is necessary, if a source contains only codewords (of an error correcting code), no sketch is needed, and thus there is no (computational)~entropy loss.  
%For example, we can not rule out a computational secure sketch with entropy loss for low entropy sources.
%This means we cannot rule out computational sketches for other arbitrary lower entropy distributions.  
This same situation occurs when considering lower bounds for information-theoretic sketches~\cite[Appendix C]{DBLP:journals/siamcomp/DodisORS08} .

Both of lower bounds arise because \rec must function as an error-correcting code for many points of any indistinguishable distribution.  It may be possible to avoid these bounds if \rec outputs a fresh random variable\footnote{If some efficient algorithm can take the output of $\rec$ and efficiently transform it back to the source $W$, the bounds of \corref{cor:rec yields sketch} and \thref{thm:imp of unp entropy} both apply.  This means that we need to consider constructions that are hard to invert~(either information-theoretically or computationally).}.  Such an algorithm is called a computational fuzzy conductor.  See~\cite{KanukurthiR09} for the definition of a fuzzy conductor.  To the best of our knowledge, a computational fuzzy conductor has not been defined in the literature, the natural definition is to replace the pseudorandomness condition in \defref{def:comp fuzzy extractor} with a HILL entropy requirement.  

Our construction~(in \secref{sec:fuzzyCompExt}) has pseudorandom output and immediately satisfies definition of a computational fuzzy extractor~(\defref{def:comp fuzzy extractor}).  It may be possible to achieve significantly better parameters with a construction that is a computational fuzzy conductor~(but not a computational fuzzy extractor) and then applying an extractor.  We leave this as an open problem.


%\subsection{Defining  Computational Fuzzy Conductors}
%\label{sec:unp secure sketch}
%In Sections \ref{sec:imp HILL sketch} and \ref{sec:imp unp sketch}, we showed lower bounds on an entropy drop of computational sketches.  If, the error correcting algorithm outputs a new random variable, bounds do not apply.  Instead of building a computational sketch, we will build a computational conductor.  We adapt the definition of a fuzzy conductor from~\cite{KanukurthiR09} to the computational setting:
%%\begin{definition}\protect{\cite[Definition]{KanukurthiR09}}
%%The procedures $(\sketch, \rec)$ are an $(\mathcal{M}, m, \tilde{m},t, \epsilon)$-\emph{weakly robust fuzzy conductor} if:
%%\begin{enumerate}
%%\item Error Tolerance: If $\dis(w, w')\leq t$ and $s\leftarrow \sketch(w)$, then $\Pr[\rec(w', \sketch(w)) \neq  \rec(w, \sketch(w))] \leq \epsilon$ where the probability is over the coins of $\sketch, \rec$.
%%\item Security of $\sketch$: If $H_\infty(W) \geq m$, then $\Hav(\rec(W, \sketch(W)) | \sketch(W))\geq \tilde{m}$.
%%\end{enumerate}
%%\end{definition}
%%We will essentially build %if we define the secure sketch using unpredictability entropy~\cite[Section 5]{DBLP:conf/eurocrypt/HsiaoLR07}, this oracle is not available to the adversary.  We begin by defining unpredictability entropy:
%
%
%%We now provide a definition of a robust computational sketch using unpredictability entropy:
%\begin{definition}\label{def:comp secure sketch}
%A pair of randomized procedures $(\sketch, \rec)$ is a $(\mathcal{M},m, \tilde{m}, \epsilon,  t)$-\emph{computational fuzzy conductor} if the following properties hold:
%\begin{enumerate}
%\item The sketching procedure \sketch on input $w\in\mathcal{M}$ returns a bit string $s\in\{0,1\}^*$ and runs in expected time $s_{sketch}$.
%\item The recovery procedure \rec takes an element $w'\in\mathcal{M}$ and a bit string $s\in\{0,1\}^*$.  The \emph{correctness} property of secure sketches guarantees that if $d(w,w')\leq t$, then $\Pr[\rec(w',\sketch(w))\neq \rec(w, \sketch(w))] = \delta$ and \rec runs in expected time $s_{rec}$.  If $\dis(w,w')>t$, then no guarantee is provided about the output of \rec.  
%\item The \emph{security} property guarantees that for any distribution $W$ over $\mathcal{M}$ with min-entropy $m$, the output $\rec$ has high HILL entropy given $\sketch(W)$.  That is, $H^{\hill}_{\epsilon, s_{sec}}(\rec(W, \sketch(W))|\sketch(W))\geq \tilde{m}$.
%\end{enumerate}
%We say that $(\sketch, \rec)$ is $(\epsilon, s_{sec})$-secure with error probability $\delta$.
%If $m=\tilde{m}$ we say a conductor is \emph{lossless} and we omit the parameter $\tilde{m}$.
%\end{definition}
%
%%We make $s_{sketch}, s_{rec}$ explicit in the definition as we are considering computationally bounded adversaries and the definition is only meaningful if both $s_{rec}$ and $s_{sketch}$ are significantly smaller than $s_{sec}$.  
%In \secref{sec:comp fuzzy extractors}, we show a natural construction of a computational fuzzy extractor from a computational fuzzy conductor that meets \defref{def:comp secure sketch}.  We show in the next section that a lossless construction is achievable for the uniform distribution using the $\LWE$ problem.

\section{Computational Fuzzy Extractor based on \class{LWE}}
\label{sec:fuzzyCompExt}

In this section we describe our main construction.  Security of our construction depends on the source $W$. We first consider  a uniform source $W$; we consider other distributions in \secref{sec:LWE block fixing sources}.  Our construction uses the code-offset construction~\cite{JW99}, \cite[Section 5]{DBLP:journals/siamcomp/DodisORS08} instantiated with a random linear code over a finite field $\Fq$.   Let $\decode_t$ be an algorithm that decodes a random linear code with at most $t$ errors (we will present such an algorithm later, in \secref{sec:time main construction}). 

\begin{construction}
Let $n$ be a security parameter and let $m\ge n$.  Let $q$ be a prime. %Let $W = W_1,..., W_m$ and let $b$ be a constant and let $W_i \in \zo^b$ for all $i$.  
Define $\gen, \rep$ as follows:%The following construction is a computational secure sketch:
%Code-offset construction of a secure sketch using a random linear code over a finite field $\Fq$. $\decode_t (\vA, \vect{D})$ denotes an algorithm that decodes up to $t$ errors, i.e., an algorithm that finds $X'\in \Fq^n$ such that $\vect{A}X'$ is of Hamming distance at most $t$ from $\vect{D}$. $m > n$ are parameters to be figured out later.
\begin{center}
\begin{tabular}{c|c}
\begin{minipage}{3in}
\textbf{\gen}
\begin{enumerate}
\item \underline{Input}: $w\leftarrow W$ (where $W$ is some distribution over $\Fq^m$).
\item Sample $\vA\in\Fq^{m\times n}, \vx\in\Fq^n$ uniformly.
\item Compute $p = (\vA, \vA \vx+w)$, \\\ $r = \vx_{1,...,n/2}$.
\item Output $(r, p)$.
\end{enumerate}
 \end{minipage} &
\begin{minipage}{3in}
\textbf{\rep}
\begin{enumerate}
\item \underline{Input}: $(w', p)$ (where the Hamming distance between $w'$ and $w$ is at most $t$).
\item Parse $p$ as $(\vA, \vect{c})$; let $\vb=\vect{c}-w'$.
\item Let $x = \decode_t(\vA, \vb)$\\
\item Output $r = x_{1,...,n/2}$.
\end{enumerate}
\end{minipage} 
\end{tabular}
\end{center}
\label{cons:informal construction}
\end{construction}


Intuitively, security comes from the computational hardness of decoding random linear codes with a high number of errors (introduced by $w$).  
In fact, we know that decoding a random linear code is NP-hard~\cite{berlekamp1978}; however, this statement is not sufficient for our security goal, which is to show  $\delta^{\mathcal{D}_{s_{sec}}}((X_{1,..., n/2},P), (U_{n/2 \log q}, P))\leq \epsilon$.  Furthermore, this construction is only useful if $\decode_t$ can be efficiently implemented. 

The rest of this section is devoted to making these intuitive statements precise.
 We describe the \class{LWE} problem and the security of our construction in \secref{subsec:LWE}.
 %a recent variant due to D\"{o}ttling and M\"{u}ller-Quade~\cite{dottling2012} in \secref{subsec:LWE}.  
We describe one possible polynomial-time $\decode_t$ (which corrects more errors than is possible by exhaustive search) in \secref{sec:time main construction}.  In \secref{sec:lossless extractor}, we describe parameter settings that allow us to extract as many bits as the input entropy, resulting in a lossless construction.  In \secref{sec:prg based comparison}, we compare \consref{cons:informal construction} to using a sketch-and-extract approach (\lemref{lem:fuzzy ext construction}) instantiated with a computational extractor. 



\subsection{Security of \consref{cons:informal construction}}
\label{subsec:LWE}
The $\LWE$ problem was introduced by Regev \cite{regev2005LWE, regevLWEsurvey} as a generalization of ``learning parity with noise." For a complete description of the $\LWE$ problem and related lattices problems~(which we do not define here) see~\cite{regev2005LWE}.  We now recall the definition of the decisional version of the problem. 


\begin{definition}[Decisional $\lwe$]\label{def:dist-LWE}
Let $n$ be a security parameter.  
Let $m = m(n) = \poly(n)$ be an integer and $q = q(n) = \poly(n)$ be a prime\footnote{%
Unlike in common formulations of LWE, where $q$ can be any integer, we need $q$ to be prime for decoding.}.
%
Let $\vA$ be the uniform distribution over  $\Fq^{m\times n}$, $X$ be the uniform distribution over $\Fq^n$ and $\chi$ be an arbitrary distribution on $\Fq^m$.
 The decisional version of the $\LWE$ problem, denoted \class{dist}-$\LWE_{n, m, q, \chi}$, is to distinguish the distribution
$(\vA, \vA X+\chi)$ from
 the uniform distribution over $(\Fq^{m\times n}, \Fq^m)$.

We say that $\distLWE_{n, m, q, \chi}$ is $(\epsilon, s_{sec})$-secure if no (probabilistic) distinguisher of size $s_{sec}$ can distinguish the $\lwe$ instances from uniform except with probability $\epsilon$.  If for any $s_{sec} = \poly(n)$, there exists   $\epsilon  = \ngl(n)$ such that  $\distLWE_{n, m, q, \chi}$ is $(\epsilon, s_{sec})$-secure, then we say  it is \emph{secure}.
\end{definition}

 Regev\cite{regev2005LWE} and Peikert \cite{peikert2009latticereduction} show that $\class{dist}$-$\lwe_{n, m, q, \chi}$ is secure when the distribution $\chi$ of errors is Gaussian, as follows.
Let $\bar{\Psi}_\rho$ be the discretized Gaussian distribution with variance $(\rho q)^2/2\pi$, where $\rho \in (0,1)$ with $\rho q > 2\sqrt{n}$.  If GAPSVP and SIVP are hard to approximate~(on lattices of dimension $n$) within polynomial factors for quantum algorithms, then $\distLWE_{n, m, q, \bar{\Psi}_\rho^m}$ is secure.  (A recent result of Brakerski et al.~\cite{brakerski2013classical} shows security of $\LWE$ based on hardness of approximating lattices problems for classical algorithms.  We have not considered how this result can be integrated into our analysis.)

The above formulation of $\LWE$ requires the error term to come from the discretized Gaussian distribution, which makes it difficult to use it for constructing fuzzy extractors (because using $w$ and $w'$ to sample Gaussian distributions will increase the distance between the error terms and/or reduce their entropy).
%\subsection{LWE with uniform errors}
%\label{subsec:LWE uniform error}
Fortunately, recent work D\"{o}ttling and M\"{u}ller-Quade~\cite{dottling2012} shows the security of $\LWE$, under the same assumptions, when errors come from the uniform distribution over a small interval\footnote{Micciancio and Peikert provide a similar formulation in~\cite{micciancio2013hardness}.  The result D\"{o}ttling and M\"{u}ller-Quade provides better parameters for our setting.}.  This allows us to directly encode $w$ as the error term in an $\LWE$ problem by splitting it  into $m$ blocks.  The size of these blocks is dictated by the following result of D\"{o}ttling and M\"{u}ller-Quade:
%As stated above this has two advantages for Construction~\ref{cons:LWESecureSketch}: 1) the bits of $W$ can be used directly and \rep can output $w$ instead of $sam_w$ 2) there is a fixed number of bits required to sample each dimension.  
%We present the formulation of D\"{o}ttling and M\"{u}ller-Quade: 
%\begin{lemma}~\protect{~\cite[Theorem 5]{dottling2012}}
%\label{lem:uniform LWE}
%Let $n$ be a security parameter.  Let $q = q(n))$ and $m = m(n) = \poly(n)$ be integers.  Let $\rho = \rho(n) \in (0,1)$ be such that $\rho q\geq 2n^2m$.  Assume there exists a PPT-algorithm that solves the LWE decision problem where $\vA,X$ are drawn uniformly at random and $E$ is drawn from the interval $[-\rho q, \rho q]$ with non-negligible probability.  Then there exists an efficient quantum-algorithm that approximates the decision-version of the shortest vector problem (GAPSVP) and the shortest independent vectors problem (SIVP) to within $\tilde{O}(n^{5/2}m/\rho)$ in the worst case.
%\end{lemma}

%This problem can be stated in the language of indistinguishability using the Decision-to-search reduction of Regev or Peikert:
\begin{lemma}~\protect{~\cite[Corollary 1]{dottling2012}}
\label{lem:uniform LWE decision}
Let $n$ be a security parameter.  Let $q = q(n) = \poly(n)$ be a prime and $m = m(n) = \poly(n)$ be an integer with $m\ge 3n$. Let $\sigma \in (0, 1)$ be an arbitrarily small  constant and let $\rho=\rho(n)\in (0,1/10)$ be such that $\rho q \geq 2n^{1/2+\sigma}m$. If the approximate decision-version of the shortest vector problem (GAPSVP) and the shortest independent vectors problem (SIVP) are hard within a factor of $\tilde{O}(n^{1+\sigma}m/\rho)$ for quantum algorithms in the worst case, then, for $\chi$ the uniform distribution over $[-\rho q, \rho q]^m$,  $\distLWE_{n, m, q, \chi}$ is secure.
\end{lemma}

To extract pseudorandom bits, we use a result of Akavia, Goldwasser, and Vaikuntanathan~\cite{akavia2009} to show that $X$ has simultaneously many hardcore bits.  The result says that if $\distLWE_{(n-k, m, q, \chi)}$ is secure then any $k$ variables of $X$ in a $\distLWE_{(n, m, q, \chi)}$ instance are hardcore.  We state their result for a general error distribution~(noting that their proof does not depend on the error distribution):
\begin{lemma}\protect{\cite[Lemma 2]{akavia2009}}
\label{lem:many hardcore bits}
 If $\distLWE_{(n-k, m, q, \chi)}$ is $(\epsilon, s_{sec})$ secure, then
\[\delta^ {\mathcal{D}_{s_{sec'}}} ((X_{1,\dots, k}, \vA, \vA X+\chi) , (U, \vA, \vA X+\chi)) \le \epsilon\,,\]
where $U$ denotes the uniform distribution over $\Fq^k$,  $\vA$ denotes the uniform distribution over $\Fq^{m\times n}$, $X$ denotes the uniform distribution over $\Fq^n$, $X_{1,\dots, k}$ denote the first $k$ coordinates of $x$, and $s_{sec}' \approx s_{sec} - n^3$.
\end{lemma}



%\begin{lemma}
%\label{lem:conversion to unpredictability}
%Assume that $\distLWE_{n-k, m, q, \chi}$ is $(\epsilon, s_{sec})$-hard then  $H^{\unp}_{\epsilon', s_{sec}'}(X_{1,..., k}|\vA, \vA X+E) \geq k\log q$
%for $\epsilon' \approx \epsilon, s_{sec}'\approx s_{sec}$.
%\end{lemma}
%\begin{lemma}
%\label{lem:dmq with unp entropy}
%Let $n$ be a security parameter.  Let $q = q(n) = \poly(n)$ be a prime integer and $m = m(n) = \poly(n)$ be an integer. Let $\gamma\in (0, 1)$ be a constant and let $\rho\in (0,1)$ such that $\rho q \geq 2n^{1/2+\sigma}m$.  Assume that approximating GAPSVP and SIVP using a polynomial time quantum algorithm is hard within a factor of $\tilde{O}(n^{1/2+\sigma}m/\rho)$.  Let $\vA\in \Fq^{m\times n}, X\in\Fq^{n}, E\in [-\rho q, \rho q]^m$ be drawn uniformly.  Then for $s_{sec} = \poly(n)$ and $\epsilon = \ngl(n)$
%\[
%H^{\unp}_{\epsilon, s_{sec}}(E| AX+E) = \min\{|X|, |E|\} = \min \{ n\log q, m\log \rho q\}
%\]
%\end{lemma}
The security of  \consref{cons:informal construction} follows from Lemmas \ref{lem:uniform LWE decision} and~\ref{lem:many hardcore bits} as long as all the parameters are set appropriately (see \thref{thm:lossless secure extractor log}),  because we use the hardcore bits of $X$ as our key.  
%\lemref{lem:uniform LWE decision} has two advantages: 1) We can use the code-offset construction from \consref{cons:informal construction}. 2) The error term is distributed over a smaller interval $[-\rho q, \rho q]$ instead of $\Fq$.

%\subsection{Computational Secure Sketch based on \class{LWE}}
%\label{subsec:fuzzyExtLWE}
%We now give a formal description of our sketch based on LWE.  We now consider a source $W$~(this fulfills the role of $E$ in standard LWE notation):
%
%\begin{construction}[Computation Secure Sketch based on LWE]
%\label{cons:LWESecureSketch} Let $n$ be a security parameter and let $m = m(n) = \poly(n), q = q(n)\geq 2$ be integers.  Let $\gamma\in (0,1)$ be a constant and let $\rho \in (0,1)$ such that $\rho q\geq 2n^{1/2+\sigma}m$.  Let $\decode_t$ be an algorithm~(not necessarily efficient) that inverts an LWE instance when no more than $t$ of $m$ dimensions have non-zero error.  Let $W$ be a distribution over $\{0,1\}^{(\log \rho q)\times m}$\,.  Then the following is a computational secure sketch:%We present the construction in 
%%\begin{figure}
%%\label{fig:construction figure}
%%\caption{Computational Secure Sketch based on the Learning with Errors problem}
%\begin{center}
%\begin{tabular}{c|c}
%\begin{minipage}{3in}
%\textbf{\sketch}
%\begin{enumerate}
%\item Input $w\leftarrow W$.
%\item Sample $\vA\in\Fq^{m\times n}, x\in\Fq^n$ uniformly.
%\item Output $p = (\vA, \vA x+w)$.\\
%\end{enumerate}
% \end{minipage} &
%\begin{minipage}{3in}
%\textbf{\rec}
%\begin{enumerate}
%\item Input $(w', p)$
%\item Parse $p$ as $(\vA, \vect{C})$
%\item Set $x' = \decode_t(\vA, \vect{C}-w') $. 
%\item Output $w = \vect{C}-\vA x'$.
%\end{enumerate}
%\end{minipage} 
%\end{tabular}
%\end{center}
%%\end{figure}
%%\textbf{\gen}
%%\begin{enumerate}
%%\item Input $w\leftarrow W$.
%%\item Sample $A\in\Fq^{m\times n}, x\in\Fq^n$ uniformly at random.
%%\item Use $w$ as the randomness for the sampling algorithm, \\$\sample$, for $\chi$.  Set $E\leftarrow  \sample(w)$.
%%\item Output $p = (A, AX+E)$.
%%\end{enumerate}
%%
%%
%%\textbf{\rep}
%%\begin{enumerate}
%%\item Input $(w', p)$
%%\item Parse $p$ as $(A, C)$
%%\item Compute $E' \leftarrow \sample (w')$.
%%\item Set $X' = \decode_t(A, C-E') $. 
%%\item Output $sam_w = C-AX'$.
%%\end{enumerate}
%\end{construction}

\subsection{Efficiency of \consref{cons:informal construction}}
\label{sec:time main construction}
%\subsection{Analysis of Construction~\ref{cons:LWESecureSketch}}
\consref{cons:informal construction} is useful only if $\decode_t$ can be efficiently implemented.  We need a decoding algorithm for a random linear code with $t$ errors that runs in polynomial time.  We present a simple $\decode_t$ that runs in polynomial time and can correct
correcting $O(\log n)$ errors (note that this corresponds to a superpolynomial number of possible error patterns).
This algorithm is only a proof of concept, and neither the algorithm nor its analysis have been optimized for constants. An improved decoding algorithm can replace our algorithm, which will increase our correcting capability and improve \consref{cons:informal construction}.


\begin{construction}
\label{cons:decoding algorithm} We consider a setting of $(n, m, q, \chi)$ where $m\geq 3n$.  We describe $\decode_t$:
\begin{enumerate}
\item Input $\vA , \vb = \vA \vx + w - w'$
\item Randomly select rows without replacement $i_1,..., i_{2n}\leftarrow [1,m]$.  
\item Restrict $\vA, \vb$ to rows $i_1,...,i_{2n}$; denote these $\vA_{i_1,...,i_{2n}}, \vb_{i_1,...,i_{2n}}$.
\item Find $n$ rows of $\vA_{i_1,..., i_{2n}}$ that are linearly independent.  
If no such rows exist, output $\perp$ and stop.
\item Denote by $\vA', \vb'$ the restriction of $\vA_{i_1,..., i_{2n}}, \vb_{i_1,..., i_{2n}}$ (respectively) to these rows. Compute $\vx' = (\vA')^{-1}\vb'$.  
\item If $\vb- \vA \vx'$ has more than $t$ nonzero coordinates, go to step (2).
\item Output $\vx'$.
\end{enumerate}
\end{construction}


Each step is computable in time $O(n^3)$. 
For $\decode_t$ to be efficient, we need $t$ to be small enough so that  with probability at least $\frac{1}{\poly(n)}$, none of the $2n$ rows  selected  in step 2 have errors (i.e., so that $w$ and $w'$ agree on those rows).  If this happens, and $\vA_{i_1,...,i_{2n}}$ has rank  $n$ (which is highly likely), then $\vx'=\vx$, and the algorithm terminates.  However, we also need to ensure correctness: we need to make sure that if $\vx'\neq \vx$, we detect it in step 6.  This detection will happen if $\vb-\vA \vx' = \vA (\vx-\vx')+(w-w')$ has more than $t$ nonzero coordinates.  It suffices to ensure that $\vA (\vx-\vx')$ has at least $2t+1$ nonzero coordinates (because at most $t$ of those can be zeroed out by $w-w'$), which happens whenever the code generated by $\vA$ has distance $2t+1$.

Setting $t = O(\frac{m}{n}\log n)$ is sufficient to ensure efficiency.    Random linear codes have distance at least $O(\frac{m}{n}\log n)$ with probability $1-e^{-\Omega(n)}$ (the exact statement is in \corref{cor:code high distance}), so this also ensures correctness.
The formal statement is below~(proof in \secref{sec:proof lem i t poly time}):
%  We provide the results in \lemref{lem:i t constant time} and \lemref{lem:i t poly time} respectively~(proofs in Sections~\ref{sec:proof lem i t constant time},~\ref{sec:proof lem i t poly time}).  \lnote{can we combine these lemmas into one?  It's hard for the reader to compare line-by-line to see where the differences are.  More errors seems better; we could simply point out that constant number of erros leads to better running time, right in the statement of the lemma or right after it}
%\begin{lemma}[Efficiency of $\decode_t$ when $t\leq m/n-1$]
%\label{lem:i t constant time}
%Let $\vA, \vA X+E$ be a  $(n,m, q, \chi)$-LWE instance that is a unique witness relation~(except with negligible probability).  Assume $d(E, E')\leq t$ where $t \leq (m-n)/n$.  Then $\decode_t$ runs in expected time $O(n^3\log q+ s_{ver})$ on input distribution $\vA, \vA X+E - E'$ and outputs $X' =X$ for all but a negligible fraction of inputs over $\vA, \vA X+E - E'$.
%\end{lemma}
\begin{lemma}[Efficiency of $\decode_t$ when $t\leq d (m/n-2)\log n$]
\label{lem:i t poly time}
%Let $\vA, \vA X+W$ be a  $\distLWE_{n,m, q, \chi}$ instance where $m\geq 3n$.  Assume that $W$ is split into $m$ blocks of length $b$ and let $\dis$ be the Hamming distance over alphabet $2^b$. 
Let $d$ be a positive constant and assume that $\dis(W, W')\leq t$ where $t\leq d(\frac{m}{n}-2)\log n$.  Then $\decode_t$ runs in expected time $O(n^{4d+3})$ operations in $\Fq$~(this expectation is over the choice of random coins of $\decode_t$, regardless of the input, as long as $\dis(w, w')\le t$).  It outputs $X$ with probability $1-e^{-\Omega(n)}$ (this probability is over the choice of the random matrix  $\vA$ and random choices made by $\decode_t$).
\end{lemma}

%\textbf{Problems:} Construction~\ref{cons:LWESecureSketch} does not meet Definition~\ref{def:comp secure sketch}.  This is because the sampling algorithm may not be invertible.  Furthermore, in standard $\LWE$ the sampling algorithm is Gaussian and there is no fixed number of bits used to sample error in any dimension.  Approximating the Gaussian distribution using a fixed number of bits may be possible but there are still two error patterns which require a significantly different number of bits.  We will address both of these problems by sampling the error from a uniform distribution.

\subsection{Lossless Computational Fuzzy Extractor}
\label{sec:lossless extractor}
We now state a setting of parameters that yields a lossless construction.  The intuition is as follows.  We are splitting our source into $m$ blocks each of size $\log \rho q$~(from \lemref{lem:uniform LWE decision}) for a total input entropy of $m\log \rho q$.  Our key is derived from hardcore bits of $X$: $X_{1,\dots, k}$ and is of size $k \log q$~(from \lemref{lem:many hardcore bits}). Thus, to achieve a lossless construction we need $k \log q = m\log \rho q$.
In other words, in order to decode a meaningful number of errors, the vector $w$ is of higher dimension than the vector $X$, but each coordinate of $w$ is sampled using fewer bits than each coordinate of $X$.    Thus, by increasing the size of $q$ we can set $k\log q = m\log \rho q$, yielding a key of the same size as our source.    The formal statement is below. 

%\begin{theorem}
%\label{thm:lossless secure sketch}
%Let $n$ be a security parameter and let $t$ be a constant.  Consider the Hamming metric with block length $b = \log 2n^3(t+1)$.  Let $W$ be uniform over $\zo^{n(t+1)b}$.  If GAPSVP and SIVP are hard to approximate within polynomial factors, there is a 
%\[
%(\zo^{n(t+1)b}, |W|,|W|, \epsilon, s, s_{sketch}, s_{rec}, t)\text{-computational secure sketch}
%\]
% for $\epsilon = \ngl(n), s = \poly(n), s_{sketch} = O(n^3\log n )$ and $s_{rec}= O(n^3 \log n)$.
%\end{theorem}

\begin{theorem}
\label{thm:lossless secure extractor log}
Let $n$ be a security parameter and let the number of errors $t = c\log n$ for some positive constant $c$.    Let $d$ be a positive constant (giving us a tradeoff between running time of $\rep$ and $|w|$). Consider the Hamming metric over the alphabet $\mathcal{Z}=[-2^{b-1},2^{b-1}]$, where  $b = \log 2(c/d+2) n^2 =O(\log n)$.  Let $W$ be uniform over $\mathcal{M}=\mathcal{Z}^m$, where $m={(c/d+2)n}=O(n)$.  If GAPSVP and SIVP are hard to approximate within polynomial factors using quantum algorithms, then there is a setting of $q = \poly(n)$ such that for any polynomial $s_{sec}=\poly(n)$ there exists $\epsilon=\ngl(n)$ such that the following holds: \consref{cons:informal construction} is a $(\M, W, m\log |\mathcal{Z}|, t)$-computational fuzzy extractor that is $(\epsilon, s_{sec})$-hard with error $\delta = e^{-\Omega(n)}$.
The generate procedure $\gen$ takes $O(n^2)$ operations over $\Fq$, and the reproduce procedure $\rep$ takes expected time $O(n^{4d+3})$ operations over $\Fq$.
% for following parameters:
%\[
%(\zo^{(c+d)/dn\log b}, |W|, \epsilon, s_{sec}, s_{sketch}, s_{rec}, t)
%\] 
\end{theorem}
\begin{proof}
Security follows by combining Lemmas~\ref{lem:uniform LWE decision} and~\ref{lem:many hardcore bits}; efficiency follows by \lemref{lem:i t poly time}. For a more detailed explanation of the various parameters and constraints see \secref{sec:parameter settings}.  %In particular, we describe how to set $q = \poly(n)$.  
\end{proof}


\thref{thm:lossless secure extractor log} shows that a computational fuzzy extractor can be built without incurring any entropy loss.  We can essentially think of $\vA X+W$ as an encryption of $X$ that where decryption works from any close $W'$.
%A natural question to ask is whether security degrades gracefully when $W$ is not the uniform distribution.  We begin to provide an answer in \secref{sec:LWE block fixing sources}.  

\subsection{Comparison with computational-extractor-based constructions}
\label{sec:prg based comparison}
As mentioned in the introduction, an alternative approach to building a computational fuzzy extractor is to use  a computational extractor (e.g.,~\cite{krawczyk2010cryptographic, barak2011leftover, dachman2012computational}) in place of the information-theoretic extractor in the sketch-and-extract construction.  We will call this approach \emph{sketch-and-comp-extract}.  (A simple example of a computational extractor is a pseudorandom generator applied to the output of an information-theoretic extractor; note that LWE-based pseudorandom generators exist~\cite{applebaum2006pseudorandom}.)
%using an information theoretic fuzzy extractor and then applying a pseudorandom generator to the output of the fuzzy extractor.  This is the extract-then-expand computational extractor suggested by Krawczyk in~\cite{krawczyk2010cryptographic}.  

This approach (specifically, its analysis via \lemref{lem:fuzzy ext construction}) works as long as the amount of entropy $\tilde{m}$ of $w$ conditioned on the sketch $s$ remains high enough to run a computational extractor.  However, as discussed in \secref{sec:impossCompSecSketch}, $\tilde{m}$ decreases with the error parameter $t$ due to coding bounds, and it is conceivable that, if $W$ has  barely enough entropy to begin with, it will have too little entropy left to run a computational extractor once $s$ is known.

In contrast, our approach does not require the entropy of $w$ conditioned on $p=(\vA, \vA X+w)$ to  be high enough for a computational extractor. Instead, we require that $w$ is not computationally recoverable  given $p$.  This requirement is weaker---in particular, in our construction, $w$ may have no information-theoretic entropy conditioned on $p$.  The key difference in our approach is that instead of extracting from $w$, we hide secret randomness using $w$. Computational extractors are not allowed to have private randomness \cite[Definition 3]{krawczyk2010cryptographic}.

The main advantage of our analysis (instead of sketch-and-comp-extract) is that security need not depend on the error-tolerance $t$.  In our construction, the error-tolerance depends only on the best available decoding algorithm for random linear codes, because decoding algorithms will not reach the information-theoretic decoding radius.

Unfortunately, LWE parameter sizes require relatively long $w$. Therefore, in practice, sketch-then-comp-extract will beat our construction  if the computational extractor is instantiated efficiently based on assumptions other than LWE (for example, a cryptographic hash function for an extractor and a block cipher for a PRG). However, we believe that our conceptual framework can lead to better constructions.  Of particular interest are  other codes that are easy to decode up to $t$ errors but become computationally hard as the number of errors increases.

%However, the construction of a computational fuzzy extractor by 
%The advantage of \consref{cons:informal construction} is that losses due to error correction $O(\log |B_t(\cdot)|)$ and randomness extraction $O(\log 1/\epsilon)$ are not necessary.  In the case where $W$ is small, these losses may leave too few bits for a secure pseudorandom generator.  The work of~\cite{dachman2012computational} shows that these losses in the extract-then-expand construction are in some sense necessary~(the samplable RT conjecture in~\cite{dachman2012computational} has been resolved positively by~\cite{dodis2013key}).  

To summarize, the advantage of \consref{cons:informal construction} is that the security of our construction does not depend on the decoding radius $t$.  
The disadvantages of \consref{cons:informal construction} are that it supports a limited number of errors and only a uniformly distributed source.  We begin to address this second problem in the next section.

\section{Computational Fuzzy Extractor for Nonuniform Sources}
\label{sec:LWE block fixing sources}
While showing the security of~\consref{cons:informal construction} for arbitrary high-min-entropy distributions is an open problem, in this section we show it for a particular class of distributions called symbol-fixing.   First we recall the notion of a symbol fixing source~(from~\cite[Definition 2.3]{KZ07}): 
\begin{definition}
Let $W = (W_1,..., W_{m+\alpha})$ be a distribution where each $W_i$ takes values over an alphabet $\mathcal{Z}$.  We say that it is a $(m+
\alpha, m, |\mathcal{Z}|) $ \emph{symbol fixing source} if for $\alpha$ indices $i_1, \dots, i_\alpha$, the symbols $W_{i_\alpha}$ are fixed, and the remaining $m$  symbols are chosen uniformly at random.  Note that $H_\infty(W)=m\log |\mathcal{Z}|$.
\end{definition}

Symbol-fixing sources are a very structured class of distributions.  However, extending \consref{cons:informal construction} to such a class is not obvious.  Although symbol-fixing sources are deterministically extractible~\cite{KZ07}, we cannot first run a deterministic extractor before using \consref{cons:informal construction}.  This is because we need to preserve distance between $w$ and $w'$ and an extractor must not preserve distance between input points.  We present an alternative approach, showing security of $\LWE$ directly with symbol-fixing sources.

The following theorem states the main technical result of this section, which is of potential interest outside our specific setting. The result is that $\distLWE$ with symbol-fixing sources is implied by standard $\distLWE$ (but for $n$ and $m$ reduced by the amount of fixed symbols).  
%\vspace{.1in}
\begin{theorem}
\label{thm:blockLWE}
Let $n$ be a security parameter, $m, \alpha$ be polynomial in $n$, and $q=\poly(n)$ be a prime and $\beta\in\mathbb{Z^+}$ be such that $q^{-\beta} = \ngl(n)$. 
Let $U$ denote the uniform distribution over $\mathcal{Z}^m$ for an alphabet $\mathcal{Z}\subset \Fq$, and let $W$ denote an $(m+\alpha, m, |\mathcal{Z}|)$ symbol fixing source over $\mathcal{Z}^{m+\alpha}$.
If $\distLWE_{n, m,q, U}$ is secure, then $\distLWE_{n+\alpha+\beta, m+\alpha, q, W}$ is also secure.
\end{theorem}

\thref{thm:blockLWE} also holds for an arbitrary error distribution~(not just uniform error) in the following sense.  Let $\chi'$ be an arbitrary error distribution.  Define $\chi$ as the distribution where $m$ dimensions are sampled according to $\chi'$ and the remaining dimensions have some fixed error.  Then, security of $\distLWE_{n, m, q, \chi'}$ implies security of $\distLWE_{n+\alpha+ \beta, m+\alpha, q, \chi}$.  We show this stronger version of the theorem in \apref{sec:proof of block theorem}.

The intuition for this result is as follows.  Providing a single sample with no error ``fixes'' at most a single variable.  Thus, if there are significantly more variables than samples with no error,  search $\LWE$ should still be hard.  We are able to show a stronger result that $\distLWE$ is still hard.  The nontrivial part of the reduction is using the additional $\alpha+ \beta$ variables  to ``explain'' a random value for the last $\alpha$ samples, without knowing the other variables.  The $\beta$ parameter is the slack needed to ensure that the ``free'' variables have influence on the last $\alpha$ samples.  A similar theorem for the case of a single fixed dimension was shown in concurrent work by Brakerski et al.~\cite[Lemma 4.3]{brakerski2013classical}.  The proof techniques of Brakerski et al. can be extended to our setting with multiple fixed dimensions, improving the parameters of \thref{thm:blockLWE}~(specifically, removing the need for $\beta$).

\thref{thm:blockLWE} allows us to construct a lossless computational fuzzy extractor from block-fixing sources: 

%\begin{theorem}
%\label{thm:lossless block sketch}
%Let $n$ be a security parameter and let $t$ be a constant.  Consider the Hamming metric with block length $b = \log 2n^3(t+1)$.  Let $W$ be an $\alpha$-symbol fixing source over $\zo^{((t+1)n+\alpha)b}$.  There is a \lnote{instead of ``there is'' can we actually point to the to construction?}
%\[
%(\zo^{((t+1)n+\alpha)b}, \Hoo(W),\Hoo(W), \epsilon, s, s_{sketch}, s_{rec}, t)\text{-computational secure sketch}
%\]
% for $s = \poly(n), s_{sketch} = O(n^3\log n )$ and $s_{rec}= O(n^3 \log n)$.
%\end{theorem}

\begin{theorem}
\label{thm:lossless block sketch log}
Let $n$ be a security parameter and let $t = c\log n$ for some positive constant $c$.  Let $d\le c$ be a positive constant and consider the Hamming metric over the alphabet $\mathcal{Z}=[-2^{b-1},2^{b-1}]$, where $b \approx \log 2(c/d+2)n^2 = O(\log n)$.  Let $\mathcal{M} = \mathcal{Z}^{m+\alpha}$ where $m= (c/d+2)n=O(n)$ and $\alpha \leq n/3$. 
Let $\mathcal{W}$ be the class of all $(m+\alpha, m, |\mathcal{Z}|)$-symbol fixing sources.  If GAPSVP and SIVP are hard to approximate within polynomial factors 
using quantum algorithms, then  there is a setting of $q = \poly(n)$ such that for any polynomial $s_{sec} = \poly(n)$
 there exists $\epsilon = \ngl(n)$ 
such that the following holds: \consref{cons:informal construction} is a $(\M, \mathcal{W}, m\log |\mathcal{Z}|, t)$-computational fuzzy extractor that is $(\epsilon, s_{sec})$-hard with error $\delta = e^{-\Omega(n)}$.
 The generate procedure $\gen$ takes $O(n^2)$ operations over $\Fq$, and the reproduce procedure $\rep$ takes expected time $O(n^{4d+3} \log n)$ operations over $\Fq$.
\end{theorem}

\begin{proof} Security follows by Lemmas~\ref{lem:uniform LWE decision} and~\ref{lem:many hardcore bits} and  \thref{thm:blockLWE} .  %Recall that \thref{thm:blockLWE} requires that $q^{-\beta} = \ngl(n)$.  Since, \lemref{lem:uniform LWE decision} requires that $q = O(n^{3/2})$, the condition of \thref{thm:blockLWE} are satisfied by any  $\beta = \omega(1)$.  
Efficiency follows by \lemref{lem:i t poly time}.  For a more detailed explanation of parameters see \secref{ssec:block params}. 
\end{proof}

\section*{Acknowledgements}
The authors are grateful to Ran Canetti, Yevgeniy Dodis, Nico D\"{o}ttling, Danielle Micciancio, J\"{o}rn M\"{u}ller-Quade, Christopher Peikert, Oded Regev, Adam Smith, and Daniel Wichs for helpful discussions, creative ideas, and important references.  In particular, the authors thank Nico D\"{o}ttling for describing his result on LWE with uniform errors.  We thank Jacob Alperin-Sheriff for pointing out the connection with the work of Brakerski et al.  The work of Benjamin Fuller is sponsored by the United States Air Force under Air Force Contract FA8721-05-C-0002. Opinions, interpretations, conclusions and recommendations are those of the authors and are not necessarily endorsed by the United States Government.

\bnote{Need grants for Leo and Xianrui}
\bibliographystyle{alpha}
\bibliography{crypto}
\appendix
\section{Properties of Random Linear Codes}
For efficient decoding of \consref{cons:informal construction}, we need the $\LWE$ instance to have high distance with overwhelming probability.  We will use the $q$-ary entropy function, denoted $H_q(x)$ and defined as $H_q(x) = x\log _q(q-1) - x\log_q x - (1-x)\log_q (1-x)$.  Note that $H_2(x) = -x\log x - (1-x)\log (1-x)$.  In the region $[0, \frac{1}{2}]$ for any value $q'\geq q$, $H_{q'}(x)\leq H_{q}(x)$.  The following theorem is standard in coding theory:

\begin{theorem}~\cite[Theorem 8]{venkatLecture}
\label{thm:random code good distance}
For prime $q, \delta\in [0, 1-1/q), 0<\epsilon< 1-H_q(\delta)$ and sufficiently large $m$, the following holds for $n = \lceil (1-H_q(\delta) - \epsilon)m\rceil$ .  If $\vA \in \Fq^{m\times n}$ is drawn uniformly at random, then the linear code with $\vA$ as a generator matrix has rate at least $(1-H_q(\delta) -\epsilon)$ and relative distance at least $\delta$ with probability at least $1-e^{-\Omega(m)}$.
\end{theorem}
Our setting is the case where $m = poly(n)\geq 2n$ and $\delta = O (\log n /n)$.  This setting of parameters satisfies \thref{thm:random code good distance}:
\begin{corollary}
\label{cor:code high distance}
Let $n$ be a parameter and let $m = \poly(n)\geq 2n$.  
Let $q$ be a prime and $\tau = O(\frac{m}{n}\log n )$.  For large enough values of $n$, when $\vA\in \Fq^{m\times n}$ is drawn uniformly, the code generated by $\vA$ has distance at least $\tau$ with probability at least $1-e^{-\Omega(m)}\geq 1-e^{-\Omega(n)}$.
\end{corollary}
\begin{proof}
Let $c$ be some constant.  Let $\delta = \tau/m = \frac{c\log n}{n}$.  We show the corollary for the case when $m = 2n$~(increasing the size of $m$ only increases the relative distance).  It suffices to show that for sufficiently large $n$, there exists $\epsilon>0$ where $1- H_q(\frac{c\log n}{n}) - \epsilon = 1/2$ or equivalently that $H_q(\frac{c\log n}{m})< 1/2$ as then setting $\epsilon = 1/2-H_q(\frac{c\log n}{n})$ satisfies  \thref{thm:random code good distance}.  For sufficiently large $n$:
\begin{itemize}
\item $\frac{c\log n}{n}< 1/2$, so we can work with the binary entropy function $H_2$.  
\item $\frac{c\log n}{n}< .1 < 1/2$ and thus $H_q(\frac{c\log n}{n})< H_q(.1)$. 
\end{itemize}  Putting these statements together, for large enough $n$, $H_q(\frac{c\log n}{n})< H_q(.1) < H_2(.1)< 1/2$ as desired.  This completes the proof.
\end{proof}

We also need that random matrices are full rank with high probability~(to allow us to decode).  We use the following claim~(techniques from Cooper~\cite{cooper2000rank}):
\begin{claim}
\label{cl:full rank matrix}
Let $q\ge 2$ be a prime.  Let $\alpha, \beta$ be integers and let 
let $\vS \overset{\$}\leftarrow \Fq^{\alpha \times (\alpha+\beta)}$ be uniformly generated.  Then $\Pr[\rank(\vS)=\alpha] >  1- q^{-\beta}$.
\end{claim}
\begin{proof}
Let $p_i$ be the probability that the $i$th
row is linearly dependent on the previous $i-1$ rows. 
By the union bound, the probability that $\alpha$ rows are linearly dependent is bounded by 
$\sum_{i=1}^\alpha p_i$.
Since $i-1$ rows can span a space of size at most $q^{i-1}$, the probability $p_i$ that a randomly chosen $i$th row is in that space is at most $q^{i-1}/q^{\alpha+\beta}$. So
\begin{align*}
\Pr[\rank(\vS) < \alpha] &=\sum_{i=1}^{\alpha} \frac{q^{i-1}}{q^{\alpha+\beta}}
= \frac{q^{\alpha}-1}{q-1}\frac{1}{q^{\alpha+\beta} }< q^{-\beta}.
\end{align*}
\end{proof}

\section{Proof of Theorem \ref{thm:blockLWE}}
\label{sec:proof of block theorem}

%\bnote{We need to be clear about notation here.  Why are $x, e$ sometimes bold and sometimes capital?} 
\begin{proof}
We assume that all of the fixed blocks are located at the end and their fixed value is $0$.  If the blocks are fixed to some other value, the reduction is essentially the same.
 In the reduction, the distinguisher is allowed to depend on the source and can know the positions of the fixed blocks and their values.  For a matrix $\vA$ we will denote the $i$-th row by $\va_i$.  For a set $T$ of column indices, we denote by $\vA_T$ the restriction of the matrix $\vA$ to the columns contained in $T$.  Similarly, for a vector $\vx$ we denote by $\vx_T$ the restriction of $\vx$ to the variables contained in $T$.  We use similar notation for the complement of $T$, denoted $T^c$.  For a matrix or vector we use $\mathsf{T}$ to denote the transpose.  We use $i$ as a index into matrix rows and the error vector and $j$ as an  index into columns and the solution vector.

%It should be clear to the reader where the reduction differs for another block fixing source.\lnote{I don't like the phrasing; maybe we should say that if the blocks are fixed to some non-0 value, the reduction is essentially the same}

Let $n$ be a security parameter, $m ,q , \alpha= \poly(n)$.  Let $\beta$ be such that $q^{-\beta} = \ngl(n)$.  All operations are computed modulo $q$, and we omit $``\bmod q$'' notation.  Let $\chi'$ be some error distribution over $\Fq^m$ and let $\chi$ over $\Fq^{m+n}$ be defined by sampling $\chi'$ to obtain values on dimensions $1,..., m$ and then appending $\alpha$ 0s.  

Let $D$ be a distinguisher that breaks $\distLWE_{(m+\alpha), (n+\alpha+\beta), q, \chi}$ with advantage $\epsilon>1/\poly(n)$.
Let  $\vA$ denote the uniform distribution over $\Fq^{(m+\alpha)\times(n+\alpha+\beta)}$, $X$ denote the uniform distribution over $\Fq^{(n+\alpha+\beta)}$, and $U$ denote the uniform distribution over $\Fq^{m+\alpha}$ . Then
\[
|\Pr[D(\vA, \vA X+\chi) = 1] - \Pr[D(\vA, U )=1]|> \epsilon.
\]


We build a distinguisher that breaks $\distLWE_{m, n, q, \chi}$.  Let $\vA'$ denote the uniform distribution over $\Fq^{m\times n}$, $X'$ denote the uniform distribution over $\Fq^n$, and $U'$ denote the uniform distribution over $\Fq^{m}$ .  We will build a distinguisher $D'$ of polynomial size for which
\begin{align}
\label{eq:block LWE dist}
|\Pr[D'(\vA', \vA'X'+\chi') = 1] - \Pr[D'(\vA', U') =1]|> (\epsilon - \ngl(n))(1-\ngl(n)) \approx \epsilon.
\end{align}
$D'$  will make a single call to $D$, so we focus on how to prepare a random block-fixing instance for $D$ from the random instance that $D'$ is given.  The code for $D'$ is given in \figref{fig:perfectLWEreduction}.

\begin{figure}[p]
\begin{framed}
\begin{enumerate}
\item Input $\vA', \vb'$, where $\vA' \overset{\$} \leftarrow \Fq^{m\times n}$ and $\vb'$ is either uniform over $\Fq^m$ or $\vb' = \vA'\vx' +\ve'$ for $\ve'\overset{\$} \leftarrow \chi'$ and uniform $\vx'\  \overset{\$} \leftarrow \Fq^n$.
\item Choose $\vect{R} \overset{\$}\leftarrow \Fq^{\alpha \times n}$ uniformly at random. Initialize $\vQ \in \Fq^{m\times (\alpha+\beta)}$  to be the zero matrix.
\item Let $\vb^* = (\vb', b^*_{m+1}, \ldots,b^*_{m+\alpha})$, for uniformly chosen $(b^*_{m+1}, \ldots, b^*_{m+\alpha} )\overset{\$} \leftarrow \Fq^\alpha$.\label{step:b generation}
\item Choose $\vect{S} \overset{\$}\leftarrow \Fq^{\alpha \times (\alpha+\beta)}$ uniformly at random.
		\subitem If $\rank(\vect{S})<\alpha$, stop and output a random bit.
\item Find a set of $\alpha$ linearly independent columns in $\vS$.  Let $T$ be the set of indices of these columns.\label{step:find columns}
\item For all $1\le j \le \alpha+\beta$, $j\notin T$:
\label{step:fill in matrix}
\subitem Choose $x_{n+j}\overset{\$}\leftarrow \Fq$ uniformly at random.  
\subitem For $i=1,..., m$:
\subsubitem Choose $\vQ_{i,j}\overset{\$}\leftarrow \Fq$ uniformly at random.
\subsubitem Set $b_i^* = b_i^* + \vQ_{i,j} x_{n+j}$.
\item Initialize $\vA^*  = \left(\begin{array}{c | c}\vA' & \vQ\\\hline \vR & \vS\end{array}\right)$.
\item \label{step:randomization}
For {$i=1,..., m$}:
\subitem Choose a row vector $\gamma_i \leftarrow \Fq^{1 \times \alpha}$ uniformly at random.
\subitem Set $\va_{i} \leftarrow \va^*_{i}+\gamma_i (\vR||\vS)$
\subitem Set $b_i \leftarrow b^*_i + \gamma_i (b^*_{m+1},..., b^*_{m+\alpha})^{\mathsf{T}}$
\item For $i=m+1,\dots, m+\alpha$:
\subitem Set $\va_i \leftarrow \va^*_i$
\subitem Set $b_i = b_i^*$.
\item Output $D(\vA, \vb)$. 
\end{enumerate}
\end{framed}
\caption{A PPT $D'$ that distinguishes LWE using distinguisher for LWE w/ block fixing source}
\label{fig:perfectLWEreduction}
\end{figure}

The distinguisher $D'$ has an advantage when $\vS$ is of rank $\alpha$.  This occurs with overwhelming probability:
\begin{claim}
\label{cl:full rank matrix 5.2}
Let $\vS \overset{\$}\leftarrow \Fq^{\alpha \times (\alpha+\beta)}$ be randomly generated.  Then $\Pr[\rank(\vS)=\alpha]\geq  1- \ngl(n)$.
\end{claim}
\begin{proof}
Direct result of \clref{cl:full rank matrix} because $q^{-\beta} = \ngl(n)$.
\end{proof}
The probability that a random $\vS$ is not full rank is $\ngl(n)$ so the distinguisher $D$ must still have an advantage when the matrix $\vS$ is full rank.  That is,
\[
|\Pr[D(\vA, \vA X+\chi) = 1  | \rank(\vS) = \alpha] - \Pr[D(\vA, U) =1 | \rank(\vS) = \alpha]|> \epsilon - \ngl(n).
\]

It suffices to show that $D'$ prepares a good instance for $D$ conditioned on $\vS$ being full rank. We show this in the following three claims: 
\begin{enumerate}
\item If $\vA'$ is a random matrix then $\vA$ is a random matrix subject to the condition that $\rank(\vS) = \alpha$.
%\bnote{This should be if $\vb' | A$ is uniformly distributed, then $\vb |A $ is uniformly distributed.}
\item If $\vb' = \vA'\vx'+\ve'$ for uniform $\vA'$ and $\vx'$, then $\exists \vx$~(uniformly distributed and independent of $\vA$ and $\ve'$) such that $\vb = \vA \vx + \ve$, where $\ve_i = \ve_i'$ for $1\leq i\leq m$ and $\ve_i = 0$ otherwise.
\item If the conditional distribution $\vb'\,|\,\vA'$ is uniform, then the conditional distribution $\vb\,|\,\vA$ is also uniform.
\end{enumerate}

\begin{claim}
\label{cl:randomMatrixDist}
The matrix $\vA$ is distributed as a uniformly random choice from the set of all matrices whose bottom-right $\alpha\times (\alpha+\beta)$ submatrix $\vS$ satisfies $\rank(\vS) = \alpha$.
\end{claim}
\begin{proof}
The bottom $\alpha$ rows of $\vA$ (namely, $\vR|\vS$) are randomly generated~(conditioned on $\rank(\vS) =\alpha$).  The top left $m\times n$ quadrant of $\vA$ is also random, because it is produced as a sum of a uniformly random $\vA'$ with some values that are uncorrelated with $\vA'$.
The submatrix of the top-right $m\times (\alpha+\beta)$ quadrant corresponding to $\vQ_{T^c}$~(recall this is the restriction of $\vQ$ to the columns not in $T$) is also random, because it is initialized with random values to which some uncorrelated values are then added. It is important to note that all these values are independent of $\gamma_i$ values.

Thus, we restrict attention to the $m\times \alpha$ submatrix of $\vA$ that corresponds to $\vQ_T$ in $\vA^*$ (note that these values are $0$ in $\vA^*$).  Consider a particular row $i$. That row is computed as $\gamma_i \vS_{T}$.  Since $\vS_T$ is a full rank square matrix and $\gamma_i$ is uniformly and independently generated, that row is also uniform and independent of other entries in $\vA$.
% In the randomized matrix $\vect{A}$, for $1\le i \le m$ and $n+1\le j \le n+\alpha$, $A_{ij}$ is assigned the value $\gamma_{i,(j - n)}$, which is a random value.  Similarly, for $m+1\le i \le m+\alpha$ and $n+1\le j \le n+\alpha$, each entry $A_{ij}$ is random due to the original construction of the matrix $\vA^*$.  Then before step \ref{step:randomization} the matrices $\vA', \vect{R}$ are random submatrices.  Thus, it remains to show that $\vA', \vect{R}$ remain random after step 5. For $1\le i \le m+\alpha$ and $1\le j \le n$, by construction $A_{ij} = A_{ij}'+\sum_{k=1}^\alpha \gamma_{i, k} R_{kj}$, which is a truly random number. Overall, each row vector $\va_{i}$ in $\vA$ is the sum of a random vector and a random linear combination independent vectors. Therefore, the entire matrix $\vA$ is a truly random matrix.  This completes the claim.
\end{proof}
\begin{claim}
\label{cl:random ax+e}
If $D'$ is provided with input distributed as $\vA', \vb' = \vA'\vx'+\ve'$ then $\vb = \vA \vx+\ve$, where
\begin{itemize}
\item $e_i = e_i'$ for $1\leq i\leq m$,
\item $e_i = 0$ for $m<i\leq m+\alpha$,
\item $x_j = x_j'$ for $1\leq j \leq n$,
\item and $x_j$ is uniform and independent of $\vA$ and $\ve'$ for $n<j\le n+\alpha+\beta$,
\end{itemize}
\end{claim}
\begin{proof}
Partially define $\vx$ as $x_j = x_j'$ if $1\leq j \leq n$ and $x_j$ as the value generated in step~\ref{step:fill in matrix} for $j>n $ and $j\not\in T$.  Define the remaining variables $\vx_T$ as the solution to the following system of equations.
\begin{eqnarray}
\vS_T  \vx_T = \begin{pmatrix} b_{m+1}^*  \\ \vdots \\b_{m+\alpha}^* \end{pmatrix}  - \vR \vx' - \vS_{T^c} \vx_{T^c}  \label{eq:x t solution}
\end{eqnarray}
A solution $\vx_T$ exists as $\vS_T$ is full rank. Moreover, it is uniform and independent of $\vA$ and $\ve$, because $b^*_{m+1}, \dots, b^*_{m+\alpha}$ are uniform and independent of $\vA$ and $\ve$. 

We now show that $\vb^* = \vA^* \vx+\ve$.  All entries in matrix $\vQ$ corresponding to variables in $T$ are set to zero.  Thus, the values of $\vx^T$ do not affect $b_i^*$ for $1\le i \le m$.  The values of $\vx_{T^c}$ are manually set, and $\vQ_{i, j} \vx_{j}$ is added to the corresponding $b_i^*$.  Thus, for $1 \leq i \le m$, we have $\vb^* = \vA^*\vx+\ve$.   For $m< i$, this constraint is also satisfied by the values of $\vx_T$ set in Equation~\ref{eq:x t solution}.  

Thus, it remains to show that step~\ref{step:randomization} preserves this solution.
%For convenience, denote by $\vx^*$ the vector where $x_i^* = x_i'$ for $1\leq i \leq n$ and $x_{n+i}^* = b_{m+i}^*$ otherwise.  
We now show that for all rows $1\leq i\leq m$, if $b_i^* = \va^*_i \vx+e_i$  then $b_i = \va_i \vx + e_i$.
%, we have added $\alpha$ equations as well as $\alpha$ unknowns, furthermore, those $\alpha$ equations have no error.  The number of unknowns we added are $x_{n+i} = b^*_{m+i} - \sum^{n}_{j=1}R_{ij}x_j$ for $i=n+1,..., n+\alpha$.
Recall the other rows are not modified.  We have the following for $1\leq i\leq m$:
\begin{align*}
\va_i \vx + e_i &= \left(\va_{i}^*+ \gamma_i(\vR || \vS)\right) \vx + e_i\\
&=\va_i^* \vx + e_i +\gamma_i(\vR||\vS) \vx\\&= b_i^* + \gamma_i (\vR||\vS)\vx
\end{align*}
Recall that $b_i =b_i^* + \gamma_i(b_{m+1}^*,..., b_{m+k}^*)$.  We consider the product $(\vR|| \vS) \vx$.  It suffices to show that $(\vR|| \vS) \vx = (b_{m+1}^*,..., b_{m+\alpha}^*)$,
\begin{align*}
(\vR|| \vS) \vx &= \vR \begin{pmatrix} \vx_1  \\ \vdots \\\vx_n \end{pmatrix}+  \vS_{T^c} \vx_{T^c}  + \vS_T \vx_T \\
&=\vR \begin{pmatrix} \vx_1  \\ \vdots \\\vx_n \end{pmatrix}+  \vS_{T^c} \vx_{T^c}   + \begin{pmatrix} b_{m+1}^*  \\ \vdots \\b_{m+\alpha}^* \end{pmatrix}  - \vR \begin{pmatrix} \vx_1  \\ \vdots \\\vx_n \end{pmatrix}- \vS_{T^c} \vx_{T^c}  
\\&=  \begin{pmatrix} b_{m+1}^*  \\ \vdots \\b_{m+\alpha}^* \end{pmatrix}
\end{align*}
This completes the proof of the claim.
\end{proof}
\begin{claim}\label{clm:random b}
If the conditional distribution $\vb'\,|\,\vA'$ is uniform, then $\vb\,|\,\vA$ is also uniform.
\end{claim}
\begin{proof}
Since $\vR, \vS$, and $\vQ$ are chosen independently of $\vb'$, the distribution $\vb'\,|\,\vA^*$ is uniform.
Let $\vb^*$ be the vector generated after step~\ref{step:fill in matrix}. Its first $m$ coordinates are  computed by adding the uniform vector $\vb'$ to values that are independent of $\vb^*$, and its remaining $\alpha$ coordinates $b^*_{m+1},\dots,b^*_{m+\alpha}$ are   chosen uniformly.  Thus $\vb^*\,|\,\vA^*$ is uniform. 

Let $\vgamma$ represent the matrix formed by $\gamma_{i}$.  It is independent of $\vb^*$ and $\vA^*$, so $\vb^*\,|\,(\vA^*, \vgamma)$ is uniform.    Let $\vgamma'=\left(\begin{array}{c | c}\vect{I_m} & \vgamma \\\hline \vect{0} & \vect{I_\alpha}\end{array}\right)$.
Note that $\vb=\vgamma' \vb^*$.  Since $\vb^*\,|\,(\vA^*, \vgamma)$ is uniform, and $\vgamma'$ is invertible, $\vb\,|\,(\vA^*, \vgamma)$ must also be uniform.
Since $\vA$ is a deterministic function of $\vA^*$ and $\vgamma$ (assuming Step~\ref{step:find columns} is deterministic---if not, we can fix the coins used), the distribution $\vb\,|\,\vA$ is the same  as $\vb\,|\,(\vA^*, \vgamma)$ and is thus also uniform.
\end{proof}

Finally, the reduction runs in polynomial time and together Claims~\ref{cl:randomMatrixDist},~\ref{cl:random ax+e}, and~\ref{clm:random b} show that when $\rank(\vS) = \alpha$ the distinguisher $D'$ properly prepares the instance thus, 
\begin{align*}
&\left|\Pr[D'(\vA, \vA X+\chi) = 1] - \Pr[D'(\vA, U) =1] \right|\\
&\, = \left| \Pr\left[D'(\vA', \vu') = 1 | \rank(\vS) = \alpha \right]- \Pr\left[D'(\vA', \vb'=\vA'\vx + \ve)=1 | \rank(\vS) = \alpha\right]\right| \Pr[\rank(\vS) = \alpha] \\
&\, =\left|\Pr[D(\vA, \vA X+\chi) = 1  | \rank(\vS) = \alpha] - \Pr[D(\vA, U) =1 | \rank(\vS) = \alpha]  \right| \Pr[\rank(\vS) = \alpha] \\
&\, \geq (\epsilon - \ngl(n))(1-\ngl(n)) \approx \epsilon
\end{align*}
Where the second line follows because we can detect when $\rank(\vS)<\alpha$ and output a random bit in this case.
Thus, Equation~(\ref{eq:block LWE dist}) is satisfied, this completes the proof.
\end{proof}
\section{Additional Proofs}

\subsection{Proof of \lemref{lem:averageToMaximalError}}
\label{sec:proof of average to maximal error}
\begin{proof}
Let $C$ be the  $(t,\epsilon)$-average error Shannon code with recovery procedure $\rec$ such that  $\Hoo(C)\geq k$.  Then for all $t'\le t$
\[
\sum_{c\in C} \Pr[C=c]\Pr[ c'\leftarrow \neigh (c, t') \wedge \rec(c') \neq c]\leq \epsilon.
\]
For $c$ denote by $\epsilon_c = \Pr[c'\leftarrow \neigh(c, t') \wedge \rec(c') \neq c]$.  
Then by Markov's inequality:
\[
\Pr_{c\in C}[ \epsilon_c \leq 2\expe_{c\leftarrow C} [\epsilon_c ] ] = \Pr_{c\in C} [\epsilon_c \le 2\epsilon ] \geq \frac{1}{2}
\]
Let $C'$ denote the of  set all $c\in C$ where $\epsilon_c\leq 2\epsilon$.  Note that $\Pr_{c\leftarrow C}[c\in C']\geq 1/2$.  Since $H_\infty(C)\geq k$, we know $|C'|\geq 2^{k-1}$~(otherwise $\Pr_{c\leftarrow C}[c\in C']=\sum_{c\in C'}\Pr[C=c]$ would be less than $2^{k-1}\frac{1}{2^k} = 1/2$).  This completes the proof of the~\lemref{lem:averageToMaximalError}.

%Suppose that for all $W'\subset W$ where $|W'|\geq 2^{k-1}$ there exists a $w\in W'$ such that $\Pr[ w'\leftarrow \sample (w,t') \wedge \rec(w') \neq w]> 2\epsilon$.  
%This implies that there exists a set $V$of size $V\geq |W|-2^{k-1}+1$ where $\forall v\in V, \epsilon_v >2\epsilon$.  Then note that $\Pr[W\in V]> 1/2$ as $W\setminus V$ contains at most $2^{k-1}-1$ points each of which has probability at most $1/2^k$ so $\Pr[W\in (W\setminus V)]\leq (2^{k-1}-1)/2^k < 1/2$.  Thus,
%\begin{align*}
%\Pr_{w\in W}[ w'\leftarrow \sample (w,t') \wedge \rec(w') \neq w]&\geq \sum_{w\in V}  \Pr[W=w]\Pr[ w'\leftarrow \sample (w,t') \wedge \rec(w') \neq w]\\
%&\geq \sum_{w\in V} \Pr[W=w] 2\epsilon   = 2\epsilon \Pr[W\in V]> 2\epsilon\left(\frac{1}{2}\right)= \epsilon
%\end{align*}
%% where $\forall d\in D, \Pr[d'\leftarrow \sample(d) \wedge \rec(d') > 2\epsilon$.  In turn this implies that  
%%\begin{align*}
%%\Pr_{c\in C}[ c'\leftarrow \sample (c) \wedge \rec(c') \neq c]&=\frac{1}{|C|} \sum_{c\in C}  \Pr[ c'\leftarrow \sample (c) \wedge \rec(c') \neq c]\\
%%&\geq \frac{1}{|C|} \sum_{d\in D}  \Pr[ d'\leftarrow \sample (d) \wedge \rec(d') \neq d]\\
%%&\geq \frac{1}{|C|}\sum_{d\in D} 2\epsilon = \frac{1}{|C|}\left(\frac{|C|}{2}+1\right)2\epsilon = \epsilon + 2\epsilon/|C|> \epsilon
%%\end{align*}
%This is a contradiction.  The statement of~\lemref{lem:averageToMaximalError} follows directly.
\end{proof}

\subsection{Proof of \thref{thm:impSketchArbitraryW}}
\label{sec:proof of thm sketch implies code}
\begin{proof}
  Let $W$ be an arbitrary distribution of min-entropy $m$.  Let $(X, Y)$ be a joint distribution such that $\Hav(X | Y)\geq k$ and
\[ 
\delta^{\mathcal{D}_{s_{sec}}}((W, \sketch(W)), (X, Y))\le \epsilon\, ,
\]  
where  $s_{sec} \geq t(s_{neigh}+s_{rec})$.  One such $(X, Y)$ must exist by the definition of conditional HILL entropy. 
%Let $X$ be arbitrarily but independently distributed over $\mathcal{M}$. Let $\delta^D((X, \sketch(X)), (W, \sketch(X)))<\epsilon$ for negligible $\epsilon$ and $D$ of size at least $s=O(s_{neigh}+s_{rec})$.  This implies that $H^{\hill}_{\epsilon, s}(X|\sketch(X))\geq k$.  
Define $D$ as:
\begin{enumerate}
\item Input $w\in\mathcal{M}, z \in\{0, 1\}^*, t$.
\item For all $1\leq t'\leq t$: 
\subitem  $w'\leftarrow \neigh(w, t')$.
\subitem If $\rec(w', z) \neq  w$ output $0$.
\item Output $1$.
\end{enumerate}
 By correctness of the sketch $ \Pr[D(W, \sketch(W)) =1]\ge 1-t\delta$.  Since 
$\delta^D((W, \sketch(W)), (X, Y))\le \epsilon$, we know $\Pr[D(X, Y) = 1]\ge 1-\epsilon-t\delta$.  Let $X_y$ denote the random variable $X|Y=y$.  By Markov's inequality,  there exists a set $S_Y$ such that $\Pr[Y\in S_Y]\ge 1/2$ and for all $y\in S_Y$, $\Pr[ D(X_y, y) =1]\ge 1- 2(\epsilon + t\delta)$.  

Because $\Hav(X | Y)\geq k$, we know that $\expe_{y\leftarrow Y} \max_x \Pr[X_y=x]\leq 2^k$.  Applying Markov's inequality to the random variable $\max_x \Pr[X_y=x]$, there exists a set $S'_Y$ such that $\Pr[y\in S'_Y]> 1/2$, and for all $y\in S'_Y$, $\Hoo(X_y)\ge k-1$ (we can use the strict version of Markov's inequality here, because the random variable $\max_x \Pr[X_y=x]$ is positive).  Fix one value $y \in S_Y\cap S'_Y$ (which exists because the sum of probabilities of $S_Y$ and $S'_Y$ is greater than 1).  
Thus, for all such that $t', 1\leq t'\leq t$, 
\[ \Pr_{x\leftarrow X_y}[x'\leftarrow \neigh(x, t') \wedge \rec(x',z) = x]\ge  1-2(\epsilon+t\delta).\]  
%Thus, $\rec(\cdot, \sketch(x'))$ is a decoding procedure that succeeds on a random neighbor of a random codeword with probability all but $\epsilon$.  
Thus,  $X_y$ is a $(t, 2(\epsilon+t\delta))$-average error Shannon code with recovery $\rec(\cdot,y)$ and $2^{k-1}$ points.  The statement of the theorem follows by application of \lemref{lem:averageToMaximalError}.  
\end{proof}

\subsection{Proof of \thref{thm:imp of unp entropy}}
\label{sec:proof of imp unp entropy}
Instead of proving the result just for Hamming metric over $\mathcal{Z}^n$, we will prove the result for any metric space that is both neighborhood samplable~(\defref{def:neighborhood samplable}) and where picking a random point in the space is easy.  We now define this second condition:
\begin{definition}
A metric space space $(\mathcal{M}, \dis)$ is $s_{sam}$-\emph{efficiently-samplable} if there exists a randomized circuit $\sample$ of size $s_{sam}$ that outputs a uniformly random point in $\mathcal{M}$.
\end{definition}

\begin{theorem}
Let $W$ be a distribution over a metric space $(\mathcal{M}, \dis)$ that is $s_{sam}$ samplable and $(s_{neigh}, t)$ neighborhood samplable.  Furthermore, assume that the number of points within distance $t$ in $\mathcal{M}$ is at least some fixed value $B_t(\cdot)$.  Let $(\sketch, \rec)$ be an unpredictability-entropy $(\mathcal{M}, \Hoo(W), \tilde{m}, t)$ secure sketch that is $(\epsilon, s_{sec})$-secure with error $\delta$.  If $s_{sec} \geq \max\{ t(|\rec| +s_{neigh}), |\rec| + s_{sam}\}$, then $\tilde{m}\leq \log |\mathcal{M}| - \log |B_t(\cdot)| + \log(1-\epsilon -t\delta)$.
\end{theorem}
\begin{proof}
Let $(X, Y)$ be two random variables such that $\delta^{\mathcal{D}_{s_{sec}}}((W, \sketch(W)), (X, Y))\leq \epsilon$.  It suffices to show that $\exists \mathcal{I}$ of size $s_{sec}$ such that $\Pr[\mathcal{I}(Y) = X]\geq |\mathcal{M}| (1-\epsilon -t\delta) / |B_t(\cdot)|$.  

Let $B_t(x)$ denote the random variable representing a random neighbor of distance at most $t$ from $x$ (note that $B_t$ may not be efficiently samplable, because we are assuming only that a neighbor a fixed distance is efficiently samplable).
We begin by showing that \rec must recover points of $X$.  
\begin{claim}
\label{clm:y is recoverable}
\begin{align*}
\Pr[\rec(B_t(X), Y) = X]&=\\
\Pr[(x, y)\leftarrow (X, Y) \wedge x'\leftarrow B_t(x) \wedge \rec(x', y) = x] &\geq 1-\epsilon -t\delta.
\end{align*}
\end{claim}
\begin{proof}
Suppose that $\Pr[\rec(B_t(X), Y) = X]<1-\epsilon -t\delta$.  We construct the following distinguisher $D\in\mathcal{D}_{s_{sec}}$ (the distinguisher design is slightly complicated by the fact that we don't know at which particular distance $t'$ the recover procedure is most likely to fail, so we have to try all distances):
\begin{itemize}
\item Input $w\in \mathcal{M}, s\in\zo^*$.
\item For all $1\leq t'\leq t$: 
\subitem  $w'\leftarrow \neigh(w, t')$.
\subitem If $\rec(w', z) \neq  w$ output $0$.
\item Output $1$.
\end{itemize}
First note that $|D| = t( |\rec|+ s_{neigh} )$.  Since $(\sketch, \rec)$ has error $\delta$ we know that $\forall w, w'\in \mathcal{M}$ where $\dis(w, w')\leq t$ \[ \Pr[s\leftarrow \sketch(w) \wedge \rec(w', s) =  w] \geq 1-\delta.\]  This implies that for all $1\leq t'\leq t$, $\Pr[\rec(\neigh(W, t'), \sketch(W) )= W)  ]\geq 1-\delta$ and thus $\Pr[D(W, \sketch(W)) = 1]\geq 1-t\delta$.  If $\Pr[\rec(B_t(X), Y) = X] < 1-\epsilon -t\delta$ there must exist at least one $1\leq t'\leq t$ for which $\Pr[\rec(\neigh(X, t'), Y) = X] < 1-\epsilon -t\delta$.  Then 
\begin{align*}
\Pr[D(W, \sketch(W)) = 1]  - \Pr[D(X, Y)=1] &\geq\\
(1-t\delta) - \Pr[\rec(\neigh(X, t'), Y) = X] &> (1-t\delta)-(1-t\delta - \epsilon)>\epsilon.
\end{align*}
This is a contradiction and the statement of the claim follows.
\end{proof}

Now define $\mathcal{I}$ as follows:
\begin{itemize}
\item Input $y\in\zo^*$.
\item Sample $x'\leftarrow \sample$.
\item Output $\rec(x', y)$.
\end{itemize}
Note that $|\mathcal{I}| =  |\rec|+ s_{sam}$. 
We now show that $\mathcal{I}$ predicts $X$:
\begin{align*}
\Pr[\mathcal{I}(Y) = X] & = \\
&= \sum_{x, y\in \mathcal{M}} \Pr[(x, y)\leftarrow (X, Y)] \Pr[\mathcal{I}(y) = x]\\
&= \sum_{x, y\in \mathcal{M}} \Pr[(x, y)\leftarrow (X, Y)] \sum_{x'\in\mathcal{M}} \Pr[\sample = x' ] \Pr[\rec(x', y) =x]\\
&\ge \sum_{x, y\in \mathcal{M}} \Pr[(x, y)\leftarrow (X, Y)] \sum_{x' \mathrm{ s. t. } \dis(x', x)\le t} \Pr[\sample = x']\Pr[\rec(x', y) =x]\\
&\ge \sum_{x, y\in \mathcal{M}} \Pr[(x, y)\leftarrow (X, Y)] \sum_{x'\mathrm{ s. t. } \dis(x', x)\le t} \frac{|B_t(\cdot)|}{|\mathcal{M}|} \Pr[B_t(x) = x'] \Pr[\rec(x', y) =x]\\
&\geq \frac{|B_t(\cdot)|}{|\mathcal{M}|}(1-\epsilon - t\delta)
\end{align*}
(the last step follows by  \clref{clm:y is recoverable}).

\thref{thm:imp of unp entropy} follows by noting that $\mathcal{Z}^n$ can be sampled in time $n\log |\mathcal{Z}|$ and neighborhood sampled in time $n\log |\mathcal{Z}|$.
\end{proof}

%\bnote{This is the proof for not relaxed unpredictability entropy}
%\begin{proof}
%Let $W$ be a distribution over $\zo^b$ and let $(\sketch, \rec)$ be an Unpredictability-entropy $(\zo^b, \Hoo(W), \tilde{m}, t)$-secure sketch that is $(\epsilon, s_{sec})$-secure with error $\delta$ where $s_{sec} = \Omega(|\rec|+b)$.  Let $Y_{W=w \wedge \sketch(w) = s} $ be a collection of distributions giving rise to a joint distribution $Y$ such that $\delta^{\mathcal{D}_{s_{sec}}}((W, \sketch(W)), (Y, \sketch(W)))\leq \epsilon$.  It suffices to show that $\exists \mathcal{I}$ of size $s_{sec}$ such that $\Pr[\mathcal{I}(\sketch(W)) = Y]\geq 2^{\tilde{m}}$ where $\tilde{m} = n - \log |B_t(\cdot)| + \log(1-\epsilon -\delta)$.  We begin by showing that $Y$ must correctly recover most of the time.
%\begin{claim}
%\label{clm:y is recoverable}
%\begin{align*}
%\Pr[\rec(B_t(Y), \sketch(W)) = Y]&=\\
%\Pr[y\leftarrow Y\wedge y'\leftarrow B_t(y) \wedge \rec(y', s) = y] &=\\ 
%\Pr[w\leftarrow W, s\leftarrow \sketch(w), \wedge y \leftarrow Y_{s} \wedge y' \leftarrow B_t(y) \wedge \rec(y', s) =y ] &\geq 1-\epsilon -\delta.
%\end{align*}
%\end{claim}
%\begin{proof}
%Suppose that $\Pr[\rec(B_t(Y), \sketch(W)) = Y]<1-\epsilon -\delta$.  We construct the following $D\in\mathcal{D}_{s_{sec}}$:
%\begin{itemize}
%\item Input $x\in \zo^b, s\in\zo^*$.
%\item Sample $x'\leftarrow B_t(x)$.
%\item Output $1$ if $\rec(x', s) = x$ otherwise output $0$.
%\end{itemize}
%Since $(\rec, \sketch)$ has error $\delta$ we know that $\forall w, w'\in\zo^b$ such that \[\dis(w, w')\leq t, \Pr[s\leftarrow \sketch(w) \wedge \rec(w', s) =  w] \geq 1-\delta.\]  This immediately implies that $\Pr[\rec(B_t(W), \sketch(W) = W)  ]\geq 1-\delta$.  The statement of the claim follows as 
%\begin{align*}
%\Pr[D(W, \sketch(W)) = 1]  - \Pr[D(Y, \sketch(W))] &=\\
%\Pr[\rec(B_t(W), \sketch(W) )= W] - \Pr[\rec(B_t(Y), \sketch(W)) = Y] &> (1-\delta)-(1-\delta - \epsilon)>\epsilon.
%\end{align*}
%\end{proof}
%We now consider the experiment $w\leftarrow W, s\leftarrow \sketch(w), y\leftarrow Y_{s}, y'\leftarrow B_t(y)$ and define the quantity $P_{y' , s, y} = \Pr[\rec(y', s) =y]$.  \clref{clm:y is recoverable} says that 
%\[
% \Pr[w\leftarrow W, s\leftarrow \sketch(w), y\leftarrow Y_s] \sum_{y'\in \zo^b} \Pr[B_t(y) =y' | Y = y] P_{y',  s, y} \geq 1-\delta -\epsilon.
%\]
%We note that for any particular $y, y'$ such that $\dis(y, y')\leq t, \Pr[B_t(y) = y' | Y=y] = 1/|B_t(\cdot)|$ and if $\dis(y, y')>t, Pr[B_t(y) = y' | Y=y] = 0$.  Thus, for all $y, y', \Pr[B_t(y) = y' | Y=y] \leq 1/|B_t(\cdot)|$
%We now define $\mathcal{I}\in\mathcal{D}_{s_{sec}}$ as follows:
%\begin{itemize}
%\item Input $s\in\zo^*$.
%\item Sample $x'\leftarrow \zo^b$.
%\item Output $\rec(x', s)$.
%\end{itemize}
%We define by $X$ the random variable corresponding to the sample $x'$ performed by $\decode_t$.
%We now show that $\mathcal{I}$ predicts $Y$:
%\begin{align*}
%2^{\tilde{m}}& = \Pr[\mathcal{I}(\sketch(W)) = Y] \\
%&= \Pr[w\leftarrow W, s\leftarrow \sketch(W), y\leftarrow Y_s] \Pr[\mathcal{I}(s) = y]\\
%&\geq \Pr[w\leftarrow W, s\leftarrow \sketch(W), y\leftarrow Y_s] \sum_{y'\in \zo^b} \Pr[X = y' ]\Pr[\mathcal{I}(s) = y | X = y']\\
%&\geq \Pr[w\leftarrow W, s\leftarrow \sketch(W), y\leftarrow Y_s] \sum_{y'\in \zo^b} \Pr[X = y' ] P_{y', s, y}\\
%&\geq \Pr[w\leftarrow W, s\leftarrow \sketch(W), y\leftarrow Y_s] \sum_{y'\in \zo^b} \frac{\Pr[X = y']}{\Pr[B_t(y) = y' | Y = y]}\Pr[B_t(y) = y' | Y=y ] P_{y', s, y}\\
%&\geq  \Pr[w\leftarrow W, s\leftarrow \sketch(W), y\leftarrow Y_s] \sum_{y'\in \zo^b} \frac{2^{-b}}{1/|B_t(\cdot)|} \Pr[B_t(y) = y' | Y=y ] P_{y', s, y}\\
%&\geq \frac{|B_t(\cdot)|}{2^b} \Pr[w\leftarrow W, s\leftarrow \sketch(W), y\leftarrow Y_s] \sum_{y'\in \zo^b} \Pr[B_t(y) = y' | Y=y ] P_{y', s, y}\\
%&\geq \frac{|B_t(\cdot)|}{2^b}(1-\epsilon - \delta)
%\end{align*}
%Finally converting to entropy, one has $\tilde{m}\leq b-\log |B_t(\cdot)| - \log (1-\epsilon - \delta)$.
%\end{proof}

%\subsection{Proof of \lemref{lem:i t constant time}}
%\label{sec:proof lem i t constant time}
%\begin{proof} We consider the expected number of iterations in step 2 when $t+1\leq m/n$.  Steps 3 and 4 can be accomplished in time $n^3\log q +s_{ver}$.  An iteration of $\decode_t$ succeeds when all selected rows of $\vA, \vect{C}$ have no error.  The probability of each selected row having an error is at most $\frac{t}{m - i}$ where $i$ is the number of rows already selected.  That is,
%\begin{align*}
%\Pr[i_1,..., i_n\text{ have no errors}]&\geq \prod_{i=0}^{n-1}\left(1 - \frac{t}{m-i}\right)\geq  \prod_{i=0}^{n-1}\left( 1-\frac{m-n}{n(m-i)}\right)\\
%&\geq \prod_{i=0}^{n-1}\left( 1-\frac{1}{n}\right)\geq \left(1-\frac{1}{n}\right)^n \geq 1/4.
%\end{align*}
%Note in the last step we assume that $n\geq 2$.  The minimum of $(1-1/n)^n$ is achieved at $n=2$ and it quickly converges to $1/e$.  Thus, the expected number of iterations is at most $4$ giving the stated running time for $\decode_t$.
%\end{proof}

\subsection{Proof of \lemref{lem:i t poly time}}
\label{sec:proof lem i t poly time}
\begin{proof}
Note that $\decode_t$ will stop if $w$ and $w'$ agree on all the rows selected in Step 2 (it may also stop for other reasons---namely, in step 4; but we do not use this fact to bound the expected running time).
The probability of each selected row having an error is at most $\frac{t}{m - i}$ where $i$ is the number of rows already selected.  That is,
\begin{align*}
\Pr[i_1,..., i_{2n}\text{ have no errors}]&\geq \prod_{i=0}^{2n-1}\left(1 - \frac{t}{m-i}\right)\geq \prod_{i=0}^{2n-1}\left( 1-\frac{d\left(\frac{m}{n}-2\right)\log n}{m-i}\right)\\
&\geq  \prod_{i=0}^{2n-1}\left( 1-\frac{d\log n}{n}\left(\frac{m-2n}{m-i}\right)\right)\geq \prod_{i=0}^{2n-1}\left( 1-\frac{d\log n}{n}\right) \\
&= \left(1-\frac{d\log n}{n}\right)^{2n}  = \left(\left(1-\frac{d\log n}{n}\right)^{\frac{n}{d\log n}}\right)^{2d\log n}\geq \frac{1}{4^{2d\log n}} = \frac{1}{n^{4d}}\,.
\end{align*}
(The second-to-last step holds as long as $n\ge 2d\log n$.) Because at each iteration, we select $2n$ rows independently at random, the expected number of iterations is at most $n^{4d}$; each iteration takes $O(n^3)$ operations in $\Fq$, which gives us the expected running time bound.

The probability that $\decode_t$ outputs $\perp$ is bounded by 
\begin{eqnarray*}
\Pr[\decode_t\rightarrow \perp]& \le & \sum_{j=1}^\infty \Pr[\decode_t\rightarrow \perp \text{ in $j$th iteration of step 4}]\\
&= &\sum_{j=1}^\infty  \Pr[\decode_t \text{ has not stopped after $j-1$ iterations} \wedge \rank(\vA_{i_1,\dots, i_{2n}})<n]\\
&\le &\sum_{j=1}^\infty  \Pr[i_1,..., i_{2n}\text{ had errors $j-1$ times} \wedge \rank(\vA_{i_1,\dots, i_{2n}})<n]\\
&= &\sum_{j=1}^\infty  \Pr[i_1,..., i_{2n}\text{ had errors $j-1$ times}]\cdot \Pr[\rank(\vA_{i_1,\dots, i_{2n}})<n]\\
&\le &\sum_{j=1}^\infty  \left(1-\frac{1}{n^{4d}}\right)^{j-1} \cdot q^{-n}\\
& = & n^{4d} e^{-\Omega(n)} = e^{-\Omega(n)}\,.
\end{eqnarray*}
The third line from the bottom follows from the fact that the locations of the errors are assumed to be independent of the sketch, and therefore independent of the matrix $\vA$.
The second line from the bottom follows from \clref{cl:full rank matrix} when $\beta = n$; note that, because we use the union bound and evaluate the probability separately for each value of $j$,  we can treat $\vA_{i_1,\dots, i_{2n}}$ as a randomly chosen $2n\times n$ matrix, ignoring the fact that these matrices are correlated.

We claim that if the code generated by $\vA$ 
has distance at least $2t+1$, then $\decode_t$ will output $\perp$ or the correct $\vx'=\vx$.
Indeed, suppose $\vx'\neq \vx$. Since $\vA (\vx-\vx')$ has at least $2t+1$ nonzero coordinates by the minimum distance of the code generated by $\vA$, and at most  $t$ of those can be zeroed out by the addition of  $w-w'$, such an $\vx'$ will not pass Step 6. 

The probability that the code generated by $\vA$ has distance lower than $2t+1$ is at most $e^{-\Omega(n)}$ (see \corref{cor:code high distance}), the probability of outputting $\perp$ is also $e^{-\Omega(n)}$~(computed above).  This gives the correctness bound for $\decode_t$.
\end{proof}

\section{Parameter Settings for \consref{cons:informal construction}}
\label{sec:parameter settings}
In this section, we explain the different parameters that go into our construction.  In \thref{thm:lossless secure extractor log} we give a lossless fuzzy extractor from a security parameter $n$ and an error $t$.  In this section, we discuss constraints imposed by 1) efficient decoding 2) maintaining security of the LWE instance and 3) ensuring no entropy loss of the construction.  We begin by reviewing the parameters that make up our construction:

\begin{itemize}
\item $|W|$: The length of the source.  
\item $t$: Number of errors that can be supported.  
\item $n$: LWE security parameter (i.e., number of field elements in $X$), which must be greater than some minimum value $n_0$ for security.
\item $q$: The size of the field.  
\item $\rho$: The fraction of the field needed for error sampling.  
\item $m$: The size of each number of samples in the LWE instance.  
\item $k$: The number of hardcore bits in $X$~(from \lemref{lem:many hardcore bits}).
\end{itemize}
We will split the source $|W|$ into $m$ blocks each of size $2\rho q+1$~(that is, $|W| = m\log (2\rho q+1)$).  We will ignore the parameter $|W|$ and focus on $t, n, q, \rho,$ and $m$.  As stated above we have three constraints:
\begin{itemize}
\item Maintain security of LWE.  If we assume GAPSVP and SIVP are hard to approximate within polynomial factors then \lemref{lem:uniform LWE decision} says that we get security for all $n$ greater than some minimum $n_0$ and $q = \poly(n)$ and $\rho q \geq 2 n^{1/2 + \sigma} m = \poly(n)$.  The only reason to increase $\rho q$ over this minimum amount (other than security) is if the number of errors in $W$ decreases with a slightly larger block size.  We ignore this effect and assume that $\rho q = 2n^{1/2+\sigma}m$.
\item Maintain efficient decoding of Construction~\ref{cons:decoding algorithm}.  Using \lemref{lem:i t poly time}, this means that $t\leq d\log n(m/n-2)$.
\item Minimize entropy loss of the construction.  We will output $X_{1,...,k}$ so the entropy loss of the construction is $|W|-|X_{1,..., k}|$.  We want the entropy loss to be zero, that is, $|W| = |X_{1,..., k}|$.  Substituting, one has $m\log 2\rho q+1 = k \log q$.
\end{itemize}
Collecting constraints we can support any setting where $t, n, q, \rho, m, k$ satisfy the following constraints~(for constants $d, f$):
\begin{align*}
n_0&< n -k \\
t&\leq d \log n\left(\frac{m}{n}-2\right)\\
q &= n^f\\
\rho q  &= 2n^{1/2+\sigma}m\\
m\log (2\rho q +1)&= k \log q
\end{align*}
Substituting $q = n^f$ and $\rho q = 2n^{1/2+\sigma}m$ yields the following system of equations:
\begin{align*}
n_0&< n - k\\
t&\leq d\log n\left(\frac{m}{n}-2\right)\\
m \log (4n^{1/2+\sigma}m +1)&= k \log n^f
\end{align*}
%\xnote{$ m \log 2n^{1/2+\sigma}m = n \log n^f$}
This is the most general form of our construction, we can support any $n, t, m$ that satisfy these equations for constants $d, f$.  However, the last equation may have no solution for $f$ constant.  Putting the last equation in terms of $f$ one has:
\begin{align*}
n_0&< n -k \\
t&\leq d\log n\left(\frac{ m }{n} -2\right)\\
%f \log n &= \frac{m}{n}\log 2n^2m\\
f &= \frac{m}{k}\frac{\log 4n^{1/2+\sigma} m+1}{\log n}
\end{align*}
To ensure $f$ is a constant, we set $t = c \log n$ for some constant $c$ and that $k = n /g$ for some constant $g> 1$.  Finally we assume that $m$ is the minimum value such that $t \leq d \log n(m/n-2)$~(that is, there are only as many dimensions as necessary for decoding using \lemref{lem:i t poly time}):
\begin{align*}
n_0&< n -k \\
m &= \frac{(c/d+2)n \log n}{\log n} = (\frac{c}{d}+2)n\\
f &= \frac{m}{k}\frac{\log 4n^{1/2+\sigma}m+1}{\log n} = \frac{g(c+2d)}{d}\frac{\log (\frac{4(c+2d)}{d} n^{3/2+\sigma}+1)}{\log n}
\end{align*}
%\xnote{$f = \frac{m}{n}\frac{\log 2n^{1/2+\sigma}m }{\log n} = \frac{c+d}{d}\frac{\log 2(c+d)/d n^{3/2+\sigma}}{\log n}$}
Note that $f$ is at a constant in $n$.
Assuming $n-k = n(1-1/g) > n_0$ and letting $t= c\log n$ we get the following setting:
\begin{align*}
m &= (\frac{c}{d}+2)n\\
q & = n^f = n^{\frac{m}{k}\frac{\log (4n^{1/2+\sigma}m+1)}{\log n}} = \poly(n)\\
\rho q &= 2n^{1/2+\sigma}m = 2(\frac{c}{d}+2)n^{3/2+\sigma}
\end{align*}

Note, that $f> \frac{m}{k}\geq \frac{m}{n} \geq \frac{(c/d+2)n}{n} \geq 3$ as long as $d<c$~(this also ensures that $m\geq 3n$, as required for \lemref{lem:i t poly time} to hold).  Since $\rho q = 2n^{1/2+\rho }m = O(n^{5/2})$ in our setting $\rho = O(n^{-1/2})$.  Thus, for large enough settings of parameters $\rho$ is less than $1/10$ as required by \lemref{lem:uniform LWE decision}.

Furthermore, we get decoding using $O(n^{4d+3})$ $\Fq$ operations.  We can output a $k$ fraction of $X$ and the bits will be pseudorandom~(conditioned on $\vA, \vA X+W$).  The parameter $g$ allows is a tradeoff between the number of dimensions needed for security and the size of the field $q$.  In \thref{thm:lossless secure extractor log}, we set $g=2$ and output the first half of $X$.  Setting $1<g<2$ achieves an increase in output length~(over the input length of $W$).   We also (arbitrarily) set $\sigma=1/2$ to simplify the statement of \thref{thm:lossless secure extractor log}, making $\rho q = 2(c/d+2) n^2$.

%The output corresponds to stretch~(in computational entropy) of $1/g$.  %, A linear stretch instead of $m\log (2\rho q +1)= k\log q$.  %Essentially, this corresponds to multiplying $g$ by the inverse of the desired stretch factor.

\subsection{Parameter Settings for \thref{thm:lossless block sketch log}}
\label{ssec:block params}
We repeat parameter settings for block fixing sources.  We now have $m+\alpha$ as the number of samples, while $n + \alpha+\omega(1)$ is the number of variables.  We can support any setting where $t, n, q, \rho, m, k, \alpha$ satisfy the following constraints~(for $\beta = \omega(1)$ and constants $d, f$):
\begin{align*}
n_0&< n -k  -\alpha -\beta\\
t&\leq d \log n\left(\frac{m}{n}-2\right)\\
q &= n^f\\
\rho q  &= 2n^{1/2+\sigma}m\\
m\log (2\rho q+1)&= k \log q
\end{align*}
Substituting $q = n^f$ and $\rho q = 2n^{1/2+\sigma}m$ yields the following system of equations:
\begin{align*}
n_0&< n - k - \alpha -\beta\\
t&\leq d\log n\left(\frac{m}{n}-2\right)\\
m \log (4n^{1/2+\sigma}m +1)&= k \log n^f
\end{align*}
As before we can support any setting any $n, t, m, \alpha$ that satisfy these equations for $\beta = \omega(1)$ and constants $d, f$.  However, the last equation may have no solution for $f$ constant.  Putting the last equation in terms of $f$ one has:
\begin{align*}
n_0&< n -k  - \alpha - \beta \\
t&\leq d\log n\left(\frac{ m }{n} -2\right)\\
%f \log n &= \frac{m}{n}\log 2n^2m\\
f &= \frac{m}{k}\frac{\log (4n^{1/2+\sigma} m+1)}{\log n}
\end{align*}
To ensure $f$ is a constant, we set $t = c \log n$ for some constant $c$ and that $k, \alpha = n/3$ and $\beta = \log n$.  Finally we assume that $m$ is the minimum value such that $t \leq  d \log n(m/n-2)$~(that is, there are only as many dimensions as necessary for decoding using \lemref{lem:i t poly time}):
\begin{align*}
n_0&< n/3 -  \log n\\
m &= \frac{(c/d+2)n \log n}{ \log n} = (\frac{c}{d}+2)n\\
f &= \frac{m}{k}\frac{\log (4n^{1/2+\sigma}m+1)}{\log n} = \left(3(\frac{c}{d}+2)\right)\frac{\log (4(\frac{c}{d}+2) n^{3/2+\sigma}+1)}{\log n} = O(1)
\end{align*}

Assuming $n/3-\log(n)> n_0$ and letting $t= c\log n$ we get the following setting:
\begin{align*}
m &= (\frac{c}{d}+2)n\\
q & = n^f = n^{\frac{m}{n}\frac{\log (4n^{1/2+\sigma}m+1)}{\log n}} = \poly(n)\\
\rho q &= 2n^{1/2+\sigma}m = 2(\frac{c}{d}+2)n^{3/2+\sigma}
\end{align*}

As before we arbitrarily set $\sigma = 1/2$, giving $\rho q = 2(\frac{c}{d}+2)n^2$.  Also, if $c<d$ then we get efficient decoding and $\rho = o(1)$ satisfying the condition of \lemref{lem:uniform LWE decision}.

\ignore{
\subsection{Old parameter setting}
The question then becomes whether there is a setting 
Our construction has two goals: 1) maximizing correcting capability~(while retaining polynomial time decoding, and 2) maximizing the unpredictability entropy of $W$ conditioned on the construction~($\tilde{m}$ in \defref{def:comp secure sketch}).  Our parameters will differ significantly depending on what setting of $t$ is used.  We will consider both settings where $t\leq m/n-2$ and $t\leq c\log n (m/n-2)$.  Recall these corresponding to Construction~\ref{cons:decoding algorithm} running either in fixed polynomial time or time approximately $n^{2c}$.  We consider each of these settings:

$\mathbf{t \leq m/n-1}$

We first find the appropriate range of $n$.  First, we need $n>n_0$ for LWE security. Increasing $n$, up to the point when $H_\infty(X)=H_\infty(W)$, i.e., $n\log q = m\log\rho q$, increases the unpredictability entropy of the construction~(see \assref{assume:entropy LWE}). However, this decreases the number of errors $t$ we can correct. There is no advantage to increasing $n$ further\footnote{We primarily consider the construction with no entropy drop, if a drop is unpredictability entropy is acceptable, more errors can be corrected.  However, if the remaining unpredictability entropy is worse than a known information theoretic construction~(see~\cite{DBLP:journals/siamcomp/DodisORS08}), an information theoretic construction should be used.
}.

We first analyze the lossless construction, when  $H_\infty(X)=H_\infty(W)$ and thus $n\log q = m\log\rho q$.
Substituting $m \ge (t+1)n$ above one has:
\[
n \log q  = m \log \rho q \geq (t+1)n \log \rho q
\]
Thus, the lossless construction can support settings where $t+1\leq \frac{\log q}{\log \rho q}$.  In order to apply \lemref{lem:uniform LWE}, we need that $\rho q\geq 2n^2m \geq 2(t+1)n^3$.  We consider the setting $q = n^d$ for some constant $d$ (larger $q$ is unlikely to improve parameters unless 
%In Appendix~\ref{sec:parameters q expo}, we review parameters where $q=2^n$, these parameters are worse unless 
lattice problems are hard to approximate within exponential factors.)  
Dividing the above equations above we have:
\[
t+1\leq \frac{\log q}{\log \rho q}\leq \frac{\log q}{\log 2(t+1) n^3}\leq \frac{\log n^d}{\log 2t n^3} \approx d/3
\]
(the last step follows because the second-to-last step already implies $t\le d\log n$).
Thus, we can support a constant $d/3-1$ number of errors where $q = n^d$ is the size of our field.  Because we need $\rho q\geq 2n^2m \geq 2(t+1)n^3$ (to apply \lemref{lem:uniform LWE}), we require $\rho > n^{3-d}$.  Thus, our construction is secure if SIVP and GAPSVP are hard within approximation factors of 
\[
\tilde{O}(n^{5/2}m/\rho) = \tilde{O}\left(\frac{n^{5/2} m }{\rho}\right)
= \tilde{O}\left(\frac{n^{5/2}t}{n^{3-d}}\right)
= \tilde{O}\left(n^{d-1/2}m\right)
\]

%As an example if $\gamma = n^{c}$, this yields:
%\[
%t\leq \frac{n}{n + \log n + \log t - \log \gamma} = \frac{n}{n+\log n +\log t  -\log n} = \frac{n}{n + \log t}
%\]
%Alternatively, if $\gamma = n^{\log n}$, this yields:
%\[
%t\leq \frac{n}{n + \log n + \log t - \log \gamma} = \frac{n}{n+\log n +\log t  -\log^2 n} = \frac{n}{n+\log n - \log^2 n + \log t}
%\]
%Thus, as $q$ grows we can support a larger number of errors.  However, the growth of $q$ with a fixed size $\rho q$ leads to the approximation factor in \lemref{lem:uniform LWE} no longer being a harder problem.  Thus, $q$ can only be increased while the lattice problems in \lemref{lem:uniform LWE} is still believed to be hard.  Furthermore, in the decision formulation of \lemref{lem:uniform LWE decision} requires that $q(n) = poly(n)$.  This means that $t\leq O(\log n)$.

$\mathbf{t\leq c\log n(m/n -1)}$

We repeat the above analysis.  Recall we now consider the case where $m\geq (t+1)n/(c\log n)$.  We seek to satisfy:
\[
n\log q = m\log \rho q \geq (t+1)n \log \rho q /(c\log n)
\]
Again, we assume that $q = n^d$ for some constant $d$.  Then 
\[
t+1\leq \frac{c \log n \log q }{\log \rho q}\leq \frac{cd\log^2 n}{ \log \frac{2(t+1) n^3}{c\log n}}\leq \frac{cd \log^2 n}{3 \log n} \approx \frac{cd \log n}{3}.
\]
This gives $\rho > n^{3-d}/\log n$ and security if SVIP and GAPSVP are hard within approximation factors of $\tilde{O}(n^{d-1/2}m/\log n)$.

\textbf{Formulation from $W, t$:}

In a standard application, a source $W$ and number of errors $t$ will be given\footnote{Parameter settings will change the alphabet size and thus might change $t$, we ignore these effects and assume $t$ is given.}.  Thus, the goal is to maximize security subject to correcting $t$ errors.  This means the following parameters must be set:
\begin{itemize}
\item How large should $X$ be?  The size of $X$ is the limiting factor for the resulting unpredictability entropy.  Thus, $n$ should be as large as possible while allowing decoding.  We must also ensure that $n>n_0$ for security.
\item How large a field should we operate over?  This is the parameter $q$.  Increasing this parameter allows correcting more errors~(subject to the underlying lattice problems still being hard to approximate).
\item How length block should $W$ be split into?  This is the parameter $\rho q$.  Setting this parameter also sets $m = |W|/\log \rho q$.
\end{itemize}

Unfortunately, the parameters $m, n$ and depend on each other and cannot be set independently.  As before, we will consider the fixed poly time decoder and variable poly time decoder in turn.

$\mathbf{t\leq (m/n-1)}$

We set parameters in the following order:
\begin{itemize}
\item Let $m, n$ be integer solutions to the following equations that maximizes $n$~(if the maximum solution is less than $n_0$, we cannot support a secure construction):
\begin{align*}
|W| &= m\log 2n^2m\\
n&\leq m/(t+1)
\end{align*}
We assume that the solution for $m,n$ exists and is tight~(in the sense that $n = m/(t+1)$).  If not, there is an entropy loss in the construction due to size mismatches~(in the entropy analysis below) and padding~(to ensure that $|W|$ is an integer number of blocks).
\item Set $\rho q = 2n^2m $.
\item Set $q = n^{(t+1)\log( 2(t+1)n)/\log n}$.  Note that although $n$ appears in the exponent if $t$ is a fixed constant~(does not depend on $|W|$), this value is $q =\poly(n)$.  
\end{itemize}  Given these parameters, we can calculate the sizes of $X$ and $W$~(as we show in the \secref{sec:security of LWE cons} $\tilde{m} =|W|$):
\begin{align*}
H^{\unp}_{\epsilon, s} ( W | AX+W) &= \min\{ H_\infty(W), H_\infty(E)\}\\
H_\infty(X) = |X| &= n \log q= \frac{m}{t+1}\log n^{(t+1)\log 2(t+1)n/\log n} \\
&= \frac{m}{t+1} \frac{(t+1)\log 2(t+1)n}{\log n} \log n = m \log 2(t+1)n^3 \approx m\log 2n^2m\\
H_\infty(W) = |W| & = m \log \rho q = m \log 2n^2 m\,.
\end{align*}

$\mathbf{t \leq c\log n(m/n-1)}$
We set parameters in the following order:
\begin{itemize}
\item Let $m, n$ be integer solutions to the following equations that maximizes $n$~(if the maximum solution is less than $n_0$, we cannot support a secure construction):
\begin{align*}
|W| & = m\log 2n^2m\\
 \frac{nt}{c\log n} + n&\leq m
\end{align*}
We assume that the solution for $m,n$ exists and is tight~(in the sense that $ \frac{nt}{c\log n} + n=m$).  If not, there is an entropy loss in the construction due to size mismatches~(in the entropy analysis below) and padding~(to ensure that $|W|$ is an integer number of blocks).
\item Set $\rho q = 2n^2m $.
\item Set $q = (2n^2m)^{m/n}$.  Note if $t = O(\log n)$, then $m/n = \frac{t}{c\log n} +1 = O(1)$ and thus $q$ is polynomial in $n$.
\end{itemize}  Given these parameters, we can calculate the sizes of $X$ and $W$~(as we show in the \secref{sec:security of LWE cons} $\tilde{m} =|W|$):
\begin{align*}
H^{\unp}_{\epsilon, s} ( W | \vA X+W) &= \min\{ H_\infty(W), H_\infty(E)\}\\
H_\infty(X) = |X| &= n \log q= n \log (2n^2m)^{m/n} = m \log 2n^2m\\
H_\infty(W) = |W| & = m \log \rho q = m \log 2n^2 m\,.
\end{align*}

We'll now consider a formulation for a fixed length biometric and a number of block errors.  We assume that $n$ is given as a security parameter and thus we ask how many errors can be corrected for a noisy uniform distribution of length $w>>n$.  We denote the number of errors by $t$~(we'll be able to correct block errors over the alphabet $[-\rho q, \rho q]$).  
Let $n$ be a security parameter and let $W$ be the uniform distribution of length $w>>n$.  Let $m$ as the minimum integer such that $m\log 4m^2 n\geq w$.  Set $q = 2^{m/n}4m^2n$.  Then split $W=( W_1,..., W_m)$ where each $W_i$ is of length $\log 4m^2 n$ bits.  Then, for $W' =( W_1',.., W_m')$ if  $t = |\{i |  W_i \neq W_i'\}| \leq m/n$ block errors, Construction~\ref{cons:decoding algorithm} allows for decoding in expected polynomial time.  Furthermore, $H^{\unp}_{\epsilon, s}(W | \vA X+W)\geq |W|$~(stating \lemref{lem:uniform LWE decision} in the language of Assumption~\ref{assume:entropy LWE}).  

\begin{theorem}
\label{thm:security of secure sketch}
Let $n_0$ be a security parameter and let $W$ be uniform over $\mathcal{M}$ and let $t$ be a constant.  Let $m,n $ be the solution to the following equations that maximizes $n$.
\begin{align*}
|W| &= m\log 2n^2m\\
n&\leq m/t
\end{align*}
If $n>n_0$ and $n = m/t$ and \assref{assume:entropy LWE} holds, then for 
\begin{align*}
q &= n^{3t\log( \sqrt[3]{2t}n)/\log n}\\
\rho q &= 2n^2m,
\end{align*}  Construction~\ref{cons:LWESecureSketch} with uniform error over the interval $[-2n^2m, 2n^2m]$ is a $(\mathcal{M}, |W|, |W|, \epsilon, s, s_{sketch}, s_{rec}, t)$-computational secure sketch for $s = \poly(n), s_{sketch} = O(m\times n )$ and $s_{rec}$ is expected polynomial time.
\end{theorem}

\bnote{Not sure what to do with this}
\begin{theorem}
\label{thm:security of block sketch}
Let $n_0$ be a security parameter and let $t$ be a constant.  Let $W\in \{0, 1\}^{(m+\alpha)\times \log 2n^2m}$ be an $\alpha$-symbol fixing source where $m,n $ are the solution to the following equations that maximizes $n$.
\begin{align*}
|W| &= (m+\alpha)\log 2n^2m\\
(n+\alpha)&\leq (m+\alpha)/t
\end{align*}
If $n>n_0$ and $n+\alpha = (m+\alpha)/t$ and \assref{assume:entropy LWE} holds, then for 
\begin{align*}
q &= n^{3t\log( \sqrt[3]{2t}n)/\log n}\\
\rho q &= 2n^2m,
\end{align*}  Construction~\ref{cons:LWESecureSketch} with uniform error over the interval $[-2n^2m, 2n^2m]$ is a $(\mathcal{M}, \Hoo(W), \Hoo(W), \epsilon, s, s_{sketch}, s_{rec}, t)$-computational secure sketch for $s= \poly(n)$,  $s_{sketch} = O(m\times n )$ and $s_{rec}$ is expected polynomial time under .
\end{theorem}

}

\ignore{
\section{Fuzzy Extractor Statement of Construction~\ref{cons:LWESecureSketch}}
\label{sec:fuzzy extractor phrasing}
In this section we restate Construction~\ref{cons:LWESecureSketch} as a computational fuzzy extractor instead of a computational secure sketch:
\begin{construction}[Computational Fuzzy Extractor based on LWE] 
\label{cons:LWEFuzzyExtractor} Let $n$ be a security parameter and let $m = m(n) = \poly(n), q = q(n)\geq 2$ be integers, furthermore let $\chi$ be a distribution $\Fq$ that can be sampled with a fixed number of bits $s_{err}$.
Let $\decode_t$ be an algorithm~(not necessarily efficient) that inverts an LWE instance when no more than $t$ of $m$ dimensions have non-zero error.  Furthermore, let $\rext$ be a reconstructive extractor.  Let $W$ be a distribution over $\{0,1\}^{s_{err}\times m}$.

\textbf{\gen}
\begin{enumerate}
\item Input $w\leftarrow W$.
\item Sample $A\in\Fq^{m\times n}, x\in\Fq^n$ uniformly at random.
\item Use $w$ as the randomness for the sampling algorithm, $\sample$, for $\chi$.  Set $E\leftarrow  \sample(w)$.
\item Sample $seed\leftarrow U$ as required for \rext.  Compute $r\leftarrow \rext(E, seed)$
\item Set $p = (A, AX+E, seed)$.
\item Output $(r, p)$.
\end{enumerate}

\textbf{\rep}
\begin{enumerate}
\item Input $(w', p)$
\item Parse $p$ as $(A, C, seed)$
\item Compute $E' \leftarrow \sample (w')$.
\item Set $X' = \decode_t(A, C-E') $. 
\item Compute $r\leftarrow \rext (C-AX', seed)$.
\item Output $r$.
\end{enumerate}
\end{construction}

\begin{theorem}[Security of Construction~\ref{cons:LWEFuzzyExtractor}]\label{thm:LWEFuzzyExtractor}
Fix $(n, m, q, \chi)$ such that Assumption~\ref{assume:general LWE} holds for $\mathcal{I}$ of size at most $s$ with success $\epsilon$.  Furthermore assume that a fixed number of bits $s_{err}$ are necessary to sample from $\chi$.  Let $\rext: \chi^m \times\{0,1\}^\ell\rightarrow \{0,1\}^{k_{len}}\times \{0,1\}^\ell$ be an extractor with $(\log 1/\epsilon - \log 1/\delta, \delta)$-reconstruction $(\cons,\decons)$.  Let $\decode_t$ be an inverter as described in Construction~\ref{cons:LWEFuzzyExtractor} of size $s_{\decode_t}$.  Then Construction~\ref{cons:LWEFuzzyExtractor} is a $(\{0,1\}^{m\times s_{err}}, m\times s_{err}, k_{len}, t, s_{rec}, s/(|\cons|+|\decons|), 5\delta)$ computational fuzzy extractor where $s_{rec, t} = s_{\decode_t}+ |\sample| + O(m\times n\times \log q) + |\rext|$.
\end{theorem}
}

\ignore{
\section{Min-entropy error LWE}
Throughout this section we will assume that \sample uses $\ell$ bits of randomness to produce the required distribution in a single dimension.  Usually, this is the normal (Gaussian) distribution with some mean.  Note that for the uniform distribution sampling requires $\log q$ bits where $q$ is the size of the range.  By information theory we know that in expectation, the number of random bits is smaller for an distribution that is not uniform, but this provides us no guarantee about the worst case number of bits.  We will begin with noting a few cases that seem significantly easier to show.  Recall that the distribution $W$ is drawn from is specified and known by the adversary.
\begin{itemize}
\item Case 1: $\Hoo(W)\leq \ell (m- n)$.  This case is not secure.  Let $W\in\{0,1\}^{\ell m}$ and we further specify $W=W_1,...,W_m$ where each $W_i\in\{0,1\}^\ell$.  We then set $W_1,...,W_n$ equal to some fixed value, $0$ without loss of generality, and $W_{n+1},...,W_m$ to the uniform distribution.  Then it is clear for the first $n$ equations, the adversary is solving a system $Ax=b$ without any error and this can be done in $P$.
\item Case 2: Full or no entropy.  As before, let $W\in\{0,1\}^{\ell m}$ and we further specify $W=W_1,...,W_m$ where each $W_i\in\{0,1\}^\ell$.  We further restrict to the case where each $W_i$ either $W_i\overset{d}=U_\ell$ or $\Hoo(W_i) = 0$.  Thus, for each dimension either, one of the two occurs: the error is properly generated or the adversary knows the error exactly. So, for each dimension, given the $W_i\overset{d}=U_\ell$, the adversary can only guess the correct random variable not better than $2^{\ell}$. Similarly, if given the $\Hoo(W_i) = 0$, the adversary can always guess the correct variable.(\xnote{This is trivial, since if we give $\Hoo(W_i) = 0$, we are actually giving the a random variable has probably of 1.})

We are considering the case that the adversary will be given the ensemble $\{W_1,...,W_n\}$, where each $W_i$ has some entropy.

\item Case 3: General case.
\item \textbf{Hypothetical worry case}  Suppose for convenience we are working with the normal distribution that has variance $\frac{2}{\pi}$ (centered around 0).  That is 
\[
\Pr[X=x] = \frac{1}{\sqrt{\frac{2}{\pi}}\sqrt{2\pi}}e^{-\frac{1}{2}{\frac{2}{\pi}x^2}}=\frac{1}{2}e^{-\pi x^2}
\]
Thus, $\Pr[X=0] =1/2$.  One can easily design a procedure $\sample'$ that generates this distribution using a maximum of some number $\ell$ bits.  Let $w_1,...,w_\ell$ be the bits that $\sample'$ takes as input.  Then when $w_1=0$ the sampler outputs $0$.  The generation for other values does not matter (only that generating some value takes $\ell$ bits).  Thus, to generate $n$ samples we need at most $\ell n$ bits.  However, there is a distribution with min-entropy $\Hoo(W)=(\ell -1) m$, that produces always produces the all zero error.  This distribution has the first bit of each $W_i$ 0 and the remaining bits truly random.  Thus, it seems impossible for our scheme to be secure for an arbitrary procedure \sample, but we will need to specify a specific \sample.  Another possibility is to use a different distribution that looks like the normai distribution but requires a ``flatter'' number of coins to be used for each distribution.  Is there a definition of an algorithm that uses the same number of random bits for each invocation?  Would this be enough to get rid of this problem?
\end{itemize}
Questions: 
\begin{enumerate}
\item How we sample this error vector $\vect{e}$?  Can we do something other than Gaussian to remove some worries?
\item Given the two biometric data which have small distance between each other, will we get the small distance between the error vector after sampling?
\textbf{Solved, each dimension is sampled independently.  Thus, our distance blowup is only as large as the number of bits needed to sample each dimension.}
\item When is LWE easy and when is LWE hard?  Are we going to be able to create a sufficient gap between these two cases?
\item Does the parallel sampler of Peikert allow for the distance in two randomness being small, creating small output distance?
\textbf{Solved, was trying to do the wrong thing.  Just Gaussian in each dimension.}
\item Does the sampler of Peikert allow for nonuniform input distribution?
\textbf{Solved, was trying to do the wrong thing (was trying to sample from lattice points, instead of adding noise around a lattice point).}
\subitem If not how does applying an extractor change the distance of two distributions?
\item How if it all can this be used concurrently?  Seems very one time right now.
\item How efficient will our decoding procedure be?  Efficient enough for practical usage?
\end{enumerate}

\textbf{What security do we need: }
}

\ignore{
\section{Robustness of LWE to error with min-entropy}
\label{sec: lwe min-entropy}

In this section we explore the robustness of the LWE assumption when the errors are drawn from a deficient distribution.  We provide a short introduction to the Learning with Errors problem in \secref{sec:fuzzyCompLWE}.  For a more complete introduction see the survey by Regev~\cite{regevLWEsurvey}.  The tools presented in this section are used to show the security of the computational secure sketch presented \subsecref{subsec:fuzzyExtLWE}.  This work is also of independent interest and can be viewed in a similar way to the work of Goldwasser et. al.~\cite{goldwasserRobustLWE}.  In their work, they consider the robustness of an LWE instance $A, Ax+e$ where $x$ is sampled from a distribution that has min-entropy but is not uniform.  They then use this result to show LWE-based cryptosystems are leakage-resilient.  Our theorem statement will follow a similar structure to the theorem statement of Goldwasser et. al. so we first present their informal theorem:
\begin{theorem}~(\cite[Theorem 1 (Informal)]{goldwasserRobustLWE})
For any super-polynomial modulus $q=q(n)$, any $k\geq \log q$, and any distribution $\mathcal{D}=\{D\}_{n\in\mathbb{N}}$ over $\{0,1\}^n$ with min-entropy $k$ , the (non-standard) LWE assumption, where the secret is drawn from distribution $\mathcal{D},$ follows from the (standard) LWE assumption with secret size $\ell\overset{\Delta}=\frac{k-\omega(\log n)}{\log q}$ (where the ``error rate'' is super-polynomially small and the adversaries run in time $\poly(n)$).
\end{theorem}

Before considering high entropy error, we must be clear about what this means.  The $x$ in an LWE instance is sampled uniformly from $\{0,q\}^n$, thus the sampling procedure requires $\lceil \log q\rceil\times n$ bits.  However, $e$ is sampled from some error distribution $\chi$.  $\chi$ is normally the normal distribution centered around $0$ with some variance $\sigma^2$ and then rounded to the nearest integer between $[-q/2, q/2]$.  Sampling from a rounded normal distribution requires a variable number of random bits for each dimension.    Indeed, using the normal distribution there are events with nonzero but negligible probability.  These events necessarily take a large number of bits to sample.  

\textbf{This is internal discussion and needs to be taken care of before presentation:}  There are two ways to provide these bits: 1) break the random string into chunks where each chunk is as long as the maximum number of random bits needed 2) provide the random bits needed to each dimensions.  One might hope that this problem is avoided by rounding to the nearest integer, but there is still some weight assigned to integers far from $0$.  We could consider a distribution whose statistical distance to the normal distance is exponentially small and gives weight 0 to all integers outside some radius.

For the moment we will ignore these problems and assume that a fixed number of bits is used to sample each dimension.  This would be the case if we substituted the uniform distribution on a significantly smaller domain.  \bnote{We need to be sure this produces a secure LWE instance}.

Thus, we assume that there exists a $\sample$ that uses $\ell$ bits of randomness to produce an error term in one dimension.  We are now ready to present the LWE assumption:
\begin{assumption}[Learning with Errors]
Let $k$ be a security parameter and define $q=\poly(k)$\bnote{Do we want the poly or exponential case?}and $n = \poly( k)$.
Let $x\overset{\$}\leftarrow \{0,q\}^n$.   An LWE sample is of the form $(a,b) = (a, a_1x_1+...+a_nx_n+\sample(e) \mod q)$ where $a_1,...,a_n\overset{\$}\leftarrow\{0,q\}^n$ and $e\overset{\$}\leftarrow\{0,1\}^\ell$.  The $(m,n,q,k)$-LWE assumption is that no algorithm given $m = \poly(k)$ LWE samples running in time $\poly(k)$ can recover $x$ with probability greater than negligible in $k$.
\end{assumption}

We then achieve our non-standard LWE assumption by sampling $e$ from a high-entropy distribution instead of the uniform distribution.

\subsection{Affine min-entropy distributions}
In the previous section, we investigated the security of the LWE assumption with deficient distributions.  Namely, we saw that the LWE assumption for block-fixing sources was implied by the LWE assumption with fewer dimensions and samples.  We now ask: what other types of min-entropy distributions can be shown to reduce to the LWE assumption?  Using a reduction to standard LWE seems to require that the errors be considered homomorphically.  This is because recovering the error terms would break LWE.  Furthermore, these errors are first put through the \sample procedure, so this needs homomorphic properties as well.  Thus, the class we can hope for in this reduction is a linear set of errors.

Recall, a function, $f:D^n \mapsto R$, over field $K$ is a \textbf{linear transformation} if $\forall x,y\in D$ and $\forall a, b\in K, f(ax+by) = af(x) + bf(y)$.

We will consider the \textbf{LWE assumption w/ linear sources}.  That is, we consider an error distribution $E = E_1||...||E_k$ where $\Hoo(E) = qn$ and $E = \mathbf{T}(U_1||...||U_n)$ for rank $n$ matrix $\mathbf{T}: \Fq^n\mapsto \Fq^k$.
\vspace{.1in}
\begin{theorem}\label{thm:linearLWE}
Let $k$ be a security parameter and define $q, n, m = \poly(k)$.  Let $E$ be a linear source over $\{0,1\}^{\ell(m+\alpha)}$ where $\Hoo(E) \geq m\ell$.  Then the $(m,n,q,k)$-LWE assumption where \sample is a linear transformation implies the $(m+\alpha, n+\alpha, q, k)$-LWE w/ linear sources assumption.
\end{theorem}
\begin{proof}
The main difference between \thref{thm:blockLWE} and \thref{thm:linearLWE} is how we produce the errors terms.  We will use the same basic strategy of introducing equations and randomizing, however, the equations will errors will now be involved in the randomization.  As before, let $\mathcal{A}$ be an algorithm that accepts $m+\alpha$ LWE samples each of length $n+\alpha$ where the errors are generated $(E_1,..., E_{m+\alpha})^T = \mathbf{T} (U_1,..., U_m)^T$ for some matrix $\mathbf{T}$.  Further, assume that $\mathcal{A}$ returns $x$ with noticeable probability.  Our algorithm $\mathcal{A'}$ will make a single call to $\mathcal{A}$ and we will focus on properly preparing the LWE w/ linear sources instance.  We use $A',A,C,D,F,G,b,b'$ with the same meaning as before.
\end{proof}
}



\ignore{
\subsection{LWE search reduction}
\begin{proof}
Let $\mathcal{A}$ be an algorithm that accepts $m+\alpha$ LWE samples each of length $n+\alpha$ where $\alpha$ of the equations have a known error (and the remainder of the errors are uniformly generated) and returns $x$ with noticeable probability.  Denote this probability as $\epsilon$.  We will show how to convert $\mathcal{A}$ into an algorithm $\mathcal{A'}$ that accepts $m$ LWE samples of length $n$ and returns $x'$ with noticeable probability.  We begin by making the following assumptions for clarity:
\begin{itemize}
\item $\mathcal{A}$ asks for all $m+\alpha$ samples and asks for them simultaneously.  Since $\mathcal{A}$ has no effect on the samples, this does not limit $\mathcal{A}$'s power.
\item For the known error samples, the added error is $0$.  It is clear in our reduction where the added error would be added, but exposition is clearer without these errors.
\item The last $\alpha$ samples are the known error samples.  For other positioning of the known error samples, we can provide a permutation from our input, but this form is clearer.
\end{itemize}
The algorithm $\mathcal{A'}$ will make a single call to the algorithm $\mathcal{A}$ and thus the entire analysis will be showing that $\mathcal{A'}$ is able to properly prepare the instance that $\mathcal{A}$ is expecting and convert the result of $\mathcal{A}$ into a solution for the input of $\mathcal{A'}$.  We will use the matrix notation for the remainder of the proof.  Thus, the job of $\mathcal{A'}$ is to perform the following transformation:
\begin{align*}
\begin{array}{c | c}
\mathcal{A'}\text{ receives:        } & \mathcal{A}\text{ expects: }\\\hline
\left(A'\right) , b' = A' x' + e' & A = \left( \begin{array}{c | c}C & D \\ \hline F & G\end{array} \right), b=Ax+e
\end{array}
\end{align*}
Where the matrices have the dimensions as shown in \figref{fig:matrixDimension}: 
\begin{figure}[b]
\[
\begin{array}{c | c | c | c | c | c}
A & A' & C & D & F & G\\\hline
m\times n & (m+\alpha)\times( n+\alpha) & m\times n & m \times \alpha & \alpha \times n & \alpha \times \alpha
\end{array}
\]
\caption{Dimensions of matrices in reduction.}
\label{fig:matrixDimension}
\end{figure}
Thus, we have two major jobs, produce an augmented matrix $A$ that is a truly random matrix and to produce an augmented $b$ that given a solution for $x$ will yield a solution for $x'$.  The code of $\mathcal{A'}$ is presented in \figref{fig:imperfectLWEreduction}.  The main idea is to generate $\alpha$ new LWE samples and use $\alpha$ new variables to exactly solve these equations.  These last $\alpha$ equations then have no errors and can be used to randomize the instance to produce a random matrix without augmenting the error in the $b$ vector.
\begin{figure}
\begin{enumerate}
\item Input $A', b' = A'x' +e'$.
\item Randomly generate $F$.
\item Set $b = (b'_1,..., b'_n, \$, ..., \$)$ that is set the first $n$ values of $b$ equal to $b'$ and randomly generate the remainder.
\item Initialize $A = \left(\begin{array}{c | c}C = A' & D = \mathbf{0}\\\hline F & G = I\end{array}\right)$.
\item For $i=1...m+\alpha$, perform the following randomization:\label{step:randomizationSearch}
\subitem Randomly generate $\gamma_{i,1},.., \gamma_{i,\alpha}$.
\subitem Set $A_{i, \cdot} = A_{i, \cdot} +\sum_{j=1}^\alpha \gamma_{i,j}A_{m+j, \cdot}$.
\subitem Set $b_i = b_i +\sum_{j=1}^\alpha \gamma_{i,j} b_j$.
\item Run $\mathcal{A}$ on input $A, b$.
\item Receive output $x$ from $\mathcal{A'}$
\item Output $x' = x_1,..., x_n$.
\end{enumerate}
\caption{Code for $\mathcal{A'}$ to generate LWE w/ symbol fixing source from standard LWE instance}
\label{fig:imperfectLWEreduction}
\end{figure}
\begin{claim}\label{cl:randomMatrix}
The matrix $A$ generated after Step \ref{step:randomizationSearch} is a random matrix.
\end{claim}
\begin{proof}
First, the submatrices $D, G$ are clearly truly random after step \ref{step:randomizationSearch}.  For the matrix $D, D_{i, j}$ is assigned the value $\gamma_{(i,j-n)}$.  Similarly, for the matrix $G$, each entry is assigned as $\gamma_{(i-n, j-n)}$.  This is due to the original construction of the matrix $G$.  Then before step \ref{step:randomizationSearch} the matrices $C, F$ are random submatrices ($C=A'$ by assumption and $F$ by construction).  Thus, it remains to show that $C, F$ remain random after step 5.  We consider a single row vector of $C$.  The new row vector $C_{(i,\cdot)} = C_{(i,\cdot)} +\sum_{j=1}^\alpha \gamma_{i,j}F_{(j, \cdot)}$.  That is, $C_{(i,\cdot)}$ is the sum of a random vector and a random linear combination independent vectors.  That is, $C_{(i,\cdot)}$ is a random vector.  

Similarly, consider a single row vector $F$. The new row vector $F_{(i,\cdot)} = \sum_{j=1}^\alpha \gamma_{i+n,j}F_{(i,\cdot)}$ is a random linear combination of independent vectors and thus truly random (since all $\gamma$ are chosen independently and randomly).  \bnote{I feel these claims are obvious but I would like to have a citation.}

Thus, the entire matrix $A=\left(\begin{array}{c | c}C & D \\\hline F & G\end{array}\right)$ is a truly random matrix.  This completes the claim.
\end{proof}
\begin{claim}\label{cl:goodLWEinstance}
The inputs $A, b$ provided to $\mathcal{A}$ is a random LWE w/ symbol fixing sources instance (provided that $A', b'$ was a random instance).
\end{claim}
\begin{proof}
By \clref{cl:randomMatrix} $A$ provided to $\mathcal{A}$ is a truly random matrix.  Thus, we must show that $b$ is of the proper form.  That is, we must show there exists an $x$ such that $b_i = A_{(i , \cdot)} x + e_i$ for $i=1,...,m$ and $b_i = A_{(i, \cdot)} x$ for $i=m+1,...,m+\alpha$.

By assumption there exists an $x'$ such that $A'x'+e' = b'$.  Before step \ref{step:randomizationSearch}, we have added $\alpha$ equations as well as $\alpha$ unknowns.  This means there no solutions have been eliminated.  Namely, $x^*_i = x'_i$ for $i=1,...,n$ and $x^*_i = b_i - \sum_{j=1}^m F_{i, j} x_j'$ for $i=n+1,..., n+\alpha$.  The case where $F$ does not have full row rank is also handled because the $G$ is the identity matrix and thus $F|G$ always has full row rank.  Furthermore, note that the last $\alpha$ equations have no error.  Thus, before step \ref{step:randomizationSearch} there exists a solution $x^*$ of the proper form.  It remains to show that the randomization in step \ref{step:randomizationSearch} retains the solution $x^*$.

The newly randomized system $A,b$ retains the same solution $x^*$  because we are not adding rows that have any error terms.  More precisely, let $A^*, b^*$ be the system before randomization (where $A^*x^*+e^* = b^*$).  Then for each row vector 
\begin{align*}
A_{(i, \cdot)} &= A^*_{(i, \cdot)}+\sum_{j=1}^\alpha \gamma_{i, j} A^*_{(m+j, \cdot)}, \\b_i &= b^*_i + \sum_{j=1}^\alpha \gamma_{i,j}b_j^*.
\end{align*}
Thus, we have the following for all $i=1,..., m$:
\begin{align*}
A_{(i, \cdot)} x^* +e_i^*&= (A^*_{(i, \cdot)}+\sum_{j=1}^\alpha \gamma_{i, j} A^*_{(m+j, \cdot)})x^*+e_i^*\\
&= A^*_{(i, \cdot)}x^* + \sum_{j=1}^\alpha \gamma_{i,j} A^*_{(m+j, \cdot)}x^*+e_i^*\\
&=\left( A^*_{(i, \cdot)}x^* + e_i^*\right)+ \sum_{j=1}^\alpha \gamma_{i,j} A^*_{(m+j, \cdot)}x^* \\
&= b^*_i + \sum_{j=1}^\alpha \gamma_{i,j}b_j^* = b_i
\end{align*}

The same equations follow without the term $e_i^*$ for $i=m+1,..., m+\alpha$.  
Thus, $x^*$ is a solution to the randomized LWE instance as well.  It remains to show that $x^*$ is a random vector.  This is clear for $x^*_i, i=1,...,n$.  For the remaining $x^*_i, i=m+1, ..., m+\alpha$ this follows since $x^*_i = b_i - \sum_{j=1}^m F_{i, j} x_j'$ where $b_i, F_{i, j}$ are randomly and independently chosen.  This completes the claim.
\end{proof}
\begin{claim}
If $x$ is a ``good'' solution output by $\mathcal{A}$ then $x'_i = x_i$ for $i=1,...,n$.  
\end{claim}
This claim follows by the analysis in \clref{cl:goodLWEinstance}.  The added equations and randomization do not affect the solution to the original $A', A'x'+e'$ instance provided.
\begin{claim}
$\mathcal{A'}$ outputs a ``good'' solution with probability only negligibly different  than the probability $\mathcal{A}$ outputs a good solution.
\end{claim}
The only time that $\mathcal{A'}$ does not produce the correct distribution is when the rows of $F$ are not linearly independent.  However, this occurs with negligible probability~\cite{something}.

Thus, we have constructed an algorithm $\mathcal{A'}$ that is able to recover a secret $x'$ of a standard LWE instance by adding samples  and equations that have no error.  Notice it was critical to add a variable for each new sample being provided to $\mathcal{A}$.  Furthermore, adding a degree of freedom for each new sample also allowed for construction of an equation with no error.  This completes the proof.

\end{proof}

\ignore{
\section{Parameters for Construction~\ref{cons:LWESecureSketch} when $q=2^n$}
\label{sec:parameters q expo}
The results of~\lemref{lem:uniform LWE decision} only support $q = \poly(n)$.  However, the reductions of Peikert may extend~\lemref{lem:uniform LWE decision} to exponential $q$.  

$q = O(2^n)$. In this setting, we are no longer constrained by field size.  We need $\rho$ be large enough that the underlying lattice problems are still hard to approximate within a $\tilde{O}(n^{5/2}m/\rho)$ factor.  Unfortunately, even correcting a constant $c$ number of errors is difficult in this domain.  Let $\gamma$ be the maximum approximation factor where the underlying lattices problems are still hard.  Then, $\rho \leq \frac{n^{5/2}m}{\gamma}$.  To correct $c$ errors we need that $\log q/\log \rho q \geq c$.  That is,
\begin{align*}
c&\leq \frac{\log q}{\log \rho q}\\
&\leq \frac{n}{\log \frac{n^{5/2}mq}{\gamma}}\\
&\leq \frac{n}{\log \frac{n^{5/2}m2^n}{\gamma}}\\
&\leq \frac{n}{n + \log n + \log m - \log \gamma}
\end{align*}
Ignoring the $\log n$ and $\log m$ terms this yields that:
\begin{align*}
c&\leq \frac{n}{n-\log \gamma}\\
\frac{n(c-1)}{c}&\leq \log \gamma\\
2^{n(c-1)/c} &\leq \gamma
\end{align*}
Thus, supporting even a constant number of errors for $q=2^n$ requires lattice problems are exponentially hard to approximate, which would be quite surprising.  
}
}
\end{document}











