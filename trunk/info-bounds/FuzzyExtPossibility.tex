\documentclass[11pt]{article}
\def\shownotes{1}

\usepackage[top=3cm, bottom=3cm, left=2cm, right=2cm]{geometry}      % [top=2cm, bottom=2cm, left=2cm, right=2cm]
\geometry{letterpaper}                   % ... or a4paper or a5paper or ...
%\geometry{landscape}                % Activate for for rotated page geometry
%\usepackage[parfill]{parskip}
\usepackage{graphicx}
\usepackage{amssymb, amsmath, amsfonts}
\usepackage{amsthm}
\usepackage{enumerate}
\usepackage{hyperref}
\usepackage{xspace}
\usepackage{graphicx}
\usepackage{latexsym}
\usepackage{color}
\usepackage{framed}
\usepackage{algpseudocode}

\mathchardef\mhyphen="2D

\newcommand{\secref}[1]{\mbox{Section~\ref{#1}}}
\newcommand{\subsecref}[1]{\mbox{Subsection~\ref{#1}}}
\newcommand{\apref}[1]{\mbox{Appendix~\ref{#1}}}
\newcommand{\thref}[1]{\mbox{Theorem~\ref{#1}}}
\newcommand{\exref}[1]{\mbox{Example~\ref{#1}}}
\newcommand{\defref}[1]{\mbox{Definition~\ref{#1}}}
\newcommand{\corref}[1]{\mbox{Corollary~\ref{#1}}}
\newcommand{\lemref}[1]{\mbox{Lemma~\ref{#1}}}
\newcommand{\assref}[1]{\mbox{Assumption~\ref{#1}}}
\newcommand{\probref}[1]{\mbox{Problem~\ref{#1}}}
\newcommand{\clref}[1]{\mbox{Claim~\ref{#1}}}
\newcommand{\propref}[1]{\mbox{Proposition~\ref{#1}}}
\newcommand{\remref}[1]{\mbox{Remark~\ref{#1}}}
\newcommand{\consref}[1]{\mbox{Construction~\ref{#1}}}
\newcommand{\figref}[1]{\mbox{Figure~\ref{#1}}}
\DeclareMathOperator*{\expe}{\mathbb{E}}
\DeclareMathOperator*{\var}{\text{Var}}


\newcommand{\class}[1]{{\ensuremath{\mathsf{#1}}}}
\newcommand{\gen}{\ensuremath{\class{Gen}}\xspace}
\newcommand{\rep}{\ensuremath{\class{Rep}}\xspace}
\newcommand{\sketch}{\ensuremath{\class{SS}}\xspace}
\newcommand{\rec}{\ensuremath{\class{Rec}}\xspace}
\newcommand{\enc}{\ensuremath{\class{Enc}}\xspace}
\newcommand{\dec}{\ensuremath{\class{Dec}}\xspace}
\newcommand{\prg}{\ensuremath{\class{prg}}\xspace}
\newcommand{\zo}{\ensuremath{\{0, 1\}}}
\newcommand{\vect}[1]{\ensuremath{\mathbf{#1}}}
\newcommand{\zq}{\ensuremath{\mathbb{Z}_q}}
\newcommand{\Fq}{\ensuremath{\mathbb{F}_q}}
\newcommand{\sample}{\ensuremath{\class{Sample}}\xspace}
\newcommand{\neigh}{\ensuremath{\class{Neigh}}\xspace}
\newcommand{\dis}{\ensuremath{\mathsf{dis}}}
\newcommand{\decode}{\ensuremath{\mathsf{Decode}}}
\newcommand{\guess}{\mathsf{guess}}


\newcommand{\A}{\mathcal{A}}


\newcommand{\metric}{\ensuremath{\mathtt{Metric}}\xspace}
\newcommand{\hill}{\ensuremath{\mathtt{HILL}}\xspace}
\newcommand{\hillrlx}{\ensuremath{\mathtt{HILL\mhyphen rlx}}\xspace}
\newcommand{\yao}{\ensuremath{\mathtt{Yao}}\xspace}
\newcommand{\unp}{\ensuremath{\mathtt{unp}}\xspace}
\newcommand{\unprlx}{\ensuremath{\mathtt{unp\mhyphen rlx}}\xspace}
\newcommand{\metricstar}{\ensuremath{\mathtt{Metric}^*}\xspace}
\newcommand{\metricd}{\ensuremath{\mathtt{Metric}^*\mathtt{-d}}\xspace}
\newcommand{\hillstar}{\ensuremath{\mathtt{HILL}^*}\xspace}
\newcommand{\hillprime}{\ensuremath{\mathtt{HILL'}}\xspace}
\newcommand{\metricprime}{\ensuremath{\mathtt{Metric'}}\xspace}
\newcommand{\metricprimestar}{\ensuremath{\mathtt{Metric'}^*}\xspace}
\newcommand{\hillprimestar}{\ensuremath{\mathtt{HILL'}^*}\xspace}
\newcommand{\poly}{\ensuremath{\mathtt{poly}}\xspace}
\newcommand{\rank}{\ensuremath{\mathtt{rank}}\xspace}
\newcommand{\ngl}{\ensuremath{\mathtt{ngl}}\xspace}
\newcommand{\Hoo}{\mathrm{H}_\infty}
\newcommand{\Hav}{\tilde{\mathrm{H}}_\infty}
\newcommand{\Hfuzz}{\mathrm{H}^{\mathtt{fuzz}}_{t,\infty}}
\newcommand{\Huse}{\mathrm{H}_{\mathtt{usable}}}
\newcommand{\Dom}{\mathsl{Dom}}
\newcommand{\Range}{\mathsl{Rng}}
\newcommand{\Keys}{\mathsl{Keys}}
\def\col{\mathrm{Col}}

\newcommand{\ddetbin}{\ensuremath{\mathcal{D}^{det,\{0,1\}}}}
\newcommand{\drandbin}{\ensuremath{\mathcal{D}^{rand,\{0,1\}}}}
\newcommand{\ddetrange}{\ensuremath{\mathcal{D}^{det,[0,1]}}}
\newcommand{\drandrange}{\ensuremath{\mathcal{D}^{rand,[0,1]}}}

\newcommand{\expinfo}{\ensuremath{\mathcal{E}}}
\newcommand{\ext}{\ensuremath{\mathtt{ext}}}
\newcommand{\cext}{\ensuremath{\mathtt{cext}}}
\newcommand{\rext}{\ensuremath{\mathtt{rext}}}
\newcommand{\cons}{\ensuremath{\mathtt{cons}}}
\newcommand{\decons}{\ensuremath{\mathtt{decons}}}


\newcommand{\lwe}{\class{LWE}}
\newcommand{\LWE}{\class{LWE}}
\newcommand{\distLWE}{\ensuremath{\class{dist\mbox{-}LWE}}}

\newtheorem{theorem}{Theorem}[section]
\newtheorem{lemma}[theorem]{Lemma}
\newtheorem{proposition}[theorem]{Proposition}
\newtheorem{corollary}[theorem]{Corollary}
\newtheorem{definition}[theorem]{Definition}
\newtheorem{assumption}[theorem]{Assumption}
\newtheorem{claim}[theorem]{Claim}
\newtheorem{problem}[theorem]{Problem}
\newtheorem{construction}[theorem]{Construction}

\newcounter{ctr}
\newcounter{savectr}
\newcounter{ectr}

\newenvironment{newitemize}{%
\begin{list}{\mbox{}\hspace{5pt}$\bullet$\hfill}{\labelwidth=15pt%
\labelsep=5pt \leftmargin=20pt \topsep=3pt%
\setlength{\listparindent}{\saveparindent}%
\setlength{\parsep}{\saveparskip}%
\setlength{\itemsep}{3pt} }}{\end{list}}


\newenvironment{newenum}{%
\begin{list}{{\rm (\arabic{ctr})}\hfill}{\usecounter{ctr} \labelwidth=17pt%
\labelsep=5pt \leftmargin=22pt \topsep=3pt%
\setlength{\listparindent}{\saveparindent}%
\setlength{\parsep}{\saveparskip}%
\setlength{\itemsep}{2pt} }}{\end{list}}

\newenvironment{tiret}{%
\begin{list}{\hspace{2pt}\rule[0.5ex]{6pt}{1pt}\hfill}{\labelwidth=15pt%
\labelsep=3pt \leftmargin=22pt \topsep=3pt%
\setlength{\listparindent}{\saveparindent}%
\setlength{\parsep}{\saveparskip}%
\setlength{\itemsep}{2pt}}}{\end{list}}


\newenvironment{blocklist}{\begin{list}{}{\labelwidth=0pt%
\labelsep=0pt \leftmargin=0pt \topsep=10pt%
\setlength{\listparindent}{\saveparindent}%
\setlength{\parsep}{\saveparskip}%
\setlength{\itemsep}{20pt}}}{\end{list}}

\newenvironment{blocklistindented}{\begin{list}{}{\labelwidth=0pt%
\labelsep=30pt \leftmargin=30pt\topsep=5pt%
\setlength{\listparindent}{\saveparindent}%
\setlength{\parsep}{\saveparskip}%
\setlength{\itemsep}{10pt}}}{\end{list}}

\newenvironment{onelist}{%
\begin{list}{{\rm (\arabic{ctr})}\hfill}{\usecounter{ctr} \labelwidth=18pt%
\labelsep=7pt \leftmargin=25pt \topsep=2pt%
\setlength{\listparindent}{\saveparindent}%
\setlength{\parsep}{\saveparskip}%
\setlength{\itemsep}{2pt} }}{\end{list}}

\newenvironment{twolist}{%
\begin{list}{{\rm (\arabic{ctr}.\arabic{ectr})}%
\hfill}{\usecounter{ectr} \labelwidth=26pt%
\labelsep=7pt \leftmargin=33pt \topsep=2pt%
\setlength{\listparindent}{\saveparindent}%
\setlength{\parsep}{\saveparskip}%
\setlength{\itemsep}{2pt} }}{\end{list}}

\newenvironment{centerlist}{%
\begin{list}{\mbox{}}{\labelwidth=0pt%
\labelsep=0pt \leftmargin=0pt \topsep=10pt%
\setlength{\listparindent}{\saveparindent}%
\setlength{\parsep}{\saveparskip}%
\setlength{\itemsep}{10pt} }}{\end{list}}

\newenvironment{newcenter}[1]{\begin{centerlist}\centering%
\item #1}{\end{centerlist}}

\newenvironment{codecenter}[1]{\begin{small}\begin{centerlist}\centering%
\item #1}{\end{centerlist}\end{small}}

\ifnum\shownotes=1
\newcommand{\authnote}[2]{{\textcolor{red}{\textsf{#1 notes: }\textcolor{blue}{ #2}}\marginpar{\textcolor{red}{\textbf{!!!!!}}}}}
\else
\newcommand{\authnote}[2]{}
\fi
\newcommand{\bnote}[1]{{\authnote{Ben}{#1}}}
\newcommand{\lnote}[1]{{\authnote{Leo}{#1}}}
\newcommand{\rnote}[1]{{\authnote{Ran}{#1}}}
\newcommand{\onote}[1]{{\authnote{Omer}{#1}}}

\newcommand{\ve}{\vect{e}}
\newcommand{\vm}{\vect{m}}
\newcommand{\vy}{\vect{y}}
\newcommand{\vE}{\vect{E}}
\newcommand{\vS}{\vect{S}}
\newcommand{\vA}{\vect{A}}
\newcommand{\vc}{\vect{c}}
\newcommand{\vW}{\vect{W}}
\newcommand{\vQ}{\vect{Q}}
\newcommand{\vR}{\vect{R}}
\newcommand{\vU}{\vect{U}}
\newcommand{\vT}{\vect{T}}
\newcommand{\vX}{\vect{X}}
\newcommand{\vB}{\vect{B}}
\newcommand{\vz}{\vect{z}}
\newcommand{\vd}{\vect{d}}
\newcommand{\vs}{\vect{s}}
\newcommand{\vx}{\vect{x}}
\newcommand{\va}{\vect{a}}
\newcommand{\vb}{\vect{b}}
\newcommand{\vgamma}{\mathbf{\Gamma}}
\newcommand{\vt}{\vect{t}}
\newcommand{\vu}{\vect{u}}
\newcommand{\vF}{\vect{F}}
\newcommand{\recout}{x}
\newcommand{\ignore}[1]{}
\newcommand{\M}{\mathcal{M}}
\newcommand{\Vol}{\mathsf{Vol}}

\title{When are Fuzzy Extractors Possible?}


\begin{document}
\maketitle

\begin{abstract}
We characterize when deriving a stable key from a noisy source is possible.  We use the notation of fuzzy extractors introduced by Dodis et al. (Eurocrypt 2004).  The goal is to produce a strong $key$ from a $w$ drawn from a \emph{strong} distribution $W$ and to produce the same when given a nearby $w'$.  A fuzzy extractor is split into two algorithms $\gen$ which produces $key$ along with some helper information $p$ and $\rep$ which takes $w'$ and $p$ to reproduce $key$.  The key must remain strong in the presence of $p$.  The goal is to produce the correct key whenever $\dis(w, w')\le t$.

Ideally, key derivation should be possible whenever a negligible portion of $W$ lies within a ball~(arbitrarily centered) of radius $t$.  We call the maximum overall weight that lies within any ball of radius $t$ the fuzzy min-entropy of a distribution.  However, current techniques for constructing fuzzy extractors lie far from this necessary condition.  We investigate whether this is necessary for both fuzzy extractors and secure sketches~(a one round information-reconciliation component).  Our results are centered on feasibility and our constructions and impossibility results are primarily information-theoretic.
\end{abstract}

\section{Introduction}
Usually, authentication requires some high quality secret.  Traditionally, this secret is assumed to have high min-entropy.  However, many sources with sufficient entropy for authentication present an additional problem of noise.  That is, when the physical instantiation of the source is read multiple times, readings are close~(according to some metric) but not identical.  To use such sources in an authentication scheme it is often necessary to remove this noise and derive the same key from the initial and subsequent readings.

This problem was introduced in the seminal work of Bennett, Brassard, and Roberts~\cite{bennett1988privacy}.  They identify two fundamental tasks in noisy key derivation.  The first is known as information-reconciliation, removing the noise without leaking significant information, and the second is privacy amplification, converting the high entropy secret to a uniform random value.  In this work, we consider the non-interaction version of these problems where these tasks are performed non-interactively~(with a single message).

The paradigm for performing noisy key derivation using a single metric is known as fuzzy extractors~\cite{DBLP:journals/siamcomp/DodisORS08}.  Consider a particular high entropy distribution $W$.  Our goal is to derive stable keys from $W$ whenever the two readings~(denoted $w$ and $w'$ respectively) are within distance $t$.  A fuzzy extractor consists of two algorithms: generate~($\gen$) which takes the initial reading $w$ and produces $key$ along with a public helper value $p$.  The reproduce~($\rep$) algorithm takes the subsequent reading $w'$ along with the helper value $p$ to reproduce $key$.  The $key$ must be strong even to an adversary that has observed $p$~(the problem is trivial if $p$ is private).

Traditionally, fuzzy extractors are constructed by using a separate information-reconciliation component, known as a secure sketch, that maps $w'$ back to $w$ and a privacy-amplification component, a strong randomness extractor~\cite{nisan1993randomness}, that maps $w$ to a random key.  These objects can either be information-theoretic or computational.  The computational versions of these objects are defined in~\cite{krawczyk2010cryptographic} and~\cite{fuller2013computational} respectively.  Fuller et al.~\cite{fuller2013computational} show that computational secure sketches cannot improve significantly on information-theoretic secure sketches.

\textbf{A Necessary Condition:}  Intuitively, error-tolerance and key strength are at odds.  If the error-tolerance allows the entire metric space to authenticate then no security is possible.  As an example, consider a distribution $W$, where the adversary attempts to learn the key simply by inputting a point to the reproduce function.  Let $x^*$ be the point input by the adversary, the adversary learns the correct key whenever $\dis(w, x^*)\le t$.  For a fuzzy extractor to be secure this must occur with negligible probability.  That is, for all $x^*, \sum_{w\in W | \dis(w, x^*)\le t} \Pr[ W= w] \le \ngl(n)$.  We call the negative logarithm of this quantity the \emph{fuzzy min-entropy} of a distribution.  Clearly, the fuzzy min-entropy of a distribution must be super-logarithmic for any security.  Indeed, in the interactive setting this is a sufficient condition as well.  

However, in the one-round setting we are far from achieving security for all distributions with super-logarithmic fuzzy min-entropy.  Traditional fuzzy extractors and secure sketches incur at entropy loss of the size of the ball to be error-corrected.  While, this is optimal for the uniform distribution, it seems far from optimal for distributions where points are far apart.  Indeed, when the goal is provide key derivation from a source that has more errors than entropy, this bound provides no guarantee on the strength of the key.  The work of Canetti et al.~\cite{canetti2014key} shows it is possible to provide key derivation for some distributions of this type.  However, their work leaves the question open for a large number of distributions.  In this work, we attempt to settle this question: for what distributions is noisy key derivation possible and impossible?  \bnote{come back to this}

\textbf{Our results:} In this work we show awesome things!\bnote{write}

\section{Preliminaries}
\label{sec:preliminaries}
For a random variables $X_i$ over some alphabet $\mathcal{Z}$ we denote by $X = X_1,..., X_\ell$  the tuple $(X_1,\dots, X_\ell)$.  For a set of indices $J$, $X_{J}$ is the restriction of $X$ to the indices in $J$.  The set $J^c$ is the complement of $J$.  The {\em min-entropy} of $X$ is $\Hoo(X) = -\log(\max_x \Pr[X=x])$,
and the {\em average (conditional)} min-entropy of $X$ given $Y$ is  $\Hav(X|Y) = -\log(\expe_{y\in Y} \max_{x} \Pr[X=x|Y=y])$~\cite[Section 2.4]{DBLP:journals/siamcomp/DodisORS08}.   For a random variable $W$, let $H_0(W)$ be the logarithm of the size of the support of $W$,  that is $H_0(W) = \log |\{w | \Pr[W=w]>0\}|$.
The {\em statistical distance} between random variables $X$ and $Y$ with the same domain is $\Delta(X,Y) = \frac12 \sum_x |\Pr[X=x] - \Pr[Y=x]|$.
For a distinguisher $D$ we write the \emph{computational distance} between $X$ and $Y$ as $\delta^D(X,Y) = \left| \expe[D(X)]-\expe[D(Y)]\right |$ (we extend it to a class of distinguishers $\mathcal{D}$ by taking the maximum over all distinguishers $D\in\mathcal{D}$).  We denote by $\mathcal{D}_{s}$ the class of randomized circuits which output a single bit and have size at most $s$.

For a metric space $(\mathcal{M}, \dis)$, the \emph{(closed) ball of radius $t$ around $x$} is the set of all points within radius $t$, that is, $B_t(x) = \{y| \dis(x, y)\leq t\}$.  If the size of a ball in a metric space does not depend on $x$, we denote by $|B_t|$ the size of a ball of radius $t$.  We consider the Hamming metric over vectors in $\mathcal{Z}^\ell$, defined via $\dis(x,y) = \{i | x_i \neq y_i\}$.  For this metric, $|B_t| = \sum_{i=0}^t {\ell \choose i} (|\mathcal{Z}|-1)^i $.  $U_n$ denotes the uniformly  distributed random variable on $\{0,1\}^n$.  Unless otherwise noted logarithms are base $2$.
Usually, we use capitalized letters for random variables and corresponding lowercase letters for their samples.

\section{Fuzzy Extractors}\label{sec:fuzz extractor}

We now recall definitions and lemmas from the work of Dodis et. al.~\cite[Sections 2.5--4.1]{DBLP:journals/siamcomp/DodisORS08}, adapted to allow for a small probability of error, as discussed in \cite[Sections 8]{DBLP:journals/siamcomp/DodisORS08}.  Let $\mathcal{M}$ be a metric space with distance function $\dis$.

\begin{definition}
\label{def:fuzzy extractor}
An $(\mathcal{M}, m, \ell, t, \epsilon)$-\emph{fuzzy extractor} with error $\delta$ is a pair of randomized procedures, ``generate'' $(\gen)$ and ``reproduce'' $(\rep)$, with the following properties: 
\begin{enumerate}
\item The generate procedure \gen on input $w\in \mathcal{M}$ outputs an extracted string $r\in\{0,1\}^\ell$ and a helper string $p\in\{0,1\}^*$.
\item The reproduction procedure \rep takes an element $w'\in \mathcal{M}$ and a bit string $p\in\{0,1\}^*$ as inputs.  The \emph{correctness} property of fuzzy extractors guarantees that for $w$ and $w'$ such that $\dis(w,w')\leq t$, if $R,P$ were generated by $(R,P)\leftarrow\gen(w)$, then $\rep(w',P)=R$ with probability~(over the coins of $\gen, \rep$) at least $1-\delta$.  If $\dis(w,w')>t$, then no guarantee is provided about the output of \rep.
\item The \emph{security} property guarantees that for any distribution $W$ on $\mathcal{M}$ of min-entropy $m$, the string $R$ is nearly uniform even for those who observe $P$:  if $(R,P)\leftarrow\gen (W)$, then $\mathbf{SD}((R,P),(U_\ell,P))\leq \epsilon$.
\end{enumerate}
A fuzzy extractor is efficient if $\gen$ and $\rep$ run in expected polynomial time.
\end{definition}

Replacing the statistical distance in \defref{def:fuzzy extractor} with a computational distance results in a computational fuzzy extractor~\cite[Definition 2.5]{fuller2013computational}.

Secure sketches are the main technical tool in the construction of fuzzy extractors.  Secure sketches produce a string $s$ that does not decrease the entropy of $w$ too much, while allowing recovery of $w$ from a  close $w'$:
\begin{definition}
\label{def:secure sketch}
An $(\mathcal{M},m, \tilde{m}, t)$-\emph{secure sketch} with error $\delta$ is a pair of randomized procedures, ``sketch'' $(\sketch)$ and ``recover'' $(\rec)$, with the following properties:
\begin{enumerate}
\item The sketching procedure \sketch on input $w\in\mathcal{M}$ returns a bit string $s\in\{0,1\}^*$.
\item The recovery procedure \rec takes an element $w'\in\mathcal{M}$ and a bit string $s\in\{0,1\}^*$.  The \emph{correctness} property of secure sketches guarantees that if $\dis(w,w')\leq t$, then $\Pr[\rec(w',\sketch(w))=w]\geq 1-\delta$ where the probability is taken over the coins of $\sketch$ and $\rec$.  If $\dis(w,w')>t$, then no guarantee is provided about the output of \rec.
\item The \emph{security} property guarantees that for any distribution $W$ over $\mathcal{M}$ with min-entropy $m$, the value of $W$ can be recovered by the adversary who observes $w$ with probability no greater than $2^{-\tilde{m}}$.  That is, $\Hav(W|\sketch(W))\geq \tilde{m}$.
\end{enumerate}
A secure sketch is \emph{efficient} if \sketch and \rec run in expected polynomial time. 
\end{definition}

\textbf{Notes:} In the above definition of secure sketches (resp., fuzzy extractors), the errors are chosen before $s$ (resp., $P$) is known: if the error pattern between $w$ and $w'$ depends on the output of $\sketch$ (resp., $\gen$), then there is no guarantee about the probability of correctness.  Also we do not consider a computational version of secure sketches as Fuller et al. showed that computational secure sketches imply information-theoretic secure sketches with almost the same parameters~\cite[Corollary 3.8]{fuller2013computational}.


A fuzzy extractor can be produced from a \emph{secure sketch} and an \emph{average-case randomness extractor}. An average-case extractor is a generalization of a strong randomness extractor \cite[Definition 2]{nisan1993randomness}) (in particular, Vadhan~\cite[Problem 6.8]{Vad12} showed that all strong extractors are average-case extractors with a slight loss of parameters):
\begin{definition}
Let $\chi_1$, $\chi_2$ be finite sets.
A function $\ext: \chi_1\times \{0,1\}^d \rightarrow \{0,1\}^\ell$ a \emph{$(m, \epsilon)$-average-case extractor} if for all pairs
of random variables $X, Y$ over $\chi_1, \chi_2$ such that
$\tilde{H}_\infty(X|Y) \ge m$, we have $\Delta((\ext(X, U_d), U_d, Y), U_\ell\times
U_d \times Y) \le \epsilon$.
\end{definition}

\begin{lemma}
\label{lem:fuzzy ext construction}
Assume $(\sketch, \rec)$ is an $(\mathcal{M}, m, \tilde{m}, t)$-secure sketch with error $\delta$, and let $\ext:\mathcal{M}\times \zo^d \rightarrow \zo^\ell$ be a $(\tilde{m}, \epsilon)$-average-case extractor.  Then the following $(\gen, \rep)$ is an $(\mathcal{M}, m, \ell, t, \epsilon)$-fuzzy extractor with error $\delta$:
\begin{itemize}
\item $\gen(w):$ generate $x\leftarrow \zo^d$, set $p=(\sketch(w), x), r=\ext(w;x)$, and output $(r,p)$.
\item $\rep(w', (s, x)):$ recover $w=\rec(w',s)$ and output $r=\ext(w;x)$.
\end{itemize}
\end{lemma}

\subsection{A necessary condition for security}
\label{sec:minimal conditions}
We now define fuzzy min-entropy and show that fuzzy extractor security is only possible when the fuzzy min-entropy is super-logarithmic.



A necessary condition for fuzzy extractor security is that an adversary should not be able to learn the key simply by inputting a point into the \rep algorithm.  This means a negligible portion of the source distribution $W$ lies within any Hamming ball.  We make this intuition formal here:

\begin{definition}
\label{def:fuzzy min-ent}
A distribution $W$ in a metric space $(\mathcal{M}, \dis)$ has $(t, k)$-fuzzy min-entropy, denoted $\Hfuzz(W) \ge k$ if the following holds:
\[
\forall m\in \mathcal{M},  \sum_{w\in W | \dis(w, m)\le t} Pr[W=w] \leq 2^{-k}.
\]
\end{definition}
For a metric space $\mathcal{M}$, let $\max |\mathcal{M}|$ the maximum length required to describe an element $m\in\mathcal{M}$~(for most natural metric spaces this is $\log |\mathcal{M}|$).
\begin{lemma}
\label{lem:fuzz necessary}
Let $n$ be a security parameter and let $W$ be a distribution over $(\mathcal{M}, \dis)$.
If $\Hfuzz (W) = \Theta(\log n)$ there is no $(\mathcal{M}, W, \kappa, t)$-computational fuzzy extractor that is $(\max |\mathcal{M}| +  |\rep|, \epsilon)$-hard for $\epsilon = \ngl(n)$ with error $\delta = \ngl(n)$~(and thus no fuzzy extractor) for $\kappa =\omega(\log n)$.
\end{lemma}
\begin{proof}
Let $W$ be a distribution where $\Hfuzz(W) = \Theta(\log n)$.  This means that there exists a point $m\in \mathcal{M}$ such that $\Pr_{w\in W}[\dis (w, m)\leq t] \geq 1/\poly(n)$.  Consider the following distinguisher $D$:
\begin{itemize}
\item On input $r, p$.
\item If $\rep(m, p) = r$, output $1$.
\item Else output $0$.
\end{itemize}
First note that $|D|$ is of size $\max |\mathcal{M}|+ |\rep|$.  Clearly, $\Pr[D(R, P) = 1]\geq 1/\poly(n) - \delta$, while $\Pr[D(U_\kappa, P)=1 ]\leq 1/2^{-\kappa}$.  Thus, when $\kappa = \omega(\log n)$:
\[
\delta^D((R, P), (U_\kappa, P))\geq \frac{1}{\poly(n)} -\delta -  \frac{1}{2^{-\kappa}} = 1/\poly(n).
\]
\end{proof}
\lemref{lem:fuzz necessary} generalizes to interactive protocols, $D$ only provides an input to the protocol and looks at the output.  This means that fuzzy min-entropy is also a necessary condition for interactive solution.  In the interactive setting, fuzzy min-entropy is also a sufficient condition using secure two-party computation\bnote{cite something for this}.  However, in the non-interactive setting there are many distributions for which no construction is known to be secure and for which no impossibility result is known.  We attempt to bridge this gap in this work.

\section{Well-spread distributions}
\subsection{Depends on Distribution}
We will start with an intuitively simple type of distribution where no results are known.  Security should be easy when all points of $W$ are far apart.  Intuitively, the fuzzy extractor/secure sketch does not need to disambiguate points and thus error-correction should be easy.  Indeed for any error-correcting code, a fuzzy extractor can be designed~(where the design of the fuzzy extractor depends on the supported distribution).  However, this intuition becomes more difficult when a fuzzy extractor should work for all well-spread distributions.  We begin by defining a well-spread distribution.

\begin{definition}
A distribution $W$ is called \emph{$t_{sec}$-well spread} if for all $w, x\in W, \dis(w, x)\ge t_{sec}$.
\end{definition}
Intuitively, it should be easy to build a fuzzy extractor for well-spread distributions when the desired error-tolerance is less than $t_{sec}/2$~(assuming the distribution has super-logarithmic min-entropy).  Indeed, this is the case when the fuzzy extractor is allowed to depend on $W$.

\begin{lemma}
\label{lem:nosketchwellspread}
Let $W$ be a $t_{sec}$ well-spread distribution with $\Hoo(W)\ge k$ there exists a $(\mathcal{M}, m, m, t_{sec}/2)$-secure sketch with no error that works for $W$.  In particular, $\sketch(w) = \perp$, and $\rec(w')$ finds the nearest $w\in W$.  \rec is efficient if there exists efficient decoding for the points in $W$.  By \lemref{lem:fuzzy ext construction} there also exists a fuzzy extractor.\bnote{fill in the parameters}
\end{lemma}  

\subsection{Works for all well-spread distributions}
Ideally, a fuzzy extractor should work for any distribution~(or at least a large family of distributions).  However, as we show below the above construction does not extend to the setting of all well-spread distributions.  In particular, any secure sketch must write down a large number of bits~(matching our known constructions of secure sketches).  However, it is not known if it is necessary to decrease the entropy of the distribution $W$.

\begin{lemma}
Let $\mathcal{W}$ be the set of all $t_{sec}$-well spread distributions.  Let $(\sketch, \rec)$ be a $(\mathcal{M}, m, \tilde{m}, t_{cor})$-secure sketch.  Then the length of $\sketch$ is at least $\log |B_{t}|$.
\end{lemma}
\begin{proof}
Let $\mathcal{M}$ be a metric space.
Let $\mathcal{W}$ be the set of all $t_{sec}$-well spread distributions on $\mathcal{M}$.  
Consider the following experiment:
\[
W\leftarrow \mathcal{W} \wedge w \leftarrow W \wedge w'\leftarrow B_{t_{sec}}(w) \wedge ss \leftarrow \sketch(w) \wedge w^* \leftarrow \rec(w', ss).
\]
Since $(\sketch, \rec)$ works for any well-spread distribution the $\Pr[w^* = w] =1 $ in the above experiment.   Denote by $X$ the joint distribution of $w, w'$ produced in the above experiment.  Now consider the following experiment
\[
v \leftarrow U_{\mathcal{M}} \wedge v'\leftarrow B_{t_{sec}}(v) \wedge ss \leftarrow \sketch(v) \wedge v^* \leftarrow \rec(v', ss).
\]
Denote by $Y$ the joint distribution of $v, v'$ produced in the above experiment.  Then $X\overset{d}= Y$.  That is, the view of $(\sketch, \rec)$ is identically distributed in both cases.  Thus, $\Pr[v^* = v] = 1$.  For the uniform distribution the entropy retained by a secure sketch is at most $\Hav(U_{\mathcal{M}} | \sketch(U_{\mathcal{M}})) \le \log |\mathcal{M} - \log |B_t(\cdot)|$~\cite[Lemma C.1]{DBLP:journals/siamcomp/DodisORS08}.  For a secure sketch to lose $\log |B_{t_{sec}}(\cdot)|$ bits of entropy it must write down at least $\log |B_{t_{sec}}(\cdot)|$ bits~\cite[Lemma 2.2b]{DBLP:journals/siamcomp/DodisORS08}.  Thus, $(\sketch, \rec)$ writes down at least $\log |B_{t_{sec}}|$ bits on the uniform distribution and its view is identical on $X$ so it writes down at least $\log |B_{t_{sec}}|$ bits on $X$.  This completes the proof.\bnote{need to deal with the case where the number of bits written by $\sketch$ is variable.}
\end{proof}

\section{Non-well spread distributions}
\subsection{Dependence on Distributions}
\lemref{lem:nosketchwellspread} showed that if a fuzzy extractor~(resp. secure sketch) is built for a particular well-spread distribution, it is not necessary to write down information-reconciliation information.  In this section, we consider information-theoretic bounds on distributions that contain multiple points in each ball.  We start by considering distributions that have a fixed number of neighbors for each point.

\begin{definition}
We call a distribution $W$, over metric space $\mathcal{M}$, $(c, t)$-\emph{fixed neighbor} if there exists a constant $c$ such that for all points in $x\in \mathcal{M}$ there are exactly $c$ points $w\in W$ such that $\dis(w, x)\le t$.  We say that $W$ is a $(c, t)$-\emph{bounded neighbor distribution} if for all point $x\in\mathcal{M}$ there are at most $c$ points $w\in W$ such that $\dis(w, x)\le t$.
\end{definition}

\textbf{Note:} A fixed neighbor distribution is a special case of a bounded-neighbor distribution.

We now present an information-theoretic construction for fixed-neighbor distributions.  We first introduce the notion of a pairwise-independent hash:

\begin{definition}
Let $F : \mathcal{K} \times D \to R$ be a function.  We say that $F$ is \emph{universal} if for all distinct $x_1, x_2 \in D$:
\[
 \Pr_{K \leftarrow \mathcal{K}}[F(K, x_1) = F(K, x_2)] = \frac{1}{|R|} \;.
\]
%In other words, $F(K,x_1),F(K,x_2)$ are all uniformly and independently random over $R$. 
\end{definition}

We now show a sufficiently-long universal hash function suffices to construct an information-theoretic fuzzy extractor.

\begin{construction}
\label{cons:pairwise hash}
Let $W$ be a $(c, t)$-fixed neighbor distribution over $\mathcal{M}$.  Let $F :\mathcal{M}\times \mathcal{M}\rightarrow R$ be a pairwise independent hash function.  We describe $\sketch, \rec$ as follows:
\begin{itemize}
\item $\sketch(w)$:
\subitem Sample $K\leftarrow \mathcal{K}$.
\subitem Set $p = F(K, w), K$.
\item $\rec(w', y, K)$:
\subitem Enumerate $W^* = \{w \in W | \dis(w, w')\le t\}$.
\subitem For all $w^*\in W^*$, if $F(K, w^*) = y$, output $w^*$.
\subitem Output $\perp$.
\end{itemize}
\end{construction}
\begin{lemma}
\consref{cons:pairwise hash} is a $(\mathcal{M}, m, m - \log |R|, t)$-secure sketch with error $\delta = \frac{c-1}{|R|}$. 
\end{lemma}
\begin{proof}
We first argue security.  Since $\mathcal{K}$ and $W$ are independent $\Hav(W | \mathcal{K}) = \Hoo(W) = m$.  Then by \cite[Lemma 2.2b]{DBLP:journals/siamcomp/DodisORS08}, $\Hav(W | \mathcal{K}, F(K, W)) \ge \Hoo(W) - \log |F(K, W)| \ge m - \log |R|$.

We now argue correctness.  Let $W^*$ denote the set of elements in $W$ within distance $t$ of $w'$ recall that the size of $W^*$ is $c$ and since $w, w'$ are chosen independently of $\sketch$ this set is independent of the choice of $\mathcal{K}$.  Note $\rec$ will never output $\perp$ as the correct $w$ will match the hash, our goal is to bound the probability that another element $w^*$ collides, that is $F(K, w^*) = F(K, w)$.
\begin{align*}
\Pr[\exists w^* \in W^* |w^* \neq w \wedge F(K, w^*) = F(K, w)] \le \sum_{w^*\in W^* | w^*\neq w} \Pr[F(K, w^*) = F(K, w)]  = \frac{c-1}{|R|}
\end{align*}
Where the inequality proceeds by union bound and the equality proceeds by the universality of $F$.  This completes the proof.
\end{proof}
\begin{corollary}
Let $n$ be a security parameter.  
If $|R| \ge c * n^{\omega(1)}$ then \consref{cons:pairwise hash} is correct with overwhelming probability.  That is, setting $\log |R| = \log |c| + \omega(\log n)$ suffices.
\end{corollary}

Note that \consref{cons:pairwise hash} writes down essentially the worst case number of bits in all setting, it knows the maximum number of points in any ball and writes down that number~(with an additional super-logarithmic factor to prevent collisions).  Recall, our hope is to find a sketch that achieves total security equal to the fuzzy min-entropy of the distribution.  Thus, the remaining entropy for this construction is 
\[
\Hav(W |\sketch(W)) = \Hoo(W) - \log c - \omega(\log n)
\]
For a flat distribution with a fixed number of neighbors, 
\begin{align*}
\Hfuzz(W) &= -\log \max_{x\in \mathcal{M}}, \sum_{w | \dis(w, x)\le t} \Pr[W=w]\\
&=-\log \max_{x\in \mathcal{M}}, -\log \sum_{w | \dis(w, x)\le t} 2^{-\Hoo(W)}\\
&= -\log c2^{-\Hoo(W)} = \Hoo(W) - \log c
\end{align*}
Thus, the construction is optimal up to a super-logarithmic factor for flat sources with a fixed number of neighbors.  We now attempt to remove both of these restrictions.

\subsection{Flat Sources with Sketches that Write a Fixed Number of Bits}
We begin by trying to remove the restriction that every point in the metric space has a fixed number of neighbors in a distribution.  We start with sketches that write down a fixed number of bits~(independent of their input $w$).   We will show a lower bound on the number of bits that a $\sketch$ algorithm must write down.  We assume that all coins of $\sketch$ are provided to $\rec$ and that $\rec$ is deterministic~(any coins needed for $\rec$ can be flipped by $\sketch$ in this model).

\begin{lemma}
Let $W$ be a flat distribution and let $(\sketch, \rec)$ be a secure sketch for $W$.  Let $W$ be a $(c, t)$-bounded neighbor distribution.  Let $S_{coin}$ be the number of sketches that can be produced by any fixing of $coin$.  Then the error $\delta$ of $(\sketch, \rec)$ is at least $\frac{\expe_{coins}L(coins)}{c}$.  
\end{lemma}
\begin{proof}
We consider a fixed $w'$ that has $c$ neighbors in the distribution.  We assume $\rec$ has access to all coins of $\sketch$.  Furthermore, we assume that $\rec$ is deterministic and that any coins necessary are provided by $\sketch$).  

Denote by $S_{coin, R}$ the set of possible sketches for a given set of coins on all values in $R$.  Furthermore, denote by $S_{coin}$ the set of possible sketches for all values $w$~(for a given value of coin) and note that $S_{coin, R} \subset S_{coin}$.  Then $\rec(\cdot, w', coin)$ must fail to recover $|R|-|S_{coin, R}|$ fraction of $w\in R$~(recalling that $\rec$ is deterministic for this fixing of $coin$).  This means that the total number of pairs $(w, coin)$ for which $\rec$ fails to recover is $2^{|coins|}R - \sum_{coins} |S_{coin, R}|$.

This means there must exist some $w$ such that the number of $coin$ values for which $\rec$ fails to recover is 
\[
2^{|coin|} - \frac{\sum_{coin} |S_{coin, R}|}{R}.
\]
 This means that for this $w$, 
\[
\Pr_{coin}[s \leftarrow \sketch(w, coin) \wedge w = \rec(s, w', coin)] < 1- \frac{\expe |S_{coin, R}|}{|R|} = 1-\frac{\expe |S_{coin, R}|}{c} \le 1-\frac{\expe |S_{coin}| }{c}.
\]

%Consider the function $\rec(w', ss, r)$ where $ss$ is the sketch produced by $\sketch(w)$ and $r$ are the coins used by $\sketch$.  
\end{proof}

For deterministic sketches:
\[
\Hav(W|S ) = -\log \left(\sum_{s\in S} \Pr[S = s] \frac{2^{-\Hoo(W)}}{\Pr[S=s]}\right) = -\log \left(2^{-\Hoo(W)} |S| \right)  = \Hoo(W) - H_0(S).
\]
\bibliographystyle{alpha}
\bibliography{crypto}


\end{document}











