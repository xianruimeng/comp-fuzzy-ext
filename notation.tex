\documentclass[11pt]{article}

\usepackage[top=3cm, bottom=3cm, left=2cm, right=2cm]{geometry}      % [top=2cm, bottom=2cm, left=2cm, right=2cm]
\geometry{letterpaper}                   % ... or a4paper or a5paper or ... 
%\geometry{landscape}                % Activate for for rotated page geometry
\usepackage[parfill]{parskip}    
\usepackage{graphicx}
\usepackage{amssymb, amsmath, amsthm, amsfonts}
\usepackage{enumerate}
\usepackage{hyperref}
\usepackage{xspace}
\usepackage{graphicx}
\usepackage{latexsym}
\usepackage{color}
\usepackage{framed}
\usepackage{algpseudocode} 

\newcommand{\class}[1]{{\ensuremath{\mathsf{#1}}}}
\newcommand{\gen}{\ensuremath{\class{Gen}}\xspace}
\newcommand{\rep}{\ensuremath{\class{Rep}}\xspace}
\newcommand{\sketch}{\ensuremath{\class{SS}}\xspace}
\newcommand{\rec}{\ensuremath{\class{Rec}}\xspace}
\newcommand{\enc}{\ensuremath{\class{Enc}}\xspace}
\newcommand{\dec}{\ensuremath{\class{Dec}}\xspace}
\newcommand{\prg}{\ensuremath{\class{prg}}\xspace}
\newcommand{\zo}{\ensuremath{\{0, 1\}}}
\newcommand{\vect}[1]{\ensuremath{\textbf{#1}}}
\newcommand{\zq}{\ensuremath{\mathbb{Z}_q}}
\newcommand{\Fq}{\ensuremath{\mathbb{F}_q}}
\newcommand{\sample}{\ensuremath{\class{Sample}}\xspace}
\newcommand{\neigh}{\ensuremath{\class{Neigh}}\xspace}
\newcommand{\dis}{\ensuremath{\mathsf{dis}}}

\newcommand{\A}{\mathcal{A}}
\newcommand{\D}{\mathcal{D}}

\newcommand{\metric}{\ensuremath{\mathtt{Metric}}\xspace}
\newcommand{\hill}{\ensuremath{\mathtt{HILL}}\xspace}
\newcommand{\hillrlx}{\ensuremath{\mathtt{HILL\mhyphen rlx}}\xspace}
\newcommand{\yao}{\ensuremath{\mathtt{Yao}}\xspace}
\newcommand{\unp}{\ensuremath{\mathtt{unp}}\xspace}
\newcommand{\unprlx}{\ensuremath{\mathtt{unp\mhyphen rlx}}\xspace}
\newcommand{\metricstar}{\ensuremath{\mathtt{Metric}^*}\xspace}
\newcommand{\metricd}{\ensuremath{\mathtt{Metric}^*\mathtt{-d}}\xspace}
\newcommand{\hillstar}{\ensuremath{\mathtt{HILL}^*}\xspace}
\newcommand{\hillprime}{\ensuremath{\mathtt{HILL'}}\xspace}
\newcommand{\metricprime}{\ensuremath{\mathtt{Metric'}}\xspace}
\newcommand{\metricprimestar}{\ensuremath{\mathtt{Metric'}^*}\xspace}
\newcommand{\hillprimestar}{\ensuremath{\mathtt{HILL'}^*}\xspace}
\newcommand{\poly}{\ensuremath{\mathtt{poly}}\xspace}
\newcommand{\rank}{\ensuremath{\mathtt{rank}}\xspace}
\newcommand{\ngl}{\ensuremath{\mathtt{ngl}}\xspace}
\newcommand{\Hoo}{\mathrm{H}_\infty}
\newcommand{\Hav}{\tilde{\mathrm{H}}_\infty}
\newcommand{\Dom}{\mathsl{Dom}}
\newcommand{\Range}{\mathsl{Rng}}
\newcommand{\Keys}{\mathsl{Keys}}
\def\col{\mathrm{Col}}

\newcommand{\ddetbin}{\ensuremath{\mathcal{D}^{det,\{0,1\}}}}
\newcommand{\drandbin}{\ensuremath{\mathcal{D}^{rand,\{0,1\}}}}
\newcommand{\ddetrange}{\ensuremath{\mathcal{D}^{det,[0,1]}}}
\newcommand{\drandrange}{\ensuremath{\mathcal{D}^{rand,[0,1]}}}

\newcommand{\expinfo}{\ensuremath{\mathcal{E}}}
\newcommand{\ext}{\ensuremath{\mathtt{ext}}}
\newcommand{\rext}{\ensuremath{\mathtt{rext}}}
\newcommand{\cons}{\ensuremath{\mathtt{cons}}}
\newcommand{\decons}{\ensuremath{\mathtt{decons}}}


\begin{document}
Notation:
\begin{itemize}
\item $B_t(x)$ - the ball of radius $t$ around a point $x$
\item $\sketch$ - secure sketch procedure
\item $\rec$ - recover procedure for secure sketch
\item $\gen$ - fuzzy extractor generate
\item $\rep$ - fuzzy extractor reproduce
\item $w/W$ - the noisy source we are drawing from
\item $s$ - output of secure sketch
\item $S$ - the distribution W is conditioned on, we also denote this by $\sketch(W)$
\item $R$ - key from fuzzy extractor
\item $P$ - helper value for fuzzy extractor
\item $X, Y$ - these are used for the indistinguishable distributions, for the $\hill$ and $\unp$ entropy.
\item $D, \mathcal{D}$ - distinguisher and distinguisher class
\item $Z$ - the alphabet when we consider Hamming metric
\item $\ell$ - derived key length
\item $\epsilon$ - distance from uniform either for extractor or fuzzy extractor
\item $\delta$ - error rate of a fuzzy extractor or secure sketch
\item $\ext$ - extractor
\item $\rext$ - reconstructive extractor
\item $(\mathcal{M}, \dis)$ - metric space and associated distance
\item $s_{sam}$ - size of $\sample$
\item $s_{sec}$ - size of security circuit
\item $s_{rec}$ - size of recover circuit
\item $t$- bounded such that $\dis(w, w')\leq t$
\item $k$- the number of hardcore dimensions in AKV09, this is the size of our resulting key
\item $m$- starting min-entropy of source
\item $\tilde{m}$ - resulting min-entropy of a construction
\item $\vect{A}$ - random matrix in LWE
\item $n$ - security parameter for LWE
\item $q$ - size of field for LWE
\item $m$ - number of samples in LWE
\item $\vect{x} / X$ - secret vector in LWE
\item $\chi$ - arbitrary error distribution for LWE
\item $\bar{\Psi}_\alpha$ - discretized Gaussian distribution
\item $\gamma$ - Parameter in DMQ error size and lattice hardness
\item $d$ - running time of inverter, tied to number of errors that can be corrected and redundancy in m
\item $c$ - constant log errors that can be corrected
\item $\rho$ - $\rho q$ is the noise width
\item $b$ - the number of bits in a block $b = \log \rho q$
\item $\mathcal{C}$ - a set that forms a code
\item $C$ - a distribution of points that we use to form a code
\item $C, D$ - the reconstruction procedures for a reconstructive extractor
   NOTE: both of these are used above, need to change
\item $\neigh , s_{neigh}$- algorithm for sampling a neighbor
\item $\sample, s_{sam} $- algorithm for picking a random point in the metric space
\item $\vect{C}, \vect{D} $ used in the decoding algorithms of code-offset construction
\item $f$ - the degree of the polynomial used to express the size of the field q
\item $n_0$ - first point where LWE is considered secure
\item $\alpha$ - the number of fixed dimensions for a block fixing source
\item $\beta$ - the extra variables added in Theorem 5.3
\item $\gamma$ - parameter in DMQ13, slightly changes their setting, we consider $\gamma = 1/2$
\item $g$ - the fraction of n that we will output as hardcore bits
\end{itemize}

\end{document}
